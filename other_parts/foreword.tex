\chapter{序}
本书为\textit{cppHusky}的二〇二四年礼。\par
这是一本面向C++初学者的教学资料,计划分为泛讲篇和精讲篇两部分。本书第二版是泛讲篇;精讲篇内容计划于下一版推出。\par
在学习和研究C++语言的过程中,我发现,C++之所以被认为``难学'',可能有以下诸多原因:
\begin{itemize}
    \item 代码本身很繁杂。相比Python之类的语言来说,初学C++的过程尤其麻烦,因为头文件包含与命名空间的相关代码是初学者理解不了的,但是我们学C++的时候又不方便不写。
    \item C++比较底层,它没有多少顶层包装——对于老手来说,这当然有它的好处,因为C++是更注重运行效率(而非简单易学)的语言;但是对于新手而言,它也是巨大的障碍。为了真正学懂C++,我们就必须理解类型,理解指针。但是这些东西离我们的生活实在太遥远,所以初学者天然地难以理解。
    \item 学习C++需要一定的计算机基础。如果读者对于内存、程序流程、编译器等一干概念一窍不通的话,那么读者也需要在学习有关内容的同时补课。
    \item ……
\end{itemize}\par
笔者曾在几年前学过C++的相关语法,但彼时所用参考教材质量不佳,以致笔者在学习过程中走了不少弯路。后来笔者逐渐理清了部分概念,就萌生了自己写一本书出来的想法。于是本书第一版诞生了。\par
第一版的做工十分粗糙,但我还是尽自己所能把它写好。可惜,知识与信息方面的欠缺不是单纯靠``意愿''就可以弥补的。一个人倘若没有可靠的资料,仅仅依靠一些拾人牙慧的信息加上自己胡乱尝试的结果,不可能产生什么真知灼见——当年的我便是如此。\par
去年一年,我找到了更多可靠信息和权威资料,并重新进行了对C++语言系统化的学习。现在我觉得自己已经有能力再版此前那本《C++哈氏教程》了。于是,我根据这一年所学,结合此前的编程经验,并加以梳理,形成了这本《C++哈氏教程》(第二版)。\par
这本书中的许多内容都是我反复查证,并经过自己的思考而整理出来的。对于很多内容,我会在兼顾严谨性的同时,加入一些自己独到的理解。希望这能让读者更好地掌握C++的相关知识,并且避免像笔者当年那样走许多弯路。\par
丰富的文后注释是本书的一大特色。我将那些补充知识以注释的方式添加到脚注当中,可供有意深入挖掘相关内容的读者参考。\par
我还为本书内容绘制了不少插图,希望这些插图可以帮助读者更好地理解书中所讲内容——毕竟,一图胜千言!\par
C++的知识相当繁杂。掌握这些内容需要时间,也需要实践。本书为读者设计了大量实践章节,从功能分析,到设计规划,再到完整实现,可以带领读者完成一个基本的开发过程(但是本书编写匆忙,没有讲解调试相关的内容)。与这些实践章节相关的代码均可在本书附件``\textbf{随书代码}''中获取。\par
本书主要关注C++的语法,可以为读者建立完善而系统的C++编程世界观;但是本书不适合作为百科全书式的资料来使用——你应该去找更权威、更可靠、更详细的资料,而不是把生命浪费在一本教程中。本书也不太重视算法,只在为了讲解和实践相关语法时才介绍少量有关内容;如果读者想要学习这方面的进阶知识,也应当另行寻找其它教材。\par
本书是编程基础读物,适合高中及以上知识水平的读者阅读;但它不是编程启蒙读物,请不要用本书折磨小学生。\par
\begin{flushright}
\textit{cppHusky}\par
\end{flushright}