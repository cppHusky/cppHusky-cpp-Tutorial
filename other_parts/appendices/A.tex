\chapter{C++运算符基本属性}\label{ch:appendix_A}
表A.1按C++17标准,列出了所有运算符的优先级、结合性和重载要求。\footnote{值得注意的是,虽然双引号\texttt{""}不被视为运算符,但它可以用在重载自定义字面量的语法中。本书不予以介绍,对此感兴趣的读者可以阅读 \href{https://zh.cppreference.com/w/cpp/language/user_literal}{用户定义字面量-cppreference} 。}\par%tbs
\begin{longtable}{|c|c|c|c|c|}
\caption{截至C++17的所有运算符}\label{tab:A-1}\\
\hline
\textbf{优先级} & \textbf{运算符} & \textbf{运算符含义} & \textbf{结合性} & \textbf{重载要求}\\
\hline\hline
1 & \lstinline@scope::a@ & 作用域解析 & \multirow{9}{*}{从左到右} & 不可重载\\
\cline{1-3}\cline{5-5}
\multirow{8}{*}{2} & \lstinline@var++ var--@ & 后缀自增/自减 &  & 特殊参数要求\footnote{重载后缀自增/自减运算符需带 \lstinline@int@ 型参数。参数可以不具名,事实上也无需使用。这样的语法是为了与前缀形式相区别。}\\
\cline{2-3}\cline{5-5}
& \lstinline@type() type{}@ & 函数风格类型转换\footnote{函数风格类型转换、C风格类型转换和 \lstinline@static_cast@ 在静态类型转换时是等价的。换言之,只需定义一个转换函数,就可以使用这三种类型转换语法。} &  & 可重载\\
\cline{2-3}\cline{5-5}
& \lstinline@fun()@ & 函数调用 &  & 非静态成员函数\\
\cline{2-3}\cline{5-5}
& \multirow{2}{*}{\lstinline@arr[index]@} & \multirow{2}{*}{下标运算\footnote{对于内置类型数组 \lstinline@arr@ 的下标运算, \lstinline@arr[index]@(和 \lstinline@index[arr]@)都会解析为 \lstinline@*(arr+index)@(和\lstinline@*(index+arr)@),而自定义类型的下标重载版本不会如此。}} &  & 非静态成员函数\\
& & & & 仅接收单个参数\footnote{C++23标准起,重载下标运算符可以接收多个参数。在此之前,如有接收多个参数需求,一般建议重载函数调用运算符代替之。}\\
\cline{2-3}\cline{5-5}
& \lstinline@obj.member@ & 成员访问 &  & 不可重载\\
\cline{2-3}\cline{5-5}
& \multirow{2}{*}{\lstinline@p_obj->member@} & \multirow{2}{*}{指针的成员访问} &  & 返回类型受限\footnote{返回类型必须是裸指针或对象。返回对象时,必须返回同样重载了运算符 \lstinline@->@ 的对象。}\\
&  &  &  & 非静态成员函数\\
\hline
\multirow{9}{*}{3} & \lstinline@++var --var@ & 前缀自增/自减 & \multirow{9}{*}{从右到左} & 可重载\\
\cline{2-3}\cline{5-5}
& \lstinline@+var -var@ & 一元加/减(正/负号) &  & 可重载\\
\cline{2-3}\cline{5-5}
& \lstinline@!var ~var@ & 逻辑非/按位取反 &  & 可重载\\
\cline{2-3}\cline{5-5}
& \lstinline@(type)var@ & C风格类型转换 &  & 可重载\\
\cline{2-3}\cline{5-5}
& \lstinline@*pointer@ & 取内容(解引用) &  & 可重载\\
\cline{2-3}\cline{5-5}
& \lstinline@&var@ & 取地址 &  & 可重载\\
\cline{2-3}\cline{5-5}
& \lstinline@new@ \lstinline@new[]@ & 动态内存分配 &  & 可重载\\
\cline{2-3}\cline{5-5}
& \lstinline@delete@ \lstinline@delete[]@ & 动态内存回收 &  & 可重载\\
\cline{2-3}\cline{5-5}
& \lstinline@sizeof a@\footnote{不使用括号的 \lstinline@sizeof a@ 只适用于数据或对象;如果要对类型求内存空间大小,必须用 \lstinline@sizeof(type)@ 形式。} & 求内存空间大小 &  & 不可重载\\
\hline
\multirow{2}{*}{4} & \lstinline@obj.*p@ & 成员指针访问 & \multirow{5}{*}{从左到右} & 不可重载\\
\cline{2-3}\cline{5-5}
& \lstinline@p_obj->*p@ & 指针的成员指针访问 &  & 可重载\\
\cline{1-3}\cline{5-5}
5 & \lstinline@lhs*rhs lhs/rhs lhs%rhs@ & 乘法/除法/模运算 &  & 可重载\footnote{注意,重载运算符对参数有特殊要求:必须接收至少一个自定义类型的参数,可以是类或枚举类。比如说, \lstinline@double operator\%\(double,double)@ 就是不允许的。}\\
\cline{1-3}\cline{5-5}
6 & \lstinline@lhs+rhs lhs-rhs@ & 加法/减法 &  & 可重载\\
\cline{1-3}\cline{5-5}
7 & \lstinline@lhs<<rhs lhs>>rhs@ & 左移位/右移位 &  & 可重载\\
\hline
\textbf{优先级} & \textbf{运算符} & \textbf{运算符含义} & \textbf{结合性} & \textbf{重载要求}\\
\hline\hline
\multirow{2}{*}{8} & \lstinline@lhs<rhs lhs<=rhs@ & 小于/小于或等于& \multirow{8}{*}{从左到右} & 可重载\\
\cline{2-3}\cline{5-5}
& \lstinline@lhs>rhs lhs>=rhs@ & 大于/大于或等于 &  & 可重载\\
\cline{1-3}\cline{5-5}
9 & \lstinline@lhs==rhs lhs!=rhs@ & 相等运算符/不等运算符 &  & 可重载\\
\cline{1-3}\cline{5-5}
10 & \lstinline@lhs&rhs@ & 按位与 &  & 可重载\\
\cline{1-3}\cline{5-5}
11 & \lstinline@lhs^rhs@ & 按位异或(互斥或) &  & 可重载\\
\cline{1-3}\cline{5-5}
12 & \lstinline@lhs|rhs@ & 按位或(可兼或) &  & 可重载\\
\cline{1-3}\cline{5-5}
13 & \lstinline@lhs&&rhs@ & 逻辑与 &  & 可重载\footnote{逻辑与/逻辑或函数有短路求值特点,如对于 \lstinline@false&&f()@,在左操作数为 \lstinline@false@ 时已经能确定整个表达式为 \lstinline@false@,故不需要再求 \lstinline@f()@,于是右操作数及其副作用都不会发生。但这两个运算符经过重载后并不保留短路求值特性,左右操作数都会被计算。}\\
\cline{1-3}\cline{5-5}
14 & \lstinline@lhs||rhs@ & 逻辑或(可兼或) &  & 可重载\\
\hline
\multirow{8}{*}{15} & \lstinline@condition?exp1:exp2@ & 条件运算符 & \multirow{8}{*}{从右到左} & 不可重载\\
\cline{2-3}\cline{5-5}
& \lstinline@throw exception@ & 抛出异常 &  & 不可重载\\
\cline{2-3}\cline{5-5}
& \lstinline@lhs=rhs@ & 直接赋值 &  & 只能是成员函数\\
\cline{2-3}\cline{5-5}
& \lstinline@lhs+=rhs lhs-=rhs@ & \multirow{5}{*}{复合赋值} &  & \multirow{5}{*}{只能是成员函数}\\
& \lstinline@lhs*=rhs lhs-=rhs@ &  &  &\\
& \lstinline@lhs%=rhs lhs|=rhs@ &  &  &\\
& \lstinline@lhs<<=rhs lhs>>=rhs@ &  &  &\\
& \lstinline@lhs&=rhs lhs^=rhs@ &  &  &\\
\hline
16 & \lstinline@exp1,exp2@ & 逗号 & 从左到右 & 可重载\\
\hline
\end{longtable}
另有这些运算符:\lstinline@static_cast@, \lstinline@dynamic_cast@, \lstinline@const_cast@, \lstinline@reinterpret_cast@, \lstinline@typeid@, \lstinline@noexcept@。它们在使用时必须对操作数加括号(相比之下, \lstinline@sizeof@ 是可以不对操作数加括号的),因而不受优先级、结合性问题困扰,不列入表A.1。以上这些运算符均不可重载,但你可以通过自定义转换函数的方式来使 \lstinline@static_cast@ 变得可用。\par
