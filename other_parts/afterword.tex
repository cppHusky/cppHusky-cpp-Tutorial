\chapter*{跋}
\addcontentsline{toc}{chapter}{跋}
很高兴认识你,本书的读者。\par
也许你有很多问题想要问我。但是,限于书籍这种介质的表达方式,我看不到你的问题。不过,我也不能让你空手而归,所以我准备了这段跋。如果这里碰巧有你需要的答案,那当然妙不可言;倘若没有——你也可以等明年嘛。\par
关于为什么一开始要做这个``年礼''——可能是高中生活太过枯燥想要找寻一点乐趣;可能是那时候考试成绩什么的太差想要挽救一点可怜的自尊心;也可能只是单纯的创作欲望太高想要非做些什么不可——然而那时我没有记日记的习惯,所以当初的想法已经无从考证。如果真是最后一种原因的话,那甚至可以要追溯到我的初中时光了。彼时我就总是在创作所谓的``推理小说'',最关键的是写完一篇还一定要做一个看似正经的``字数统计'',好像每多了一个字就多了一番莫名其妙的成就感——或许那时我尚处于``以字数论高下''的思想阶段吧。\par
总之,不管什么原因吧,年礼就这么做下来了,一年一份,成了一项传统,到现在已是五个年头。但若真论``达到被认可乃至受赞誉''的程度,那便只有去年那一部《积分精选》而已!去年我写后记的时候貌似说``外界的评价不是我做年礼的动机'',诚不欺人。不过我也仅仅是个普通人,怎可能完全不在乎别人怎样看待和评价我辛苦几个月乃至一整年的成果呢?\par
去年年礼做完之后,短时间内并无多少反响。有反响的时候,还是数月之后,我会在各个地方机缘巧合地收到零星的感谢消息——还有人说他的期末考试是对着我的书复习的,这实在让我有点受宠苦惊。\par
我想,这本书大概也会遵循同样的历程,像我前面那些年的作品一样,在发布之后很快销声匿迹,只在网络中的狭缝中留住一线生机。或许哪一天,它被某个网络考古爱好者发现了,对方通过网络发来一条贺信,在两个素不相识的人之间建立了一丝转瞬即逝的联系,这就够我一整天的好心情了。\par
本书追求尽善尽美。我可以在一段叙述,两行代码,乃至一张插图的配色上较劲。但是毕竟时间有限,我的能力也有欠缺。今年我所能做的,也就是这些。至于本书中仍然遗留的错误——一经发表,它们就是本书的一部分了,我不会再改动;其中的是与非,对与错,就交由别人来评价了。\par
如果我还能活到``那个时候''的话,这本书还会有第三版。到那时,如果还有机会,我们还会在跋中重逢。谢谢你能读到这里,我们后会有期。\par
\begin{flushright}\textit{cppHusky}\end{flushright}
