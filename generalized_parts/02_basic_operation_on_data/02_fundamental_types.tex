\section{基本数据类型}
\subsection*{整型}
在第一章我们已经谈过,整型就是只能表示整数的数据类型。\par
常用的整型变量一般有 \lstinline@short@, \lstinline@int@, \lstinline@long@, \lstinline@long long@ 及其相应的 \lstinline@unsigned@ 版本\footnote{\lstinline@long long@ 及 \lstinline@unsigned long long@ 在C++11标准以后可用。本书默认使用C++17标准,故不再反复提及此问题。}。某一类型在不同电脑上占用的内存空间大小可能是不同的,而内存空间的多少又和数据表示范围紧密相关,所以我们不妨把类型和数据范围分开讨论。\par
2字节的数据类型可以表示$(2^8)^2=65'536$种状态,也就是可以存储$65'536$种值。有符号类型编码了$0$\~{}$32'767$和$-32'768$\~{}$-1$的数字(换言之,从$-32'768$到$32'767$);而无符号类型就不需要编码负数了,节省出来的容量可以编码更多正数,所以无符号类型编码了$0$\~{}$65'535$的数字。\par
4字节的数据类型可以表示$(2^8)^4=4'294'967'296$种状态,可以存储$4'294'967'296$种值。有符号类型编码了$-2'147'483'648$\~{}$2'147'483'647$的数字;无符呈类型编码了$0$\~{}$4'294'967'295$的数字。\par
8字节的类型可以表示$(2^8)^8\approx1.84\times10^{19}$种状态。有符号类型编码了$-2^{63}$\~{}$2^{63}-1$的数字;无符号类型编码了$0$\~{}$2^{64}-1$的数字。\par
于是乎,我们只需要通过 \lstinline@sizeof@ 算出它的内存空间大小,再根据它是有符号类型还是无符号类型,就可以判断出它的数据范围是什么了。\par
C++还有另一种更简便的方法,那就是直接使用 \lstinline@limits@ 头文件中定义的 \lstinline@numeric_limits@ 来输出某一类型的数据上下限。\lstinline@numeric_limits@ 给出的结果正是本电脑的类型数据范围。\par
以下是实现这个功能的完整示例代码:
\begin{lstlisting}[caption=\texttt{Integer\_Limits.cpp}]
// 这是一个简单的C++程序,用于输出一些整型数值的上/下限值
#include <iostream> //标准输入输出头文件
#include <limits> //标准算术类型属性头文件,numeric_limits定义于此
using namespace std; //使用std命名空间
int main(){ //主函数,程序执行始于此
    cout << numeric_limits<short>::lowest(); //输出short类型的下限值
    cout << endl; //输出换行,下同
    cout << numeric_limits<short>::max(); //输出short类型的上限值
    cout << endl;
    cout << numeric_limits<unsigned>::lowest(); //输出unsigned类型的下限值
    cout << endl;
    cout << numeric_limits<unsigned>::max(); //输出unsigned类型的上限值
    return 0;
}
\end{lstlisting}
在Coliru上,该程序代码的运行结果如下:\\\noindent\rule{\textwidth}{.2pt}\texttt{
-32768\\
32767\\
0\\
4294967295
}\\\noindent\rule{\textwidth}{.2pt}\par
这里需要介绍几点:\par
首先,\lstinline@numeric_limits@ 是一个类模版\footnote{关于类模版的有关知识,我们将在第十一章中讲解。},它要接收一个``类''作为模版参数。比如说,在这个例子中,我们分别使用了 \lstinline@short@ 和 \lstinline@unsigned@ 作为模版参数,然后用 \lstinline@lowest()@ 和 \lstinline@max()@ 成员函数\footnote{关于函数的基本知识,我们将在第四章中讲解;而关于类的成员函数,我们将在第八章讲解。}来求出我们想要知道的值。如果你愿意,也可以把这段代码中的类型改成其它任何一个基本数据类型,然后观察输出,比如 \lstinline@double@ 或 \lstinline@char@。\par
其次是``输出换行''的问题。当我们的代码中有多个输出语句的时候,程序会在运行时按顺序执行这些输出语句。但是程序不会自己添加分隔符\footnote{分隔符(Delimiter),指的是将一系列内容按同样的格式分开的符号,它有点像标点符号,但又不尽然。分隔符既方便人类阅读,又让计算机能够分辨不同的数据。比如说,我们用空格将一系列词语分开,这时空格就是它们的分隔符。},它会这样输出(以刚才程序为例):\\\noindent\rule{\textwidth}{.2pt}\texttt{
-327683276704294967295
}\\\noindent\rule{\textwidth}{.2pt}\\
这不便于我们阅读和理解,所以我们应当人为地把两个输出用分隔符隔开。而 \lstinline@cout@ 就是一个换行符,输出它的效果相当于另起一行。我们当然可以使用其它的换行符,比如我们把输出换行的语句改成输出逗号,
\begin{lstlisting}
    cout << ','; //这相当于输出一个字符字面量',',于是程序会输出一个逗号作分隔符
\end{lstlisting}
那么输出应当是这样的:\\\noindent\rule{\textwidth}{.2pt}\texttt{
-32768,32767,0,4294967295
}\\\noindent\rule{\textwidth}{.2pt}\\
我们现在还没到讲解输入输出控制的时候,但这方面的知识我们又不得不了解,所以在这里先开个头,以后我们还会频繁遇到的。\par
最后还有一个问题:假如我的运算超过了整型的表示范围,会发生什么?比如对于有符号型2字节的变量来说,\lstinline@32767+1@ 会得到什么?答案是 \lstinline@-32768@!这是一种整数溢出(Integer Overflow)现象,详细的原理我们之后再谈;总之我们需要记住,在写代码的过程中需要对可能出现的数据进行预判,尽量避免出现这种运算结果``装不下''导致的情况。\par
\subsection*{浮点型}
浮点型数据的编码方式比整型数据要复杂些,我们在精讲篇中再作详细探讨。\par
我们一般使用 \lstinline@float@, \lstinline@double@ 和 \lstinline@long double@ 来存储浮点型变量。一般来说,\lstinline@float@ 型占用4字节内存空间,\lstinline@double@ 型占用8字节内存空间,而 \lstinline@long double@ 型占用16字节内存空间。我们同样可以用 \lstinline@numeric_limits@ 来求出它的上/下限。这里我就不再演示代码了,想要尝试的读者可以自行修改代码2.1并运行之。\par
浮点数据和整型数据有一个重要的差异在于,很多时候浮点数表示的都是``近似值''。例如,在17位小数的显示精度下,\lstinline@double@ 型的 \lstinline@0.1@ 和 \lstinline@0.2@ 相加的结果是 \lstinline@0.30000000000000004@;而 \lstinline@0.3@ 则显示为 \lstinline@0.29999999999999999@。\par
那这个结果对还是不对呢?\textbf{在精度要求不高的情况下它是对的},因为当你四舍五入到16位小数的时候它就相等了。而 \lstinline@cout@ 默认显示至多6位有效数字,所以这样的误差是微乎其微的。\footnote{这里还涉及浮点数相等性的判断问题。正因浮点数多为近似值,所以直接使用相等性运算符极易出现失误。实际操作中我们多用两数相减,通过差值的绝对值是否足够小(比如,限定小于$10^{-5}$)来判断它们是否相等。}\par
还有另一个问题。假如我们需要很高精度的计算,比如要求圆周率取到小数点后20位,那么有效数字位数仅有15\~{}17位的 \lstinline@double@ 型是不能满足我们的要求的。这时有经验的程序员当然会首选 \lstinline@long double@了。
\begin{lstlisting}
    const long double Pi = {3.14159265358979323846};
    //定义一个长双精度浮点型常量Pi,精确到小数点后20位
\end{lstlisting}
看起来没有丝毫问题,对吗?但是问题恰恰就出在字面量上!\par
前面我们就提过,编译器会将所有浮点字面量统一视为 \lstinline@double@ 类型的字面量。因此,字面量 \lstinline@3.14159265358979323846@ 本身就是一个 \lstinline@double@ 型数据,它早在编译时就因为不能容纳如此高的精度而发生了精度上的损失。那么用损失后的只有15\~{}17位精度的数字去给 \lstinline@long double@ 数据初始化也是枉然!\par
但我们也有应对方法:指定字面量的类型。在C++中,我们可以用后缀的方式来指定整型或浮点型字面量的类型。当我们在一个浮点数后加 \lstinline@f@ 后缀时,就指明了它是 \lstinline@float@ 单精度浮点数,比如 \lstinline@1.f@, \lstinline@.57f@, \lstinline@2.718f@;而当我们在一个浮点数后加 \lstinline@l@ 后缀时,就指明了它是 \lstinline@long double@ 长双精度浮点数,比如 \lstinline@1.l@, \lstinline@.57l@,以及 \lstinline@3.14159265358979323846l@,这样就不致有20位以内的精度损失了\footnote{诚如上文所言,浮点数的微小精度损失普遍存在,但只要在我们的承受范围内就可以。}。\par
除了使用后缀以外,我们还有一些其它的字面量表示语法。如果你刚才已经试过输出浮点数的上/下限,那么你可能会看到这种格式的输出结果:\lstinline@1.79769e+308@。这是什么意思呢?\par
实际上,浮点数表示的范围跨度很大,比如 \lstinline@float@ 可以表示$-3.402'823'4\times10^{38}$\~{}$3.402'823'4\times10^{38}$范围的值。这么大的值不适合用一般方法表示,所以在计算机中我们有一种类似科学记数法的形式。\par
它的规则很简单。一个整数或浮点数都可以在后面加上 \lstinline@e@(或 \lstinline@E@,均可)和指数(可正可负)的形式。例如:\\
\lstinline@2.99792458e+8@(小写字母,带正号);\\
\lstinline@6.02214076e23@(正号可省略);\\
\lstinline@1.602176634E-19@(大写字母也可以);\\
\lstinline@.000662607015E-30@(不一定非要在1到10之间嘛);\\
\lstinline@25.81280745E+3@(形式很灵活,这样写也可以)\\
\lstinline@3141.59265358979323846264338327950288e-3l@(末尾的 \lstinline@l@ 说明它是长双精度浮点数,注意:其位置在指数之后)\par
当然,浮点的字面量表示还有更多花样,不过这里就不再介绍了。\par
整型数据的字面量同样有很多后缀可以用来指定类型,比如 \lstinline@ull@ 后缀规定了 \lstinline@unsigned long long@ 类型。但这并不是刚需,因为大多数情况下编译器都能自动为整型字面量匹配合适的类型。\par
\subsection*{字符型}
在第一章,我们就介绍过 \lstinline@char@ 字符类型和相应的ASCII编码。我们也提及,字符类型其实是一种特殊的整型。它与整型的编码方式有相同之处,而与浮点型的编码方式则是天差地别。\par
字符型在输出时,主要是为了表示,而非为了计算。所以在用 \lstinline@cout<<@ 输出字符型数据时,输出的并不是它的ASCII码值,而是这个字符本身,比如
\begin{lstlisting}
    cout << '0'; //输出字符型字面量'0'
\end{lstlisting}
虽然\lstinline@'0'@的ASCII值是\lstinline@48@,但程序的输出实际上是 \lstinline@0@。\footnote{在代码中,\lstinline@'0'@ 和 \lstinline@0@ 类型不同。但在控制行界面上,无论是人类的肉眼还是计算机程序,都无法分辨它究竟是字符还是数字——甚至说,它们就是同一个东西!我们在第三章会对此有所涉及。}\par
如果我们要输出它的ASCII码值呢?我们可以使用后文马上就要讲到的类型转换方法,将 \lstinline@char@ 类型数据转换成 \lstinline@int@ 类型来输出,或者是通过减去ASCII码为 \lstinline@0@ 的字符 \lstinline@'\0'@ 来实现——因为两个 \lstinline@char@ 类型之差的结果为 \lstinline@int@ 类型。
\begin{lstlisting}
    cout << '0' - '\0'; //输出相当于48-0,也就是48
\end{lstlisting}\par
你可能有注意到,这个ASCII码为 \lstinline@0@ 的字符长得好像有点奇怪。在C++中还有许多类似的字符,像 \lstinline@'\t'@, \lstinline@'\n'@ 之类,它们统称为\textbf{转义字符(Escape sequence)}。转义字符的真实含义不是字面上的含义,它表达了特定的控制字符。比如 \lstinline@'\n'@ 指的是换行符,如果我们输出一个 \lstinline@'\n'@,就相当于为输出换了一行。那么我们也可以用它来替代 \lstinline@endl@\footnote{实际上这两种换行方法有更细微的区别,即何时清空缓冲区的问题。不过它不会影响我们的输出结果,我们不必细究。}。
\begin{lstlisting}
    cout << numeric_limits<char>::lowest(); //输出char类型的下限值
    cout << '\n'; //输出换行符,效果相当于cout<<endl;
    cout << numeric_limits<char>::max(); //输出char类型的上限值
\end{lstlisting}\par
不同电脑环境下 \lstinline@char@ 类型的范围有两种可能:一种是-128\~{}127,另一种是0\~{}255。无论哪种可能,都涵盖了ASCII码的基本范围0\~{}127。\par
那么剩下的范围呢,它们有什么作用?这不仅取决于你的系统,还取决于你使用的语言。如果你使用的是Windows中文版,你可以测试下这段代码:
\begin{lstlisting}
    //适用于GBK编码
    cout << char (196) << char (227) << char (186) << char (195);
\end{lstlisting}
它的输出应该是``\lstinline@你好@''。你还可以在Coliru上测试下这段代码,这个结果不受你使用的系统和语言的影响:
\begin{lstlisting}
    //适用于UTF-8编码
    cout << char (228) << char (189) << char (160)
         << char (229) << char (165) << char (189);
\end{lstlisting}
它的输出也应该是``\lstinline@你好@''。\par
实际上这里涉及到更深奥的可变宽度编码\footnote{可变宽度编码(Variable-width encoding)是一种字符编码方案,它对不同的字符用不同长度的编码来表示。相对地,ASCII是一种定宽编码,它的每个字符都用统一的单个字节来表示。}技术。在Windows简体中文的环境中常用GBK编码,这是一种变宽编码,其中ASCII字符占一个字节,汉字占两个字节(比如第一段代码);在Windows繁体中文的环境中早期常用Big5编码,这是一种定宽编码,所有字符均占两个字节;而许多系统现在都支持Unicode,其中以UTF-8实现方式为多,所有字符占一至四个字节不等,而CJK文字\footnote{CJK文字是中文、日文和韩文的统称,包括全数汉字及变体、假名和谚文等。}常用者为三字节(比如第二段代码),罕见字为四字节。在泛讲篇中我们不会深入讲这些知识,也不会普遍使用,我们留到精讲篇再来谈。\par
\subsubsection*{字符串简介}
字符串并非基本类型,但是它很重要,我们就在这里一并介绍。\par
还记得我们的\hyperref[lst:HelloWorld]{代码1.1}吗?在这个代码中,我们输出的 \lstinline@"Hello World!"@ 就是一个字符串字面量。顾名思义,字符串字面量就是一串字符构成的字面量。每个字符串都有一个我们看不到的结束符作为结尾,它就是 \lstinline@'\0'@。\par
我们可以用 \lstinline@sizeof@ 来取它的内存大小。这个字符串一共有10个大小写字母、1个标点和1个空格\footnote{在本书中,空格字符以\ {\color{red}\textvisiblespace}\ 表示,以便于阅读。(当然,这可能会对读者使用Ctrl+C/V造成一定的麻烦)},外加结束符 \lstinline@'\0'@,全部是ASCII字符,所以它的内存占用应当是13字节。
那么我们用 \lstinline@sizeof@ 算算,验证一下我们的猜想。
\begin{lstlisting}
    cout << sizeof ("Hello World!"); //也可以不加括号
\end{lstlisting}
输出果然是 \lstinline@13@。\par
需要留心,字符是用\textbf{单引号}套起来的单个字符(转义字符是个例外,它允许你使用反斜线加一个字符)\footnote{实际上C++中允许使用单引号套起来的多字符字面量(Multicharacter literal),但我认为没有必要在本书讲它。感兴趣的读者可以阅读\href{https://stackoverflow.com/questions/7459939/what-do-single-quotes-do-in-c-when-used-on-multiple-characters}{What do single quotes do in C++ when used on multiple characters?-Stack Overflow}。};而字符串是用\textbf{双引号}套起来的任意多个字符。转义字符同样可以用在字符串中,比如
\begin{lstlisting}
    cout << "ABCD\nabcd";
\end{lstlisting}
这段代码的输出应为\\\noindent\rule{\textwidth}{0.2pt}\texttt{
ABCD
abcd
}\\\noindent\rule{\textwidth}{0.2pt}\\
这里的 \lstinline@\n@ 就起到了换行符的作用。如果要用 \lstinline@sizeof@ 来计算,它的内存占用应当是10字节。\par
\subsection*{布尔型}
布尔型是一种特殊的整型,它一般用来表示一种逻辑判断的信息。举例来说,``\lstinline@2>3@''是一个逻辑判断,它的值为``假''(False);而``\lstinline@"Hello World!"@ 的内存占用为13字节''也是一个逻辑判断,它的值为``真''(True)。\par
布尔型数据用关键字 \lstinline@bool@ 来定义,它只有两个可能的取值:\lstinline@true@ 和 \lstinline@false@。在编码时,它们分别被编为 \lstinline@1@ 和 \lstinline@0@。\par
按理说,我们只需要一个比特就能表示一个布尔变量。然而,内存当中的比特并不是独立的寻址单元,所以一个布尔型变量必须要用一整个字节来存储。\footnote{标准模版库中的 \lstinline@vector<bool>@ 采用了优化方法,使得每个字节可以存储8个布尔数据,这就提升了存储空间的利用率。}读者可以用 \lstinline@sizeof@ 来验证它。\par
布尔型数据在输出时默认以整型数据的格式输出,于是 \lstinline@false@ 会被输出为 \lstinline@0@,而 \lstinline@true@ 会被输出为 \lstinline@1@。
\begin{lstlisting}
    bool judgement {2>3}; //定义bool型变量judgement,其值为false
    cout << judgement; //因为judgement的值为false,所以输出0
\end{lstlisting}
我们可以通过一些语法设置来强制 \lstinline@cout@ 输出 \lstinline@true@ 或 \lstinline@false@。
\begin{lstlisting}
    cout.setf(ios_base::boolalpha); //设置格式化标志
    cout<<true; //这样一来,它就会输出true而非1
\end{lstlisting}
我们现在先不讨论这些细节,在第十三章中会谈及。\par
