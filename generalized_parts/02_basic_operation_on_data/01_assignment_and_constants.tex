\section{赋值语句与常量}
在此之前我们定义了许多数据,但是只进行了初始化,并未在定义之后修改过它们的值。实际上我们之前定义过的那些数据(它们也被称为\textbf{变量(Variable)})都可以通过赋值、输入、运算等方式来修改其值。这里先来讲一下赋值语句,然后是常量;输入和运算将在之后讲解。\par
\subsection*{基本赋值语句}
最基本的赋值语句的结构是这样的:
\begin{lstlisting}
    <变量名> = <值>; //用字面量为其赋值
    <变量名> = <表达式>; //用其它数据或表达式为其赋值
\end{lstlisting}
如果你愿意,也可以将初始化看作一个赋值过程\footnote{对于内置类型,我们看不出细节来;但对于自定义类型来说,\textbf{初始化和赋值是不同的}。初始化时会调用构造函数,而赋值时会调用赋值运算符。我们在第八章中会细讲。};但我绝不会这样讲。\par
在基本赋值语句中出现的 \lstinline@=@ 叫做\textbf{赋值运算符(Assignment operator)},它的含义是:将右边的数的值赋给左边,从而改变左边的值。\par
赋值运算符的用途很广泛,C++中的许多数值改变都要通过赋值来实现。以下是一些常见的例子:
\begin{lstlisting}
    int n = {3}; //定义n并初始化为3
    n = n * n; //这是在做什么?
    cout << n; //输出会是什么样的?
\end{lstlisting}
初学者可能会对$n=n\times n$这样的表达感到困感。但是不要忘记,这里是编程,而不是数学!\par
\lstinline@n=n*n@ 这个语句可以由赋值运算符分为两部分,左边是 \lstinline@n@,右边是 \lstinline@n*n@。按照赋值运算符的含义,它的功能是将右边的数的值(\lstinline@n*n@)赋给左边,从而改变左边数(\lstinline@n@)的值。也就是说,它的含义是在把$n^2$的值赋给$n$,于是输出就变成了 \lstinline@9@。\par
赋值运算符和数学上的等号\footnote{数学上的等号在C++中可以用相等性运算符 \lstinline@==@ 来表示。我们会在后面讲解。}不同的地方还在于,它是不满足交换律的。比如说,\lstinline@a=b@ 的含义是把 \lstinline@b@ 的值赋给 \lstinline@a@,而 \lstinline@b=a@ 的含义是把 \lstinline@a@ 的值赋给 \lstinline@b@。再比如,如果我们把刚才的代码改成这样,它是不能通过编译的:
\begin{lstlisting}
    int n = {3}; //定义n并初始化为3
    n * n = n;
//error: lvalue required as left operand of assignment
    cout << n;
\end{lstlisting}
我已将编译器的报错信息以注释的方式添加到代码中。它的意思是:赋值运算符的左操作数需要是一个左值——我们现在还不需要管什么是``左值'',只需知道,给两个数的乘积赋值是没有意义的,编译器也不允许我们这样做。\par
\subsection*{常量}
任何一个变量(广义地讲,对象)在被定义之后都可以修改,这固然很方便,但也可能会遭遇一些问题。\par
圆周率的近似值是 \lstinline@3.14@,我们希望定义一个变量 \lstinline@Pi@,它不会被我们自己,或是别的人随便改动;围棋棋盘上有 \lstinline@19*19@ 个可放子的交叉点,这是恒定不变的,我们也希望不会有人乱改。\par
但是传统意义上的变量无法禁止修改。C语言使用宏\footnote{宏(Macro),是一种批量转换特定内容的方法。它的功能很像我们常用的``查找替换''。}来定义一个标识符,并通过预处理将其转换为字面量(字面量无法被修改)来实现禁止修改的效果。而在C++中,我们有更好的方法:使用 \lstinline@const@ 限定符,将变量限定为\textbf{常量(Constant)}。\footnote{广义上讲,``常量''是``变量''的一部分。根据 \href{https://en.wikipedia.org/wiki/Variable_(computer_science)}{Variable-Wikipedia} 的阐释,凡在内存中占据存储空间的具名量都是变量。这点似与数学上的``变量''概念不同。本书于处理常量变量之关系处也很为难,故约定:\textbf{在需要分清常量和变量的场合下,将它们视为不重合的概念;而在一般场合下,统称其为变量。}}\par
定义一个常量的语法是
\begin{lstlisting}
    const <类型> <名字> = <初始化>; //const可以在类型关键字前
    <类型> const <名字> = <初始化>; //const在类型关键字后,效果相同
\end{lstlisting}
这两种方式均正确,读者可以选择自己喜欢的方式来定义。\par
我们可以定义一个 \lstinline@double@ 型的常量 \lstinline@Pi@ 来表示圆周率的近似值,再定义一个 \lstinline@int@ 型的常量 \lstinline@Grid@ 来表示围棋棋盘的边长。那么我们可以这样写:
\begin{lstlisting}
    const double Pi = {3.14}; //定义一个double常量Pi
    const int Grid {19}; //定义一个int常量Grid
\end{lstlisting}
如果你试图在之后的代码中修改常量,那么编译器将报告这样的错误:
\begin{lstlisting}
    Pi = 3.1; //试图修改常量Pi的值
//error: assignment of read-only variable 'Pi'
\end{lstlisting}
这说明 \lstinline@Pi@ 是一个只读的数据,我们不能对它赋值来进行修改。\par
在C++中,常量是一种\textbf{一经定义就无法修改的量}。这意味着,我们只能——也必须——在定义它的同时就给它初始化。\footnote{某种意义上,这也体现出初始化与赋值的区别。}变量则不同,我们可以不进行初始化,或者是在定义之后进行赋值。
\begin{lstlisting}
    int i; //定义变量但不初始化,此时i的值未定
    i = 2; //赋值,这样i的值就确定下来了
    const int i_c; //定义常量但不初始化
//error: uninitialized 'const i_c' [-fpermissive]
\end{lstlisting}
这个报错信息的含义是:\lstinline@i_c@ 作为一个常量,没有被初始化。所以这是不允许的。\par
\subsection*{常量与字面量的区别}
说到这里,你也许会感到困感:如果我们用 \lstinline@const@ 常量就是为了防止数据被修改的话,那为什么不全部使用字面量呢?在需要 \lstinline@Pi@ 的场合下,全都使用 \lstinline@3.14@ 不就行了?\par
当然可以了。但是相比于直接使用字面量来说,使用常量有一些显著的优点:\par
首先,常量可以用标识符来阐述其含义。比如用 \lstinline@Grid@,这就比单纯放一个字面量 \lstinline@19@ 要容易理解得多,也便于后期我们回顾代码。\par
而且,常量容易统一修改。这里的``修改''不是上文中提到的那种赋值式的修改,而是直接修改定义语句中用来初始化的值。\par
我们在一开始设计围棋程序的时候可能没有想到,围棋棋盘的大小还可以有$13\times13$和$9\times9$两种简化版。如果咄咄逼人的甲方要求你改成 \lstinline@13@ 或者 \lstinline@9@ 的边长,而你又在你的上千行代码中写下了数百个 \lstinline@19@,你可能会崩溃的。\par
而如果我们都用常量呢?我们可以不用写数百个 \lstinline@19@,而是把它们都用 \lstinline@Grid@ 代替。这样一来,如果甲方要求你改变棋盘的大小,你就可以通过改变初始化值的方式来解决问题了。
\begin{lstlisting}
    const int Grid {13}; //现在把19改成13
\end{lstlisting}
这样就很省事了。\par
常量还有一个优点,它可以使用变量来初始化。比如说,我现在换了个思路,不想再修改源代码了。我通过一个中间变量来接收输入,然后把这个输入的值作为棋盘大小,用来给常量初始化。这个过程对于初学者来说可能有点复杂,我把代码列在这里:
\begin{lstlisting}
    int g; //定义一个临时变量g。因为通过输入来赋值,我们不必初始化
    cin >> g; //通过输入来改变g。输入多少,g就变为多少
    const int Grid = {g}; //用g的值来为Grid初始化,g的值取决于输入
\end{lstlisting}
这是一个相当有创造性的做法!在这里,\lstinline@Grid@ 不是预先决定的常量,而是可以通过变量来控制的常量。它仍然符合\textbf{一经定义就无法修改}的限制,同时又有了一定的灵活性。这种做法在将来的函数参数传递中非常常用,后面我们还会继续这个话题。\par
\subsection*{枚举常量简介}
\textbf{枚举(Enumeration)}在C++中的使用不像常量那么频繁,但它也很好用。在有些场合中,定义整型常量比较麻烦,而定义枚举很简单,这时我们可以用枚举来简化我们的代码。在另一些场合中,我们可以用枚举来方便地为一些标识符制定常量编码,并限制可能的取值范围。\par
具名的枚举可以视作一个自定义类型,而常量只能依托于已有的类型(无论是内置类型还是自定义类型),比如 \lstinline@int@, \lstinline@double@ 之类。\par
定义一个枚举要用到 \lstinline@enum@ 关键字。定义枚举的最简单语法是:
\begin{lstlisting}
    enum { <标识符1>, <标识符2>, <标识符3>, ... }; //无枚举名
    enum <枚举名> { <标识符1>, <标识符2>, ... }; //有枚举名
\end{lstlisting}\par
无枚举名的语法一般不用于自定义类型,而是纯粹当作整型常量。我们可以指定不同的标识符的值。
\begin{lstlisting}
    enum { MiniGrid = 9, SmallGrid = 13, NormalGrid = 19 };
    //定义无枚举名的枚举,相当于整型常量
\end{lstlisting}
这样我们就可以像常量一样使用它们了。\par
有枚举名的语法当然也可以当作整型常量来看待,不过它支持的功能就更丰富了。
\begin{lstlisting}
    enum Color { Red, Blue, Green}; //定义有枚举名的枚举
    Color sky_color = {Blue}; //定义Color类数据sky_color,其值为Blue
    Color grass_color = {Green}; //定义Color类数据grass_color,其值为Green
\end{lstlisting}
这里的 \lstinline@sky_color@ 和 \lstinline@grass_color@ 就相当于 \lstinline@Color@ 类型的两个数据,或者也可以说是 \lstinline@Color@ 类的两个对象。\par
这里只对 \lstinline@enum@ 作简单介绍,而不要求读者掌握。我们会在第六章讲解关于枚举的更多细节。\par
