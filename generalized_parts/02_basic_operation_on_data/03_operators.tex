\section{运算符}
本节我们将会接触到一个非常有趣的知识:运算符。许多教材在讲运算符的时候讲解不够透彻,虽然不太影响基础层面的知识掌握,但对以后的学习,尤其是运算符的重载,是比较不利的。我在本节会用函数的视角来剖析运算符,希望能为读者带来清晰的认识和比较透彻的理解。\par
\subsection*{运算符的操作数与返回值}
来看一个简单的例子:
\begin{lstlisting}
    int num {13 * 5 + 9}; //定义num并初始化为13*5+9
\end{lstlisting}
我们可以把这句代码分成两步操作:第一步是把 \lstinline@13*5+9@ 求出来,第二步是初始化。第一步的求值过程还可以进一步拆分,先求乘法,再求加法。那么我的问题是:乘法操作把哪两个数乘起来了?加法操作又把哪两个数加起来了?初始化操作又把哪个数的值给了 \lstinline@num@?\par
让我们来梳理一下程序的操作过程,这样就会对这些问题有清晰的认知了:
\begin{enumerate}
    \item 首先,乘法操作把 \lstinline@13@ 和 \lstinline@5@ 乘起来,并求得 \lstinline@65@,这是毫无疑问的。
    \item 其次,加法操作把 \lstinline@13*5@,也就是 \lstinline@65@,和 \lstinline@9@ 加起来,并求得 \lstinline@74@。这时候加号左边的 \lstinline@13*5@ 被加号看作一个整体,它操作的不是 \lstinline@5@ 而是 \lstinline@13*5@。
    \item 最后,初始化操作把 \lstinline@13*5+9@,也就是 \lstinline@74@ 的值初始化给 \lstinline@num@,这个过程没有再求出什么。这时候初始化所用的 \lstinline@13*5+9@ 被初始化语句看作一个整体,它操作的不是单个数而是整个表达式。
\end{enumerate}\par
从这个过程中,我们可以看出这样一个特点:有些操作过程会直接用我们给定的字面量(也可以是变量)来计算;而有些操作过程需要用到前面某一步算出来的值,并把它当作一个整体来运算。有些操作过程还会产生一个结果,后面的操作可以利用这个结果。\par
现在我开始甩专有名词:我们把一个运算符用来计算的数据(包括表达式整体)叫作它的\textbf{操作数(Operand)},而一个运算符计算出来得到的结果(一个表达式整体)叫作它的\textbf{返回值(Return value)}。操作数和返回值都是对于某个运算符而言的,比如 \lstinline@13*5@ 既是乘法运算的返回值,又是加法运算的操作数。初始化的过程其实不涉及某个运算符\footnote{如果涉及自定义类型的初始化,它其实是一种构造函数。},所以我们这里不再探讨。\par
我们在关注操作数和返回值的时候必须留意它们的类型,因为类型不同可能会带来天差地别的结果。我们在前面已经提过 \lstinline@8/3@ 和 \lstinline@8./3@ 的区别了,这是因为:当除号的操作数都是 \lstinline@int@ 型时,返回值就是 \lstinline@int@ 型;而只要有一个操作数是 \lstinline@double@ 型,即便另一个操作数是 \lstinline@int@ 型,返回值也将是 \lstinline@double@ 型\footnote{实际上这个过程更复杂,它先将 \lstinline@int@ 型经过类型转换变为 \lstinline@double@ 型,再进行两个 \lstinline@double@ 数据的计算。}。\par
两个 \lstinline@char@ 类型数据相减,得到的结果是 \lstinline@int@ 类型的,因此我们上一节中讲到的 \lstinline@'0'-'\0'@ 的语法能够求出 \lstinline@'0'@ 的ASCII码值。\par
赋值语句也有返回值,它返回的是左操作数的一个引用\footnote{我们会在第五章介绍引用。即便读者尚不知道何为引用,也不会对读者理解此节造成困难。}。在C++中我们偶尔会遇到连续赋值的语法,如下例:
\begin{lstlisting}
    int a, b, c; //连续定义a,b,c,但不初始化,此时它们的值均不确定
    a = b = c = 0; //连续赋值
\end{lstlisting}
鉴于我们还没有讲运算符结合性,我先用一个容易理解的带括号版本来介绍:
\begin{lstlisting}
    a = (b = (c = 0)); //连续赋值,等价于a=b=c=0;
\end{lstlisting}
圆括号改变了计算的顺序\footnote{再次提醒,方括号\lstinline@[]@和花括号\lstinline@\{\}@它们有各自的作用。我们不能用它们来达到改变运算顺序的目的,只能用圆括号\lstinline@()@。},它将把括号内的部分当作一个整体来对待。于是这段代码执行的操作就是
\begin{enumerate}
    \item \lstinline@c=0@ 中的赋值运算符将左操作数 \lstinline@c@ 的值变为右操作数 \lstinline@0@。最后返回 \lstinline@c@ 的引用。
    \item \lstinline@b=(c=0)@ 中的赋值运算符将左操作数 \lstinline@b@ 的值变为 \lstinline@(c=0)@ 的返回值,即 \lstinline@c@ 。因为此时 \lstinline@c@ 的值为 \lstinline@0@,所以 \lstinline@b@ 的值变为 \lstinline@0@。最后返回 \lstinline@b@ 的引用。
    \item \lstinline@a=(b=(c=0))@ 中的赋值运算符将左操作数 \lstinline@a@ 的值变为 \lstinline@(b=(c=0))@ 的返回值,即 \lstinline@b@。因为此时 \lstinline@b@ 的值为 \lstinline@0@,所以 \lstinline@a@ 的值变为 \lstinline@0@。最后返回 \lstinline@a@ 的引用。
\end{enumerate}
在上述过程中,每个赋值运算符都接收了两个操作数,并给出了一个返回值。当然,\lstinline@a=(b=(c=0))@ 的返回值就不需要使用了,因为我的目的已经达成。如果你愿意,还要以加上一个 \lstinline@cout<<@ 来输出 \lstinline@a@ 的值。
\begin{lstlisting}
    cout << (a = (b = (c = 0))); //输出a=(b=(c=0))的值,应为0
\end{lstlisting}\par
再举一例。\lstinline@<<@ 在左操作数是整型数据的时候,意思是左移位运算符。而C++中重载了这个运算符\footnote{对于重载后的运算符,我们仍然称其为左移运算符,但它的功能已经发生了实际上的变化。},当其左操作数为 \lstinline@ostream@ 类的对象时,实现输出功能,最后返回左操作数的引用。C++中我们最常用的 \lstinline@ostream@ 类的对象就是 \lstinline@cout@,所以我们可以用它和 \lstinline@<<@ 操作符实现连续输出。
\begin{lstlisting}
    const double a {3.6e3}, b {6.02e-23}; //定义两个double常量并初始化
    cout << a << '\n' << b; //输出a,再输出换行符,再输出b
\end{lstlisting}
我们按照相同的思路来解释一下这段代码的行为:
\begin{enumerate}
    \item \lstinline@cout<<a@ 中的左移运算符的左操作数是 \lstinline@cout@,所以标准输出将把右操作数 \lstinline@a@ 的值输出到相应设备上。最后返回 \lstinline@cout@ 的引用。
    \item \lstinline@cout<<a<<'\n'@ 中的第二个左移运算符的左操作数是 \lstinline@cout<<a@ 的返回值,即 \lstinline@cout@,所以标准输出将把右操作数 \lstinline@'\n'@ 这个转义字符输出到相应高备上(起到换行作用)。最后返回 \lstinline@cout@的引用。
    \item \lstinline@cout<<a<<'\n'<<b@ 中的第三个左移运算符的左操作数是 \lstinline@cout<<a<<'\n'@ 的返回值,即 \lstinline@cout@,所以标准输出将把右操作数 \lstinline@b@ 的值输出到 相应设备上。最后返回 \lstinline@cout@ 的引用(这个返回值我们不再需要用到了)。
\end{enumerate}\par
\lstinline@>>@ 叫做右移位运算符。C++中重载了这个运算符,当其左操作数为 \lstinline@istream@ 类的对象时,实现输入功能,最后返回左操作数的引用。C++中我们最常用的 \lstinline@istream@ 类的对象就是 \lstinline@cin@,所以我们可以用它和 \lstinline@>>@ 运算符实现连续输入。
\begin{lstlisting}
    int i, j, k; //定义int数据i,j,k。不能定义为常量,否则将发生错误
    cin >> i >> j >> k; //先后为i,j,k连续输入
    cout << i << endl << j << endl << k; //连续输出i,j,k并以换行为间隔
\end{lstlisting}
读者可以仿照上例,对本例中连续输入的语法进行解读。\par
\subsection*{运算符的优先级}
在上小学时我们就学过``先乘除,后加减''这类的口诀。这其实就体现出一个先后顺序的问题,亦即``先算什么,后算什么''。C++中有数十种运算符,除了我们熟知的四则运算以外,还有赋值运算符、模运算符和移位运算符等等。如果我们不人为规定一个``先操作什么,后操作什么''的顺序的话,有些代码的理解处理将会变得很麻烦。我们举个例子:
\begin{lstlisting}
    num = 6 * 7; //假设num是一个已经定义了的int型变量
\end{lstlisting}\par
如果先操作乘法运算符,再操作赋值运算符的话,应当是这样的过程:\lstinline@6*7@ 的返回值作为赋值运算符的右操作数,经由赋值运算符把它赋给 \lstinline@num@。赋值运算符的返回值不需要再使用。最终 \lstinline@num@ 变为 \lstinline@42@。\par
假如先操作赋值运算符,再操作乘法运算符呢?(虽然对于人来说有点费解,但电脑不会怀疑,只会默默执行)它先将 \lstinline@6@ 赋值给 \lstinline@num@,然后将赋值运算符的返回值,即 \lstinline@num@,与 \lstinline@7@ 相乘。乘法运算的返回值没有被使用。最终 \lstinline@num@ 变为 \lstinline@6@。\par
再比如,\lstinline@a+b*c@ 究竟是先算加法还是先算乘法,结果很有可能是不同的。因而我们需要人为规定,某个运算符要优先于某个运算符来计算——这就是\textbf{优先级(precedence)}的问题。\par
在初学阶段,我建议读者把``优先计算''理解成``套括号''。C++规定乘法运算符的优先级比加法要高,那么我们就为乘法运算符和乘法运算符所操作的数套上括号,变成 \lstinline@a+(b*c)@。套上括号,就意味着 \lstinline@(b*c)@ 是一个整体,它的返回值将会作为加法运算的右操作数。\par
C++中的赋值运算符优先级很低,比四则运算都要低。那么按照我们的思路,如果加法,乘法和赋值运算符同时出现,我们应该先给乘法套括号,然后是加法,最后才是赋值。如图2.1所示。\par
\begin{figure}[htbp]
    \centering
    \includegraphics[width=0.5\textwidth]{../images/generalized_parts/02_Precedence_of_multiply_addition_and_assignment_300.png}
    \caption{代码 \lstinline@num=a*b+c*d;@ 的套括号解释}
\end{figure}
再看这段代码:
\begin{lstlisting}
    cout << 3 + 4; //输出3+4的值
\end{lstlisting}
在C++中加法运算符的优先级高于移位运算符,所以它就可以解释为 \lstinline@cout<<(3+4)@,就会输出 \lstinline@7@。\par
比较运算符,包括大于号 \lstinline@>@,小于号 \lstinline@<@,相等号 \lstinline@==@\footnote{在C++中,单个等号 \lstinline@=@ 是赋值运算符,而 \lstinline@==@ 是一个比较运算符。},不等号 \lstinline@!=@ 大于或等于 \lstinline@>=@,小于或等于 \lstinline@<=@,它们会比较左右操作数的关系,然后以 \lstinline@bool@ 数据的形式给出一个逻辑判断。正如前文所说,我们可以用 \lstinline@cout@ 来输出一个布尔型数据(只是默认以数值的方式输出),比如这样:
\begin{lstlisting}
    cout << 3 != 4; //输出3!=4的结果,预期是ture,输出是1
//error: no match for 'operator!=' (operand types are
//'std::basic_ostream<char>' and 'int')
\end{lstlisting}
但是这段代码在编译期就出现了问题,而报错信息也显得十分奇怪:``没有匹配的运算符 \lstinline@!=@(操作数类型分别是 \lstinline@std::basic_ostream<char>@ 和 \lstinline@int@)。''我们甚至都搞不懂为什么会出现这样的报错信息!\par
但是解决方法也很简单,只需为 \lstinline@3!=4@ 套上括号:
\begin{lstlisting}
    cout << (3 != 4); //输出3!=4的结果,编译正常,并输出1
\end{lstlisting}\par
为什么是这样呢?原因在于,\lstinline@<<@ 的优先级要高于这一系列的比较运算符。所以如果按照 \lstinline@cout<<3!=4@ 的写法的话,\lstinline@cout<<3@会先套上括号,于是变成了 \lstinline@(cout<<3)!=4@。显然,\lstinline@cout@(也就是 \lstinline@cout<<3@ 的返回值)和 \lstinline@4@ 比较是没有意义的,C++也没有定义这种语法,于是才有了我们之前遇到的问题,即``没有匹配的运算符''。正因如此,我们才需要把 \lstinline@3!=4@ 套上括号,以此来表示它要作为一个整体来计算。\par
而假如我们需要为一个 \lstinline@bool@ 变量来赋值的话,那就不会有这样的问题了,因为比较运算符的优先级要高于赋值运算符。\par
\begin{lstlisting}
    bool judgement; //定义bool型变量judgement,不初始化
    judgement = 2 * 5 > 9; //judgement应当被赋值为1
\end{lstlisting}
在这个赋值语句中,乘法运算符的优先级最高,大于号次之,赋值运算符最低,于是它的正确表达就是 \lstinline@judgement=((2*5)>9);@。\par
C++中的运算符实在太多,优先级规则也相当复杂,这里无法尽数讲解,读者可以参考\hyperref[ch:appendix_A]{附录A}的内容。在后面的章节中我们会接触到更多运算符,到那时我们会再细致讲解。\par
\subsection*{运算符的结合性}
那么同一优先级的运算符又该按照什么顺序来计算呢?这就涉及到结合性的问题了。\par
举一个简单的例子:
\begin{lstlisting}
    cout << 8 / 4 / 2; //结果会是什么?
\end{lstlisting}
我们知道 \lstinline@/@ 的优先级比 \lstinline@<<@ 要高,所以套括号的顺序一定是先给除法运算符套括号。但是这里有两个除法运算符,究竟是套成 \lstinline@((8/4)/2)@ 还是 \lstinline@(8/(4/2))@ 呢?这就要涉及到结合性的问题了。\par
C++中规定了每个运算符的\textbf{结合性(Associativity)}。对于同一优先级\footnote{相同的运算符一定是同一优先级。当然,同一个优先级可能包含若干种运算符,比如乘法、除法和模运算这三种运算符就属于同一优先级。}来说,如果它的优先级是从左向右(Left-to-right),那么套括号的顺序就是从左向右;如果它的优先级是从右向左(Right-to-left),那么套括号的顺序就是从右向左。\par
除法运算符所在的优先级,其结合性是从左向右的,所以在乘法、除法和取模混合运算时,它就会从左向右套括号,而直观表现上就是``从左向右算''。赋值运算符所在的优先级,其结合性是从右向左的,所以在连续赋值的时候,它就会从右向左套括号,而直观表现上就是``从右向左算''。\par
\begin{figure}[htbp]
    \centering
    \includegraphics[width=0.4\textwidth]{../images/generalized_parts/02_Associativity_of_division_and_assignment_300.png}
    \caption{除法运算符和赋值运算符的结合性}
\end{figure}
所以连续除法中 \lstinline@8/4/2@ 就应该解释成 \lstinline@(8/4)/2@ 而非 \lstinline@8/(4/2)@;而连续赋值中 \lstinline@a=b=0@ 就应该解释成 \lstinline@a=(b=0)@ 而非 \lstinline@(a=b)=0@。如图2.2所示。\par
综上所述,\textbf{运算符的优先级决定了不同种运算符套括号的顺序,而结合性决定了同种运算符套括号的顺序}。于是编译器可以通过这套规则,给每个运算符都``套起括号'',从而让每个运算符都知道它的操作数是什么,而不至于出现歧义。\par
当然我也需要提醒读者,``为所有运算符都套括号''只是我们初学编程时的权宜之计;实际编程中,只有当默认的优先级与结合性不能满足我们的要求时,我们才通过加括号的方式改变运算逻辑(比如,在 \lstinline@(a+b)*c@ 中加括号是为了先算加法后算乘法)。当我们把这些知识掌握熟练了之后,我们自然就清楚在什么时候有必要加括号,而那些可有可无的括号就不会再用了。\par
\subsection*{运算的语义}
我们在实际编程的时候经常会犯一些\textbf{语法(Syntax)}上的错误,包括但远不限于:
\begin{itemize}
    \item 给常量赋值。这是因为常量不能赋值,编译器会禁止这种行为。
    \item 用字面量进行统一初始化时发生了收缩转换。比如某电脑的 \lstinline@short@ 类型占2字节,但使用了 \lstinline@short s {40000};@ 这样的语法。
    \item 写错了变量名,比如我定义的是 \lstinline@num@ 结果不小心写成 \lstinline@nun@,然后编译器就会报错,理由是``\lstinline@nun@''未定义。
    \item 写出 \lstinline@cout<<a<b@ 这种表达式。可能你想输出 \lstinline@a<b@ 的结果但是忘记套括号了;也可能你想先后输出 \lstinline@a@ 和 \lstinline@b@ 但是不小心把 \lstinline@<<@ 写成了 \lstinline@<@。但无论如何,编译器都会以``没有匹配的运算符''为由报错。
\end{itemize}
这些语法上的错误都很容易被编译器查出来,然后告诉你。你可能一眼就知道怎么回事,也可能花了好久才终于找到了问题所在,总是这些语法错误都是会在编译期由编译器告诉你的。\par
但是还有一类\textbf{语义(Semantics)}上的错误,包括但远不限于:
\begin{itemize}
    \item 不小心把 \lstinline@a==3@ 写成了 \lstinline@a=3@。前一个是进行相等性判断,而后一个是赋值。这种失误常见于新手,但对于老手来说也在所难免。
    \item 受到数学表达的影响,在C++代码中写出诸如 \lstinline@2<x<10@ 这样的表达式。
    \item 想要计算 \lstinline@8/3*1.@ 的浮点结果,但是算出来的却是整型的,原因不明。
\end{itemize}
语义错误的特点是,编译器不会认为你的代码是``错误''的。它会认为你写的代码都有你自己的目的。只要不违反C++标准对于语法的规定,你就可以想怎么写就怎么写。但是很多时候,因为各种原因,``我们真正写出来的代码''并没有符合``我们想要实现的功能'',这时我们就会说,你的代码\textbf{在语法上正确,但在语义上错误}。这种语义上的错误是很可怕的,因为很多编译器不会给你提醒,你也不会立刻知道哪里出错,只有在你运行程序并发现结果不合预期之后,才会去找寻问题的来源。\par
\lstinline@2<x<10@ 这句代码就属于一种语义错误,因为C++没有三元比较运算符\footnote{C++中唯一的三元运算符是条件运算符,它可以有三个操作数。我们之后再谈。},我们不能用这样的方式来进行比较。但是它在语法上正确,所以编译器会作如下分析:因为 \lstinline@<@ 的结合性是从左向右,那么\lstinline@2<x<10@ 可以被解释成 \lstinline@(2<x)<10@。\par
那么这样的解释方式能否达到我们的目的呢?我们分析一下就知道了:无论 \lstinline@2<x@ 的返回值是 \lstinline@true@ 还是 \lstinline@false@,它都小于 \lstinline@10@\footnote{\lstinline@bool@ 型数据和 \lstinline@int@ 型数据比较的时候会发生类型转换,这点我们会在之后提及。},所以 \lstinline@2<x<10@ 得到的返回值永远是 \lstinline@true@!\par
那么怎样写才能达到我们的目的呢?这里要用到逻辑运算符``\lstinline@&&@''(逻辑与)。逻辑与运算符用于计算两个布尔数据,仅当左右操作数均为 \lstinline@true@ 时,返回值才为 \lstinline@true@;否则,返回值为 \lstinline@false@。\par
那么我们拆解一下数学表达式$2<x<10$,它的含义可以阐述为``$2<x$''并且``$x<10$'',必须两个要求都满足。于是我们可以通过逻辑与运算符来把它们拼合起来,写成 \lstinline@(2<x)&&(x<10)@。\par
又因为 \lstinline@&&@ 运算符的优先级低于 \lstinline@<@,所以我们可以不用套手动括号了,直接写成 \lstinline@2<x && x<10@,这才是语义上正确的写法。\par
至于 \lstinline@8/3*1.@ 为什么在语义上是错误的,我们放到下节的类型转换部分探讨。\par
