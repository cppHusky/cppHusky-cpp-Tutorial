\section{实操:简单矩阵类的设计}
本节我们来设计一个简单的矩阵类。C++没有原生的矩阵功能,我们需要从零开始自己搭建一个 \lstinline@Matrix@ 类模板。\par
\subsection*{功能设计}
首先,这个类模板只有一个参数,那就是每个矩阵元素的类型。有些读者可能会想到把矩阵的行数 \lstinline@N@ 和列数 \lstinline@M@ 也作为模板参数,但这样就意味着任何一个矩阵的尺寸必须在编译时确定下来——一个矩阵的大小应该不变,这是合理的;但是一个矩阵的大小要在定义时就确定下来,这就未免有些严格了。所以比较好的处理方法是把矩阵的行数和列数当作常量,一经定义就不可改变;但是定义它的时候传入什么参数,这还是可以在运行时决定的。\par
接下来我们思考一下矩阵元素要用什么来存储。理论上讲,\lstinline@std::vector@, \lstinline@std::valarray@ 这种可以在运行时确定长度的数组套数组都是没问题的,甚至用一个二阶指针配合动态内存分配也行;而 \lstinline@std::array@ 和 \lstinline@int[N][M]@ 这种需要在编译时确定大小的就不行了。在这里我们选择 \lstinline@std::valarray@,这是因为 \lstinline@valarray@ 本身就支持数组之间的算术运算,这样我们写起代码来会比较省事。\par
这个矩阵类应该能支持列表初始化,或者根据某些参数来确定其大小(因为大小是常量,必须在定义时确定)。至于析构,如果我们选择 \lstinline@std::valarray@ 或 \lstinline@std::vector@ 的话,那我们没必要自己写一个析构函数了,它有自己的析构函数。\par
这个矩阵类要能支持根据下标取元素。很遗憾,在C++17我们还不能进行多参数的下标运算符重载(比如写成 \lstinline@mat[i,j]@ 这样)。但是我们可以选择另一种方式,用函数调用运算符(比如写成 \lstinline@mat(i,j)@ 这样)来替代之。我们可以为它设计范围检测,如果发生了越界行为,我们就抛出异常。\par
这个矩阵类要能支持加法、减法和乘法。至于取逆运算,我们就不实现了。在进行这些运算的时候要注意一个问题:矩阵行数或列数不同的话,可是不能相加减的;左矩阵列数与右矩阵行数不同的话,也是不能相乘的。如果发生了这些情况,怎么办?我们选择抛出异常。\par
这个矩阵类还可以支持六种比较运算符。在比较时仍需注意,不同大小的矩阵没有什么比较的意义,所以对于这种情况,我们可以抛出异常。(当然,如果你有其它的理解方式和实现方式,那也未尝不可)\par
这个矩阵要能支持转置操作。其实不难,就是把行与列对调。\par
这个矩阵还要能够返回行数与列数。对此,我们有两种实现方式:其一是,定义两个成员函数 \lstinline@row_num@ 和 \lstinline@col_num@,分别返回行数与列数;其二是,用一个 \lstinline@std::pair@ 对象作返回值,把行数与列数打包进同一个对象中再返回。此处我们选择第二种方式;读者可以自行尝试第一种方式,很容易就写出来了。单纯返回个值,不涉及什么危险操作,我们可以放心地把它设为 \lstinline@noexcept@。\par
至于其它的功能,比如分块、合并、特征值之类的,我们就不实现了。\par
\subsection*{规划}
首先我们把这个类模板的成员类型和私有成员声明一下。多使用成员类型信息是一种好习惯。
\begin{lstlisting}
//namespace user
template<typename T>
class matrix { //类模板定义
public:
    using value_type = T;
    using reference = T&;
    using const_reference = const T&;
    using size_type = std::size_t;
private:
    const size_type N;
    const size_type M;
    std::valarray<std::valarray<value_type>> _element;
    //...待补充
\end{lstlisting}\par
\subsubsection*{自定义异常类}
在上文中,我们提及的异常只有两类:一类是下标错误,另一类是矩阵尺寸不合适。不过我们可以稍微细分一下,变成四类:下标错误、不允许的加减法、不允许的乘法和不允许的比较。
这样一来,我们可以为不同的错误种类设置不同的异常信息(也就是体现在 \lstinline@what@ 中的部分)。\par
我们可以在 \lstinline@matrix@ 的私有成员部分定义几个异常类,它们可以归为 \lstinline@std::out_of_range@ 或者 \lstinline@std::logic_error@,那么我们就用继承的语法给它们表示出来。\par
\begin{lstlisting}
private:
    struct index_error : std::out_of_range { //有两种构造函数和默认拷贝构造函数
        index_error(const char *msg) : out_of_range {msg} {}
        index_error(const std::string &msg) : out_of_range {msg} {}
    };
    struct invalid_plus : std::logic_error {
        invalid_plus(const char *msg) : logic_error {msg} {}
        invalid_plus(const std::string &msg) : logic_error {msg} {}
    };
    struct invalid_multiplies : std::logic_error {
        invalid_multiplies(const char *msg) : logic_error {msg} {}
        invalid_multiplies(const std::string &msg) : logic_error {msg} {}
    };
    struct invalid_compare : std::logic_error {
        invalid_compare(const char *msg) : logic_error {msg} {}
        invalid_compare(const std::string &msg) : logic_error {msg} {}
    };
\end{lstlisting}
它们每个都可以接收 \lstinline@const char*@ 参数或 \lstinline@cons std::string&@ 参数来构造(因为它们的基类也能)。不过根据我们上一节中积累的经验,还是选择 \lstinline@const std::string&@ 更好一些。\par
\subsubsection*{其它成员函数}
关于构造函数,我们可以接收两个 \lstinline@size_type@ 参数,来决定矩阵的大小;或者允许列表初始化,接收一个 \lstinline@std::initializer_list<std::initializer_list<value_type>>@,于是我们就可以像这样初始化一个矩阵对象了:
\begin{lstlisting}
    user::matrix<int> mat {
        {1},
        {0,1},
        {0,0,1}
    }; //定义一个三阶单位矩阵
\end{lstlisting}
至于拷贝构造函数,用默认的就够了。析构函数也是。
\begin{lstlisting}
public:
    matrix(size_type, size_type); //接收两个矩阵大小参数,所有元素置为0
    matrix(std::initializer_list<std::initializer_list<value_type>>);
    //允许二维数组式的列表初始化
\end{lstlisting}\par
接下来是元素访问。我们已经说过,下标运算符不能接收多个参数,因此我们使用的替代方案是重载函数调用运算符 \lstinline@()@。这里有常成员函数版本和非常成员函数版本,它们的返回值分别是常量引用和引用。\par
\begin{lstlisting}
public:
    reference operator()(size_type, size_type);
    const_reference operator()(size_type, size_type)const;
\end{lstlisting}\par
下面是一系列算术运算。加法和减法自不必说;乘法要分为三种情况:矩阵乘矩阵、数乘矩阵、矩阵乘数。后两种其实是一样的,我们可以通过代码重用来简化。在这里我尽可能把它们写成非成员函数,这样能很好地兼容``二维列表加矩阵''的写法——因为二维列表可以隐式类型转换成矩阵。\par
\begin{lstlisting}
//非成员函数声明部分
template<typename T>
matrix<T> operator+(const matrix<T>&, const matrix<T>&); //非成员函数声明
template<typename T>
matrix<T> operator-(const matrix<T>&, const matrix<T>&);
template<typename T>
matrix<T> operator*(const matrix<T>&, const matrix<T>&);
template<typename T>
matrix<T> operator*(const T&, const matrix<T>&);
//友元/成员函数声明部分
public:
    template<typename U>
    friend matrix<U> operator+(const matrix<U>&, const matrix<U>&); //矩阵相加
    template<typename U>
    friend matrix<U> operator-(const matrix<U>&, const matrix<U>&); //矩阵相减
    template<typename U>
    friend matrix<U> operator*(const matrix<U>&, const matrix<U>&); //矩阵相乘
    template<typename U>
    friend matrix<U> operator*(const U&, const matrix<U>&); //数乘以矩阵
    matrix operator*(const T&); //矩阵乘以数
\end{lstlisting}
在声明友元的时候还是不要忘了,模板参数不要用和 \lstinline@matrix@ 一样的。\par
至于六个比较运算符的声明/友元声明,写法大抵一致,我就不在这里写了。读者若要参考,请看本节末的完整代码。\par
转置函数有不同的写法:可以直接操作这个对象本身,使其转置;也可以不操作这个对象本身,而是返回另一个临时对象。我们在这里采取第二种实现方式。
\begin{lstlisting}
public:
    matrix transpose()const;
\end{lstlisting}\par
至于 \lstinline@size@,这里我们会用到 \lstinline@utility@ 库中提供的 \lstinline@std::pair@。它是一个类模板,其对象可以表示一个 \lstinline@<T1,T2>@ 二元组。说得好像很高大上,其实就是包装了两个成员的对象而已。我们可以用 \lstinline@first@ 取其第一个成员,\lstinline@last@ 取其第二个成员。至于构造,我们把两个对象装进函数参数中就行,像这样:
\begin{lstlisting}
    std::pair<int, double> pr(1, 0); //两个成员分别是1和0
    std::cout << pr.first << ' ' << pr.second; //分别输出它的两个成员
\end{lstlisting}
回到 \lstinline@size@,我们只需要把行数和列数信息包装到一个 \lstinline@std::pair@ 中就好。
\begin{lstlisting}
public:
    std::pair<size_type, size_type> size()const noexcept;
\end{lstlisting}\par
\subsection*{实现}
\subsubsection*{成员函数}
我们先实现一下成员函数中的构造函数。
\begin{lstlisting}
template<typename T>
user::matrix<T>::matrix(size_type n, size_type m) : N {n}, M {m} {
    //常量成员必须通过初始列来初始化
    _element.resize(N); //先改变_element的大小,这是行数
    for (auto &row : _element) //这样写方便些,注意row是引用形式
        row.resize(M); //对每一行进行resize,M为列数
    //resize之后每个元素都是默认值value_type{},这样,一个矩阵构造就完成了
}
\end{lstlisting}
这个构造函数写起来还是比较容易的,因为行数和列数都直接给到参数当中了。对于 \lstinline@_element@ 来说,因为它是一个``数组的数组'',所以我们第一步需要通过 \lstinline@resize@ 来改变数组的个数——也就是矩阵的行数。接下来,我们通过一个范围 \lstinline@for@ 循环\footnote{值得注意的是,虽然 \lstinline@std::valarray@ 没有内置的迭代器,但是它提供了 \lstinline@begin@ 和 \lstinline@end@,这使它具备了能在范围 \lstinline@for@ 循环中正常使用的资格。}遍历每一行(每一行又是一个 \lstinline@std::valarray@ 对象),对每一行再通过 \lstinline@resize@ 改变列数。这样,一个矩阵的构造工作就完成了。其中的每个元素都取 \lstinline@value_type@ 的默认值——对于内置类型来说就是各种零(\lstinline@0@, \lstinline@0.@, \lstinline@'\0'@, \lstinline@false@, \lstinline@nullptr@ 等)。\par
至于用二维初始化列表的写法,就显得有些难度了。倘若我们要允许以下面的语法定义一个三阶单位矩阵的话,我们就不仅要根据 \lstinline@ilist.size()@ 确定 \lstinline@N@ 的值,还要根据 \lstinline@ilist@ 每个成员的 \lstinline@size()@ 确定 \lstinline@M@ 的值。
\begin{lstlisting}
    user::matrix<int> mat {
        {1},
        {0,1},
        {0,0,1}
    }; //定义一个三阶单位矩阵,N=3,M=3
\end{lstlisting}
但是初值列中不允计我们进行额外的计算(除非用闭包之类的),我们只能为成员提供值。为了解决这个问题,我们还需要补充一个辅助函数,它可以帮助我们根据二维初始化列表,确定合理的矩阵列数。\par
我们希望正确的列数是 \lstinline@ilist@ 各行长度的最大值,那么我们可以把辅助函数写成这样:
\begin{lstlisting}
//声明
template<typename T>
std::size_t max_column(
    const std::initializer_list<std::initializer_list<T>>&
); //辅助函数,取二维列表各行的最大值
//定义
template<typename T>
std::size_t user::max_column(
    const std::initializer_list<std::initializer_list<T>> &ilist
) {
    std::size_t result {}; //初始化为0
    for (auto row : ilist)
        if(result < row.size())
            result = row.size();
    return result; //返回的就是二维列表各行的最大值
}
\end{lstlisting}
接下来,我们就可以把初值列完成:
\begin{lstlisting}
user::matrix<T>::matrix(
    std::initializer_list<std::initializer_list<value_type>> ilist
) : N {ilist.size()}, M {user::max_column(ilist)} { /*...*/ }
    //在初值列中通过ilist和辅助函数确定行/列数
\end{lstlisting}\par
再来考虑函数体内的操作。\lstinline@std::initializer_list@ 有一个十分不便的点在于,我们只能通过迭代器来遍历它的每一个元素。至于下标运算符或者 \lstinline@at@ 成员函数,它都是没有的。为此,我们需要在 \lstinline@for@ 之外的部分定义循环变量,并进行循环迭代。以下是函数体的部分:
\begin{lstlisting}
    _element.resize(N); //同样的道理,先改变_element的大小,代表行数
    size_type i {0}; //循环变量
    for (const auto &row : ilist) { //访问initializer_list不能用下标运算符
        _element[i].resize(M); //改变_element的大小,代表列数
        size_type j {0}; //也是循环变量
        for (const auto &x : row) {
            _element[i][j++] = x; //注意++是后缀形式
        }
        ++i; //别忘了循环迭代
    } //这样,一个矩阵构造就完成了
}
\end{lstlisting}
这里的代码还是有些繁琐,读者可以自行尝试写一遍,能加深理解。\par
接下来是成员访问,我们选择重载函数调用运算符。它有异常检测机制,如果下标不正确,它就会抛出一个 \lstinline@index_error@ 类型的异常信息。只有检测无误,才正常返回结果。
\begin{lstlisting}
template<typename T>
auto user::matrix<T>::operator()(size_type i, size_type j) -> reference {
    if (i >= N) {
        std::string msg {"Error: Row index "};
        msg += std::to_string(i) + " >= " + std::to_string(N);
        throw typename matrix<T>::index_error {msg};
        //靠加法运算符疯狂拼接,然后抛出
    }
    if (j >= M) {
        std::string msg {"Error: Column index "};
        msg += std::to_string(j) + " >= " + std::to_string(M);
        throw typename matrix<T>::index_error {msg}; //同上
    }
    return _element[i][j];
}
\end{lstlisting}
在这里我们仍然使用了 \lstinline@auto@ 配合尾随返回类型来简化写法。函数体内的异常处理看起来很复杂,其实都是一些 \lstinline@string@ 拼接的操作。只有最后的 \lstinline@throw@ 语句这里,需要提醒读者,编译器未必能识别出 \lstinline@matrix<T>::index_error@ 是个类型名,所以我们需要以 \lstinline@typename@ 指明它是一个类型,进而才能构造一个该类型的对象。总之,\lstinline@typename@ 在这里只有提示作用。\par
至于常成员函数的版本,道理一样。尾随返回类型改成 \lstinline@const_reference@ 就好了。\par
接下来看一下矩阵乘数的乘法运算符,它也是成员函数。我们需要先拷贝构造一个临时对象,然后用循环语句对每一个矩阵元素赋值。这里就显现出 \lstinline@std::valarray@ 的强大之处了:我们可以直接用乘赋值运算符 \lstinline@*=@,来让一整行的数据都乘上一个数!
\begin{lstlisting}
template<typename T>
auto user::matrix<T>::operator*(const T &val) -> matrix {
    matrix mat {*this}; //用*this来拷贝构造mat
    for (auto &row : mat._element) //遍历每一行,每一个row都是一个valarray引用
        row *= val; //std::valarray特有的乘赋值功能
    return mat; //返回值
}
\end{lstlisting}\par
接下来是转置函数。很简单,我们只需定义一个行列数相反的临时对象,然后把对应位置的元素赋值到转置的位置上就可以了。
\begin{lstlisting}
template<typename T>
auto user::matrix<T>::transpose()const -> matrix{
    matrix mat(M,N); //行列数颠倒
    for (size_type i = 0; i < N; ++i)
        for (size_type j = 0; j < M; ++j)
            mat(j,i) = operator()(i,j); //注意i,j别写错了,想清楚
    return mat;
}
\end{lstlisting}\par
最后的 \lstinline@size@ 很简单,我们用一个 \lstinline@std::pair@ 构造函数包装一下就好。这里的 \lstinline@std::pair@ 不需要 \lstinline@typename@ 关键字来提示,因为编译器知道它是一个类名。
\begin{lstlisting}
template<typename T>
auto user::matrix<T>::size()const noexcept
-> std::pair<size_type, size_type> {
    return std::pair<size_type, size_type> {N, M}; //不需要typename,但可以有
}
\end{lstlisting}\par
\subsubsection*{非成员函数/辅助函数}
加法和减法的套路是一样的,我们只写加法。
\begin{lstlisting}
template<typename T>
auto user::operator+(const matrix<T> &lhs, const matrix<T> &rhs)
-> matrix<T> {
    if (lhs.N != rhs.N || lhs.M != rhs.M) {
        std::string msg {"Cannot add "};
        msg += std::to_string(lhs.N) + " * " + std::to_string(lhs.M)
            + " matrix to " + std::to_string(rhs.N)
            + " * " + std::to_string(rhs.M) + "matrix";
        throw typename matrix<T>::invalid_plus {msg};
    }
    matrix mat {lhs}; //用lhs为mat赋值
    for (std::size_t i = 0; i < lhs.N; ++i)
        mat._element[i] += rhs._element[i]; //再加上rhs的每一行
    return mat;
}
\end{lstlisting}
首先是异常处理,我们不允许行数或列数不同的矩阵相加,所以遇到这种情况要抛出异常。至于计算本身,我们还是利用 \lstinline@std::valarray@ 对于数组加/减法的支持,直接每行每行地相加就好。\par
矩阵乘法不易用 \lstinline@std::valarray@ 本身的功能来实现(能倒是能,但代码更复杂了,效率也更低了),我们老老实实地写三重循环就好。
\begin{lstlisting}
template<typename T>
auto user::operator*(const matrix<T> &lhs, const matrix<T> &rhs)
-> matrix<T> {
    if (lhs.M != rhs.N) {
        std::string msg {"Cannot multiply matrix with "};
        msg += std::to_string(lhs.M) + " columns by matrix with "
            + std::to_string(rhs.N) + " rows";
        throw typename matrix<T>::invalid_multiplies{msg};
    }
    matrix<T> mat(lhs.N, rhs.M); //新建一个空的矩阵
    for (std::size_t i = 0; i < lhs.N; ++i)
        for (std::size_t j = 0; j < rhs.M; ++j)
            for (std::size_t k = 0; k < lhs.M; ++k)
                mat(i,j) += lhs(i,k) * rhs(k,j); //三重循环,就不讲了
    return mat;
}
\end{lstlisting}
至于这个三重循环的逻辑,那就不是本书该讲的东西了,请读者自行查阅相关资料吧。\par
数乘矩阵的功能也可以仿照矩阵乘数的代码来写,不过我们也可以直接选择代码重用。
\begin{lstlisting}
template<typename T>
auto user::operator*(const T &value, const matrix<T> &mat) -> matrix<T> {
    return mat * value; //节省代码,就这样写了
}
\end{lstlisting}
在六个比较运算符方面,我们也可以选择使用 \lstinline@<@ 定义其它五个运算符;不过对于相等/不相等运算符来说,这样做的效率会比较低。对于矩阵来说,一次二重循环比较带来的开销还是不小的,如果我们通过一次比较就能判断是否相等,那就没必要用两次比较。因此我们的思路是:重载小于号,并用它来定义大于号、不小于号、不大于号;重载不相等号,并用它来定义相等号。
\begin{lstlisting}
template<typename T>
bool user::operator<(const matrix<T> &lhs, const matrix<T> &rhs) {
    if (lhs.N != rhs.N || lhs.M != rhs.M) {
        std::string msg {"Cannot compare "};
        msg += std::to_string(lhs.N) + " * " + std::to_string(lhs.M)
            + " matrix with " + std::to_string(rhs.N) + " * "
            + std::to_string(rhs.M) + " matrix";
        throw typename matrix<T>::invalid_compare {msg};
    }
    for (std::size_t i = 0; i < lhs.N; ++i)
        if (lhs._element[i] < rhs._element[i])
            return true;
        else if (lhs._element[i] > rhs._element[i])
            return false;
    return false;
}
\end{lstlisting}
在这里,出于代码简化的目的,我们直接使用了 \lstinline@std::valarray@ 自带的小于号。
\begin{lstlisting}
template<typename T>
bool user::operator!=(const matrix<T> &lhs, const matrix<T> &rhs) {
    if (lhs.N != rhs.N || lhs.M != rhs.M) {
        std::string msg {"Cannot compare "};
        msg += std::to_string(lhs.N) + " * " + std::to_string(lhs.M)
            + " matrix with " + std::to_string(rhs.N) + " * "
            + std::to_string(rhs.M) + " matrix";
        throw typename matrix<T>::invalid_compare {msg};
    }
    for (std::size_t i = 0; i < lhs.N; ++i)
        if (lhs._element[i] != rhs._element[i])
            return true;
    return false;
}
\end{lstlisting}
其它四个运算符都可以靠它们两个来实现,我就不赘述了。\par
\subsection*{完整代码}
以下是 \lstinline@user::matrix@ 类声明及定义的完整代码。
\lstinputlisting[caption=\texttt{Header.hpp}]{code_in_book/12.2-12.3/Header.hpp}
\lstinputlisting[caption=\texttt{Definition.tpp}]{code_in_book/12.2-12.3/Definition.tpp}
