\section{作用域、存储期与链接}
在前面的章节中我们已经介绍过作用域的概念。我曾说,作用域的存在是为了控制变量的生存期,其实这是很片面的见解。更准确的说法是:\textbf{作用域(Scope)}和\textbf{存储类说明符(Storage class specifiers)}二者共同作用,控制一个标识符\footnote{未必是变量,也可以是函数和类等}的\textbf{生存期(Storage Duration)}和\textbf{链接方式(Linkage)}。本节我们就来详细讲讲相关知识。\par
\subsection*{作用域}
作用域像是程序中的一片领地或者范围,作用域之间相互嵌套,则有点像是行政区划。一个外层作用域嵌套了若干内层作用域,它对自身和内层作用域都有管辖权。而内层作用域仅仅能管理自己的一亩三分地,对外界没有任何作用。\par
最外层的作用域叫作\textbf{全局作用域(Global scope)}。我们在全局作用域中定义的类、函数或变量,对所有内层作用域都是可见的\footnote{作用域是针对编译时的概念,这决定了编译器会怎样解释代码。而在运行时,编译器早已解决代码如何解释的问题,因而``作用域''这个概念也就不再需要;而我们讨论作用域时,当然也不需要思考某段代码是如何运行的。}。\par
当我们声明一个函数时,这个函数就引入了一个\textbf{函数形参作用域(Function parameter scope)}。
\begin{lstlisting}
void func1(int a); //a位于函数形参作用域中
void func2(int a); //这两个a并不相同
\end{lstlisting}
函数形参作用域会在分号之后结束。所以一旦离开了这个函数,\lstinline@a@ 就变得不可见了。如果是函数定义的话,那么函数形参作用域会延续到函数定义的末尾。
\begin{lstlisting}
constexpr double Pi {3.1415926}; //Pi在全局作用域中
double circ_area(double r) { //在定义之时,函数形参作用域开始
    return Pi * r * r; //函数形参作用域延续到函数定义中;全局作用域中的Pi也可见
}
\end{lstlisting}\par
当我们定义一个结构体、共用体或类时,它们就会引入一个\textbf{类作用域(Class scope)}。如果我们在其中定义了别的类或者成员函数,那么它们也会引入新的作用域。\par
\begin{lstlisting}
class dynamic { //struct引入一个类作用域,这是较外层的作用域
    enum Types {integer,floating,string}; //这个枚举不会引入新的作用域,但有些会
    Types type; //Type成员位于类作用域中
    union Value { //union引入一个类作用域,这是较内层的作用域
        long long vll;
        long double vld;
        char str[16];
    }; //Value类的定义结束,作用域随之结束
    Value value;
}; //dynamic类的定义结束,作用域随之结束
\end{lstlisting}
这是我们在介绍联合体时讲过的例子。这里我们在 \lstinline@struct dynamic@ 内部定义 \lstinline@union Value@ 的写法,就是一种嵌套定义。\par
定义枚举时,我们可以人为地引入一个\textbf{枚举作用域(Enumeration scope)}——这个取决于我们的需求。有的时候我们希望使用的名字太普遍了,容易出现重名问题,这时我们就可以定义一个\textbf{有作用域枚举(Scoped enumeration)}。
\begin{lstlisting}
    enum class Bluetooth : bool {on,off}; //蓝牙是否打开(枚举基为bool)
    enum class Wifi : bool {on,off}; //Wi=fi是否打开
    enum struct AirplaneMode : bool {on,off}; //用struct或class,效果相同
\end{lstlisting}
以上这些都是有作用域枚举。当需要它们时,我们不能像以往那样直接用 \lstinline@on@/\lstinline@off@ 了——这些 \lstinline@on@ 和 \lstinline@off@ 未必是同一种类型!为了区分,我们需要使用作用域解析运算符 \lstinline@::@ 来操作它们。
\begin{lstlisting}
struct Device{
    //在这里定义Bluetooth等,所以
    Bluetooth bluetooth = Bluetooth::off; //(出厂设置)
    Wifi mobilehotspot = Wifi::on; //这里都要用到作用域解析运算符
    AirplaneMode airplanemode = AirplaneMode::off;
    };
\end{lstlisting}
在老式数据结构的成员中,这样设置默认值是可以的,而且也很简便。但是在类中,我们一般不推荐这么做,而是建议用构造函数来实现数据的初始化。关于这方面的内容,我们会在第八章中介绍。\par
命名空间也会引入一个\textbf{命名空间作用域(Namespace scope)}。命名空间作用域不像函数作用域那样封闭。只要它不是定义在一些封闭的作用域中,我们就可以使用作用域解析运算符来找到它;或者干脆使用 \lstinline@using@ 把它带到全局作用域中来。这点想必不用我再多作解释了。\par
而最经典,也是我们最熟悉的作用域,还是要数\textbf{块作用域(Block scope)}了。一对花括号,它们套住若干语句\footnote{也称为复合语句(Compound statement)。实际上,绝不是所有``由花括号套住''的内容都能算作复合语句,比如统一初始化中所用的语法。},会引入一个块作用域。在这个作用域中定义的名字,无论是变量、类,还是函数\footnote{我们可以在函数内部定义Lambda表达式,这是一种函数对象。我们会在精讲篇中讲到它。},都只对这个作用域可见。\par
\lstinline@for@, \lstinline@while@, \lstinline@if@, \lstinline@else@, \lstinline@switch@ 等结构会自带一个块作用域。我们在 \lstinline@for@ 中定义的内容,在紧随其后的执行部分可见,但在执行部分结束后就不可见了。\par
对于 \lstinline@switch@ 语句来说,我们还有一个注意事项——如果在不同的 \lstinline@case@ 下需要定义局部变量的话,那么直接这样写是会出错的:
\begin{lstlisting}
    switch(num) {
        case 1:
            int x;
            //...
            break;
        case 2:
            int x; //错误
//error: redeclaration of 'int x'
            //...
            break;
    }
\end{lstlisting}
我们试图使用 \lstinline@switch@ 语句来实现分支,而在某些 \lstinline@case@ 下我们可能有定义局变变量的需要——这时候我们就会发现有问题——编译器给我们的报错信息是``再次定义了 \lstinline@x@''!\par
出现这个问题的根本原因在于,\lstinline@case@ 本身只是一个标签,它不会自行引入一个内层作用域。换句话说,如果我们不自行引入一个作用域的话,所有的操作都是在 \lstinline@switch@ 的作用域下进行的,那么出现``再次定义了 \lstinline@x@''这样的报错误息也就不足为怪了!\par
解决这个问题的方法也很简单,就是人为地引入一个块作用域:
\begin{lstlisting}
    switch(num) {
        case 1:{ //人为地用花括号会引入一个更内层的作用域
            int x;
            //...
            break;
        }
        case 2:{ //这样就不会引起冲突了
            int x;
            //...
            break;
        }
    }
\end{lstlisting}\par
\subsection*{名称查找}
C++允许我们在不同的作用域中使用相同的名字。毕竟,我们划分命名空间和作用域的一个主要目的正是避免名称冲突。在互不相干的作用域中自然可以使用同样的名字,比如我们定义一些函数时就经常用 \lstinline@a@, \lstinline@b@ 这样的名字。
\begin{lstlisting}
template<typename T> //这是一个函数模版
T max(T a, T b) { return a > b ? a : b; } //在这个作用域中使用a和b
template<typename T>
T min(T a, T b) { return a < b ? a : b; } //在另一个作用域中也用a和b
\end{lstlisting}
在这里,编译器绝不会搞混这两组 \lstinline@a@ 和 \lstinline@b@。\par
而在互相嵌套的作用域中,我们也可以使用同样的名字。举个例子吧:很多初学者在熟悉了用 \lstinline@i@ 作为单层 \lstinline@for@ 循环的枚举变量之后,就很容易在多重 \lstinline@for@ 循环中不慎把多个变量都用成 \lstinline@i@。
\begin{lstlisting}
    for (int i=0; i < 10; ++i) 
        for (int i=0; i < 10; ++i) {
            //某些操作
        }
\end{lstlisting}
但是这样写并不会被编译器报错,反而可以通过编译,继续运行——但具体的运行结果会如何,那就难说了。\par
其实这是\textbf{名称查找(Name lookup)}的问题。每当我们使用一个名字的时候,编译器都会从此处开始向上查找这个名字的定义。只有找到了这个名字的定义,编译器才能知道这个名字意味着什么(它是变量、还是函数、还是类?什么类型的/接收什么参数/怎么定义的?)。\par
关于名称查找的规则,简单来说就是一句话:编译器会从这个作用域开始,一步步向外找,直到全局作用域为止,只要找到了一个,编译器就会认为这个名称就是这样定义的;如果没找到,就会认为这个名字没有定义,于是报错。\par
那么就以上述的双重 \lstinline@for@ 循环为例,如果我们在``某些操作''中使用了 \lstinline@i@(比如说,输出它的值),这时编译器会如何解读呢?它会向上寻找,当找到内层 \lstinline@for@ 循环中的 \lstinline@int i@ 时,编译器就认定这就是它要找的 \lstinline@i@ 的定义。那么这样一来,``某些操作''中使用的 \lstinline@i@ 全都是内层 \lstinline@for@ 循环中定义的那个 \lstinline@i@;而外层 \lstinline@for@ 揗环中定义的那个 \lstinline@i@ 对它毫无影响。\par
那么接下来请读者做一个练习,以此深化对名称查找的认识,来看这段代码:
\begin{lstlisting}
    int a; //(1)
    int b; //(2)
    {
        int c; //(3)
        {
            int b; //(4)
            int c; //(5)
            cout << c; //(6)
        }
        cout << b; //(7)
        cout << a; //(8)
        int a; //(9)
    }
\end{lstlisting}
这段代码是编译无误的,所以请读者放心。接下来请问读者:(6)(7)(8)三句中所使用的变量 \lstinline@c@, \lstinline@b@, \lstinline@a@ 分别对应哪个定义?\par
先来看(6)。我们在(3)和(5)当中都定义了 \lstinline@c@,但编译器只会选择``它第一个遇到的''定义,所以在寻找过程中当然会先找到(5)了。\par
再来看(7)。我们在(2)和(4)当中都定义了 \lstinline@b@,而且看上去 \lstinline@b@ 离得更近,但是请注意,编译器只会从当前的作用域开始向外找,而当遇到一个内层作用域时就会跳过,所以(4)被自动忽略,编译器将选择(2)。\par
最后来看(8)。(9)显然和(8)在同一个作用域中,而(1)就要远一层了,所以很多人自然会认为是(9)。但是不要忘记,这些代码的执行是有顺序的!我们不可能先输出 \lstinline@a@ 再定义 \lstinline@a@,所以编译器进行名称查找的方向都是向前面的代码去寻找,而不能向后!所以最终的答案是,编译器向前寻找到了(1),然后认定它就是这个 \lstinline@a@ 的定义。\par
这样的名称查找规则也能帮助我们理解``为什么当函数定义写在调用之后时,我们要声明这个函数''。这是因为,编译器在查找这个函数时,只会向前进行查找。所以我们最好把需要定义的那些函数声明在先,这样才能让编译器知道它的存在。\par
\subsection*{对象(变量)的生存期(存储期)}
在前面的章节中,我们谈到,当一个作用域结束时,本作用域中定义的对象\footnote{包括变量、自定义类的对象、枚举项、函数对象等。我们已经了解了其中的部分内容。}的存储期也就终止了。这就是一种自动存储期的例子。而有些对象,比如 \lstinline@std@ 命名空间中的 \lstinline@cin@, \lstinline@cout@ 和 \lstinline@endl@ 等,不仅作用域很大,而且存储期很长,不会因为离开哪一个作用域就终止,它们都是静态存储期的。\par
在C++中,所有的对象都具有以下四种\textbf{存储期(Storage duration)}类型之一:
\begin{itemize}
    \item 自动存储期(Automatic storage duration)。这种存储期的对象在定义之时开始分配内存空间\footnote{虽然我们习惯把它叫作静态分配,但是它和下文要提到的``静态存储期''并不是同类概念。},而在作用域结束时就自动释放(不需要手动,这是它与动态存储期的显著区别)。我们定义的局部变量多为自动存储期变量。
    \item 静态存储期(Static storage duration)。静态存储期的对象在整个程序运行过程中都存在,它会一直保留下来,直到程序结束时释放。\footnote{静态存储期变量一般存储于图5.1中所示的数据段或BSS段,而自动存储期变量一般存储于栈段,这也是与它们的存储期有关的。至于相关细节,我们就不多说了。}
    \item 动态存储期(Dynamic storage duration)。动态存储期的对象由动态内存分配运算符(如 \lstinline@new@/\lstinline@new[]@\footnote{\lstinline@new@/\lstinline@delete@ 的确是我们最常用的动态内存分配方式,但它们不是全部。我们还可以使用 \lstinline@malloc@。} 等函数来实现)分配产生,它不受作用域的影响,只有当我们手动回收动态内存空间时,它们才会被销毁;否则可能产生内存泄漏问题。
    \item 线程存储期(Thread storage duration)。这是C++11标准中添加的新内容。本书不打算讲解有关并发的知识,所以就不介绍线程存储期了。
\end{itemize}\par
我们在局部作用域内定义的变量都有自动存储期,除非我们用 \lstinline@static@, \lstinline@extern@ 或 \lstinline@thread_local@ 说明符来指定它的类型。其中 \lstinline@thread_local@ 属于并发相关的知识,本书不介绍;我们稍后就来讲 \lstinline@static@ 和 \lstinline@extern@。\par
而在全局作用域(或者是隶属于全局作用域下的命名空间,如 \lstinline@std@)中定义的对象,都具有静态存储期\footnote{还可能是线程存储期,但是本书不再深究了。}。\lstinline@std::cin@ 就是一个静态存储期变量,所以我们可以在 \lstinline@input_clear@ 函数中先清理 \lstinline@std::cin@ ,然后再在主函数中继续使用它,没有任何问题!\par
至于动态存储期的对象,我们只能通过动态内存分配来开始,也只有通过动态内存回收才能终止,所以它的存储期也是在运行时动态改变的,未必能事先预知。
\begin{lstlisting}
    if (/*...*/) { //谁能预言这个条件是true还是false呢
        delete[] arr; //回收,此时arr指向的动态内存存储期结束
        arr = nullptr; //置空,防止野指针出现
    }
\end{lstlisting}
\subsection*{对象(变量)的链接方式}
在多文件编译过程中,每个源文件都是一个\textbf{翻译单元(Translation Unit)}。在每个翻译单元中,我们定义了同名的变量,它们是否是``同一个东西''呢?这就涉及到\textbf{链接(Link)}的问题。\par
链接这个概念解决的是变量重名的问题。对于两个重名变量,编译器会根据链接方式来判断它们是否是同一个东西。C++中的变量有三种链接方式\footnote{C++20起增加了``模块链接''方式,但本书基于的标准是C++17。}:\textbf{无链接(No linkage)}、\textbf{内部链接(Internal linkage)}和\textbf{外部链接(External linkage)}。我们分别简单介绍一下这三者。\par
无链接的含义是,只要离了这个作用域(或者是重新进入这个作用域),这个名字对应的变量就会被销毁。
\begin{lstlisting}
    for (int i = 0; i < 5; i++) {
        int x {0};
        ++x;
        cout << x << ' ';
    } //将输出1 1 1 1 1
\end{lstlisting}
这个很好理解,因为每次循环结束,\lstinline@x@ 的生存期都会终止,于是下一次我们定义的 \lstinline@x@ 还是从 \lstinline@0@ 开始自增一次,进而得到输出 \lstinline@1@。也即,在每次循环中出现的 \lstinline@x@ 彼此之间没有关系。\par
但是当我把它改成内部链接时,情况就不一样了。
\begin{lstlisting}
    for (int i = 0; i < 5; i++) {
        static int x {0}; //在这里加了一个static说明符
        x++;
        cout << x << ' ';
    } //将输出1 2 3 4 5
\end{lstlisting}
我们在这里用到了 \lstinline@static@ 说明符,它起到了两个作用:其一,把 \lstinline@x@ 变成了静态存储期对象;其二,规定了 \lstinline@x@ 是内部链接的——在每次循环中出现的 \lstinline@x@ 彼此是同一个事物!\par
那么既然每次循环时的 \lstinline@x@ 都是同一个事物,它就应该从 \lstinline@0@ 一路自增到 \lstinline@5@,而不是在每轮循环中都从 \lstinline@0@ 自增到 \lstinline@1@ 然后就销毁了。\par
在循环体中是如此,在函数和类中也是如此。
\begin{lstlisting}
void func() {
    static unsigned count {0}; //定义一个静态变量count,并初始化为0
    ++count; //count自增
    //...
}
\end{lstlisting}
在这个函数中,我们也把 \lstinline@count@ 定义成一个静态存储期对象。这种静态存储期对象的定义语句只会被执行一次;而在此之后,程序每次运行到这里时,都会直接跳过 \lstinline@count@ 的定义。\par
那么这个 \lstinline@count@ 可以表示什么呢?试想,每次这个函数调用时,程序都会执行 \lstinline@++count@ 这步(前面的定义被跳过),那么 \lstinline@count@ 的值当然就可以用来表示 \lstinline@func@ 函数的执行次数!\par
但是也请读者注意:用 \lstinline@static@ 改变 \lstinline@x@ 和 \lstinline@count@ 的存储期及链接,并不会改变它的作用域。我们不能在上文的 \lstinline@for@ 循环之外使用 \lstinline@x@ 这个名字,也不能在 \lstinline@func@ 函数之外使用 \lstinline@count@。这方面的特例只有静态成员,如果我们定义一个公有静态成员,那么外界就可以获取它的值。
对于类也是如此,我们可以在一个类中声明静态成员:
\begin{lstlisting}
class A
    public:
        static int count; //声明一个静态成员count
    private:
        static int num;
};
\end{lstlisting}
这里我们也使用 \lstinline@count@ 这个名字,但它与 \lstinline@func@ 属于两个不同的作用域,所以不会产生任何冲突。\par
静态成员,无论它的访问权限是什么,都只能在类中声明;静态成员的定义需要写在类定义的外部。
\begin{lstlisting}
int A::count {0}; //必须在外部定义
int A::num {}; //即便是私有成员
\end{lstlisting}
在使用静态成员的时候,我们有两种写法。可以用类名搭配作用域解析运算符来写,即 \lstinline@A::count@;或者用对象名搭配成员访问运算符来写,即 \lstinline@a.count@。第一种写法是静态成员所独有的,这是因为,静态成员有单独的存储空间,对于所有对象来说都相同。想想,这不正是内部链接的特点吗——每个对象的 \lstinline@count@ 成员都是同一个事物!\par
内部链接和外部链接的区别是,内部链接仅仅说明``在本翻译单元内它们是同样的事物'',那么对于不同翻译单元来说呢?别忘了,我们在第一章中讲过跨文件编译,这里起码有两个源文件,也就是起码两个翻译单元\footnote{头文件只是一种代码复用的形式,它并不会影响什么链接作用。我们完全可以不用头文件,直接用多个源文件来搞定那些功能——只是写起来要有很多重复代码,修改也比较麻烦。},它们之间是什么关系呢?\par
我们在前文中讲过,我们可以把函数的声明写在某个源文件中,而把定义写在另一个源文件中——它们二者就是``同一个事物''。这正是一种外部链接,不同翻译单元中的同名(且同参数列表)函数对应的是同样的事物。\par
在全局作用域中声明(定义)的对象是默认为外部链接的,我们也可以使用 \lstinline@extern@ 说明符。
\begin{lstlisting}
//1.cpp
extern int a; //这里的extern不可省略
//2.cpp
extern int a {3}; //这里的extern可省略
\end{lstlisting}
标记为 \lstinline@extern@ 的对象都是静态存储期对象;但它与 \lstinline@static@ 规定的内部链接不同,\lstinline@extern@ 对象是外部链接的。\par
\textbf{单一定义规则}要求我们,任何一个类/函数/对象都只能有一个定义。既然这里的 \lstinline@a@ 是外部链接的,那么它们当然是同一个事物,也就只能有一个定义。因而,\lstinline@2.cpp@ 中的是定义,而其它源文件中不应该再有定义。但是,\lstinline@extern@ 也不会改变作用域,我们要想在 \lstinline@1.cpp@ 中使用 \lstinline@a@ 这个变量,就必须要声明。于是我们就要写 \lstinline@extern int a@ 了。\par
如果我们希望一个全局对象是内部链接的,那么我们可以在定义语句中使用 \lstinline@static@ 说明符。
\begin{lstlisting}
//1.cpp
static int a {-1}; //内部链接
//2.cpp
static int a {1}; //这两个a是不同的事物
\end{lstlisting}
在这里,不同源文件中的 \lstinline@a@ 各自具有内部链接;但它们没有外部链接,也就意味着它们是两个不同的事物。(读者可以自己尝试设计一个方案,来分别输出这两个 \lstinline@a@ 的地址,作为挑战题)所以我们给这两个 \lstinline@a@ 分别进行定义,这是不违反单一定义规则的。\par
