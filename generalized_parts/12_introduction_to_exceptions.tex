\chapter{异常处理简介}
我们在写代码的时候常常会不可避免地犯错误。比如说,一个不小心把 \lstinline@==@ 写成了 \lstinline@=@,像这样:
\begin{lstlisting}
    if (n * m = c) { /*...*/ }
\end{lstlisting}
不过所幸,编译器能够一下子发现错误,进而提醒我们这里出错了。\par
但不是所有的bug都可以在编译期发现,比如:
\begin{itemize}
    \item 动态内存空间不足,无法分配新的内存空间。
    \item 数组访问越界。比如,循环结构 \lstinline@\for(int i=n-1;i>=0;i++)@ 中本来是要写成自减运算符的,但是没注意就不小心写了个自增。
    \item 为函数/运算符提供了不当的参数。比如,除数为 \lstinline@0@。
    \item ……
\end{itemize}
这些问题的发生将导向不可预测的结果,比如程序崩溃。\par
但是编译器可查不出这些问题是否存在,因为它们不是在编译期就能够知道的信息。所以我们需要有一套机制,来对这些运行过程中可能产生的\textbf{异常(Exception)}进行捕获和处理。这也就是本章的核心内容。\par
``异常''这个词的概念十分宽泛,没有确定的标准。你可以说,任何非一般情况都是异常;你也可以说,只有能引起程序错误的操作是异常。这种没有定规的情形给了我们很大的自由度——我们可以随意地安排异常处理的场合及手段;不过这样也会让初学者盲目使用,乃至滥用。\par
而关于异常处理的知识,说多不多,说少也不少。限于篇幅,在泛讲篇中我们就介绍一下最基础的部分。至于更深入的内容,让我们留到精讲篇吧。\par
\import{../generalized_parts/12_introduction_to_exceptions/}{01_try_catch_throw.tex}
\import{../generalized_parts/12_introduction_to_exceptions/}{02_std_exception.tex}
\import{../generalized_parts/12_introduction_to_exceptions/}{03_exercise_matrix.tex}