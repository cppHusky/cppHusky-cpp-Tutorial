\section{默认参数的设置}
在数学上,对数函数$\log_aN$是一个二元函数,其参数分别为$a$和$N$。不过在数学当中,我们一般都用$a=\mathrm{e}$作为对数的底。因此为了方便,我们就约定俗成,不写底数的时候默认以$\mathrm{e}$为底,写成$\log{N}$(一些数学文献中也用$\ln{N}$代替底数为$\mathrm{e}$时的对数函数)。\par
$n$次方根函数$\sqrt[n]{a}$也是一个二元函数,其参数分别为$n$和$a$。我们最常用的是二次方根,所以当$n=2$时,我们习惯省略它,直接写作$\sqrt{a}$。\par
以上是在数学中常见的情况。而在C++语言中,我们也可能遇到类似的问题——某个函数的某个参数在多数时候都是某个特定值,我们能否想一种方法,省略掉一些常常为特定值的参数,这样我们就不必在每次调用此函数时都写这个参数了呢?\par
答案是肯定的。例如我们要写一个开 \lstinline@n@ 次方根函数,我们就可以用上一节提到的重载方法,写一个接收 \lstinline@a@ 和 \lstinline@n@ 的 \lstinline@root@ 函数,再写一个只接收 \lstinline@a@ 的 \lstinline@root@ 函数(将 \lstinline@n@ 当作 \lstinline@2@)。鉴于 \lstinline@root@ 函数本身的实现比较麻烦,我们退而求其次,使用 \lstinline@pow(a,1./n)@\footnote{\lstinline@double pow(double base,double exp)@ 是 \lstinline@cmath@ 库中的一个函数,返回值为 \lstinline@base@ 的 \lstinline@exp@ 次幂。这个函数还重载了其它浮点类型的版本,但我们在这里不需要它们。}来实现它。\par
\begin{lstlisting}
#include <iostream>
#include <cmath> //pow定义于此
using namespace std;
double root(double, int);
double root(double);
int main() {
    cout << root(64, 3) << endl; //将调用root(double,int)
    cout << root(64) << endl; //将调用root(double)
    return 0;
}
double root(double radicand, int index) {
    return pow(radicand, 1. / index); //注意index是int型,所以要用1./index
}
double root(double radicand) {
    return pow(radicand, .5); //默认开平方根,就可以直接用0.5次幂表示
}
\end{lstlisting}
关于函数重载,读者已经比较熟悉了,所以这里我就不再对代码作过多解释。这里的 \lstinline@root(double)@ 也可以定义成下面这个样子,效果相同:
\begin{lstlisting}
double root(double radicand) {
    return root(radicand, 2); //利用root(double,int)的返回值
}
\end{lstlisting}\par
除了重载之外,我们还可以使用另一种方法:为函数设置\textbf{默认参数(Default argument)}。\par
当我们在声明或定义一个函数的时候,我们可以为它设置一些默认值。这样,当我们调用函数的时候,如果把对应的参数留空,编译器就会用默认值来替代它。以下是一个例子:
\begin{lstlisting}
double root(double, int = {2}); //等号不能省略,花括号并非必需,但建议使用
int main() {
    cout << root(64, 3) << endl; //将调用root(double,int),第二个参数传入3
    cout << root(64) << endl; //将调用root(double,int),第二个参数取默认值2
    return 0;
}
double root(double radicand, int index) {
    return pow(radicand, 1. / index); //注意index是int型,所以要用1./index
}
\end{lstlisting}
看起来函数重载和默认参数这两种方法都可以达到我们的目的,那么它们的区别是什么呢?\par
其实这两种方法最大的区别在于,函数重载是``定义了两个函数'',而设置默认参数``不会定义新函数''!在函数重载的情况下,我们计算 \lstinline@root(64,3)@ 时,调用的是 \lstinline@root(double,int)@,而计算 \lstinline@root(64)@ 时,调用的是 \lstinline@root(double)@;但如果我们只是设置了默认参数,那么无论计算 \lstinline@root(64,3)@ 还是 \lstinline@root(64)@,都会调用 \lstinline@root(double,int)@ 这个函数。如果我们只是想设置默认参数而无需做大的细节改动,那么用默认参数的方法当然比重载要方便了(我们无需重新定义一个函数!)。\par
而函数重载应该用在更需要处理细节改动的情况下,比如两个数求最大值和三个数求最大值,函数内部的细节是不太相同的——这时若用默认参数就显得更麻烦了,而不是更简单。\par
在默认参数方面,其实还有很多需要留意的内容。不过大部分问题不常见,我在泛讲篇中就不多说了,只说几个小的要点。\par
其一,默认值最好是在函数声明时就规定出来,而不是等到定义时再规定。这是因为,编译器遇到函数调用的时候,会向前找寻函数声明。只有在函数声明中规定了默认值的情况下,编译器才会考虑匹配;如果在函数声明中没有规定默认值,编译器就找不到这个函数(或者好巧不巧,匹配上了其它同名函数,但那不是我们想要的),即便我们后面的函数定义中再规定默认值也是没用了!\par
其二,有默认值的参数必须在函数参数列表的右侧。这可能有点费解,我举个例子你就懂了。
\begin{lstlisting}
//假设这四个函数定义在不同场合下,互不影响
double root(double, int = {2}); //允许
double root(int = {2}, double); //这是禁止的!
//error: default argument missing for parameter 2 of 'double root(int, double)'
double root(double = {1.}, int = {2}); //允许
double root(int = {2}, double = {1.}); //允许
\end{lstlisting}
编译器的解释方式是,但凡某个参数有默认值,从它开始到后边的所有参数都应该有默认值。因此,我们定义函数时,必须要把有默认值的参数放在参数列表的右侧。\par
我们在先前接触过一个自定义函数 \lstinline@input_clear()@,它可以清理错误输入。我们最常用 \lstinline@istream@ 类的 \lstinline@cin@ 对象来进行输入,但这不代表我们在其它时候不需要用别的对象。为此,我们可以改一下 \lstinline@input_clear@ 函数,让它可以接收任何一个 \lstinline@istream@ 类的对象\footnote{事实上还可以接收 \lstinline@istream@ 派生类的对象,但这就说得远了。}。而鉴于我们常用 \lstinline@cin@,所以不妨把它设为默认参数。\par
\begin{lstlisting}
//不单独声明,直接定义
void input_clear(istream &in = {cin}) { //不提供参数时使用默认参数cin
    in.clear(); //清除错误状态
    while (in.get() != '\n') //清除本行输入
        continue;
}
\end{lstlisting}
这里的形参长得有点不同,它是一个引用,我们会在下一章中讲到。\par
