\chapter{继承中的常见问题}
在上一章中,我们讲解了继承的基本语法和基本概念。本章我们继续深入,以期进一步搞清楚继承概念之下的许多常见问题——
\begin{itemize}
    \item 继承关系中的类型转换是怎么进行的?又有什么规则?\par
    \item 多态又是怎么一回事?虚函数到底在做什么?\par
    \item 如果描述基类所需要的成员比描述派生类的还要多,那怎么办?\par
    \item 多重继承是什么?怎么理解?又要怎么用?\par
    \item 棱形继承关系中,基类的成员重复了怎么办?\par
\end{itemize}
这些问题在实际编程中很常见,所以我们需要好好研究一下,以防将来真的用的时候你突然一拍脑门说:``哎呀,我没学过这玩意。''那就有点麻烦了。\par
\import{10_common_problems_in_inheritance/}{01_type_cast_in_inheritance_relationship.tex}
\import{10_common_problems_in_inheritance/}{02_virtual_function_and_polymorphism.tex}
\import{10_common_problems_in_inheritance/}{03_abstract_base_class_and_pure_virtual_function.tex}
\import{10_common_problems_in_inheritance/}{04_multiple_inheritance.tex}
\import{10_common_problems_in_inheritance/}{05_virtual_inheritance.tex}
