\chapter{数据的基本操作}
在第一章,我们通过一些简单的代码,对C++语言有了最基本的认识。我们还了解了数据、信息、内存和面向对象的相关常识。在后面的章节中,我们将更多关注于具体的知识。\par
本章我们先来介绍数据。按照分类标准的不同,我们可以把C++语言中的数据分成不同的类别:
\begin{itemize}
    \item 如果按照数据的表达方式来分类,数据可以分为整型数据、浮点型数据等等。
    \item 如果按照是否具名来分类,数据可以分为不具名量\footnote{我们先前提及的``字面量''正是不具名量的一种。}和变量。前者没有名字,后者有名字。
    \item 如果按照是否可以改变来分类,数据可以分为常量和可变量。前者不可改变,后者可以改变。
    \item 如果按照是否可取地址来分类,数据可以分为左值和右值。\footnote{有关左值和右值的问题,我们会在精讲篇中讲解。}
    \item \ldots\ldots
\end{itemize}\par
光是分类还不够。我们还要学会如何更好地使用数据。在第一章中我们通过简单的代码实现了计算器的功能,但如果我们仅仅止步于此,那还远远没有达到C++能够支持的水平。在这一章我们会学习更多有关运算符的知识,以此扩展代码的可能性。\par
数据类型的差异可能会为我们造成一些困扰。在实际编程时,我们经常需要在不同数据类型之间作转换。本章我们也会谈及数据类型转换的基础语法。\par
C++的世界非常丰富,但一切都要从数据开始。\par
\import{02_basic_operation_on_data/}{01_assignment_and_constants.tex}
\import{02_basic_operation_on_data/}{02_fundamental_types.tex}
\import{02_basic_operation_on_data/}{03_operators.tex}
\import{02_basic_operation_on_data/}{04_type_cast.tex}
