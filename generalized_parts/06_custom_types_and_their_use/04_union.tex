\section{联合体}
联合体初看上去很像结构体,它们的定义语法极其相似,但是它们是完全不同的事物。
\begin{lstlisting}
union ValType {
    int val_i;
    long long val_ll;
    char val_c;
    double val_d;
    long double val_ld;
};
int main() {
    cout << sizeof(ValType); //输出8或16,总之不是29或者更大的数
}
\end{lstlisting}
一个结构体会把所有成员分别存储在不同的内存区域中,所以这个类型的内存占用不能小于所有成员的总和。以第2节中定义的 \lstinline@Cuboid@ 类型为例,我们可以用这样一段代码来观察某个对象各成员的地址:
\begin{lstlisting}
    Cuboid cub;
    cout << &cub.length << endl << &cub.width << endl << &cub.height;
\end{lstlisting}
下面是一种可能的运行结果:\\\noindent\rule{\linewidth}{.2pt}\texttt{
0x7ffeea18b070\\
0x7ffeea18b074\\
0x7ffeea18b078
}\\\noindent\rule{\linewidth}{.2pt}\par
从结果中我们可以看到,\lstinline@cub@ 的三个 \lstinline@int@ 型成员分别在不同的内存空间中,\lstinline@cub.length@ 在 \lstinline@...070@ 到 \lstinline@...073@ 位置,\lstinline@cub.width@ 在 \lstinline@...074@ 到 \lstinline@...077@ 位置,\lstinline@cub.height@ 在 \lstinline@...078@ 到 \lstinline@...07b@ 位置。它们之间没有任何重叠,互不干扰。\par
但联合体不然,它会把所有成员存储到同一个内存区域中。
\begin{lstlisting}
    ValType a;
    cout << &a.val_i << endl
        << &a.val_ll << endl
        << (void*)&a.val_c << endl //记得把char*转成void*输出
        << &a.val_d << endl
        << &a.val_ld << endl;
\end{lstlisting}
下面是一种可能的运行结果:\\\noindent\rule{\linewidth}{.2pt}\texttt{
0x7ffc00a4d280\\
0x7ffc00a4d280\\
0x7ffc00a4d280\\
0x7ffc00a4d280\\
0x7ffc00a4d280
}\\\noindent\rule{\linewidth}{.2pt}
这说明联合体的所有成员都在同一个位置存着呢。