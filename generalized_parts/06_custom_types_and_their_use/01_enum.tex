\section{枚举常量\texttt{enum}}
定义一个枚举常量的基本格式如下:\footnote{本节只介绍无作用域枚举,关于有作用域枚举,请见精讲篇。}
\begin{lstlisting}
enum <枚举名> {<枚举项>=<整型常量表达式>, <枚举项>=<整型常量表达式>, ...};
\end{lstlisting}\par
枚举名不是必须的。如果不使用枚举名,那么这些枚举项的类型就是匿名枚举,这样做相当于定义了一些整型常量。如果我们规定了名字,那它们就是具名枚举,有丰富的作用。\par
枚举常量都是基于整型的,它的枚举项可以隐式类型转换为整型来输出。
\begin{lstlisting}
    enum {A=1, B=3, C=5}; //不具名枚举
    cout << B; //输出3
\end{lstlisting}
我们可以省略其中的部分枚举项的值,这时枚举项的值是基于前一个值递增的。
\begin{lstlisting}
    enum {A=-2, B=4, C, D}; //C为递增为5,D递增为6
    enum {E, F, G}; //第一项E不设值则为0,后面F,G分别为1和2
\end{lstlisting}
如果我们需要一次性定义大量整型常量,那么除了 \lstinline@const@ 及 \lstinline@constexpr@ 语法之外,我们还可以选择 \lstinline@enum@。\par
如果这个枚举有枚举名,那么我们还可以把它作为一个类型来用!具名枚举类型的优点是:
\begin{itemize}
    \item 它的值可以用标识符来表示,而不是用单纯的字面量——我们提过,常量表达式比字面量的一大好处在于方便在源代码层面做统一修改。
    \item 具名枚举类型可以隐式转换为整型常量,所以我们使用起来很方便。
    \item 整型常量不能隐式转换为具名枚举类型,而且具名枚举类型数据的取值范围有很大限制,我们可以很防止这类数据取一些不允许的值。
\end{itemize}
举个例子,想必读者就可以理解了。关于服装的尺码,不同国家/地区都有各自的规定。以男士西装为例,欧洲/俄罗斯、英国/美国、日本、韩国都有自己的尺寸对照表。我们可以用枚举常量来表示各个尺寸。
\begin{lstlisting}
enum SuitsSize {
    XXS=40, XS=42, S_1=44, S_2=46, M_1=48, M_2=50,
    L_1=52, L_2=54, XL=56, XXL=58, XXXL=60
}; //这是欧洲/俄罗斯标准
\end{lstlisting}
接下来我们可以定义一些 \lstinline@SuitsSize@ 类型的变量,比如说就定义成对应的人名吧。
\begin{lstlisting}
    SuitsSize Erik {XXL}; //Erik的值就是XXL,不是58,只是它可以隐式转换成58
    SuitsSize Jacques {XS}; //Jacques的值就是XS
    SuitsSize Carlos {M_1}; //M尺码有两个,为了区分而分别写成M_1和M_2
\end{lstlisting}\par
当我们想要知道枚举项的具体值时,我们可以直接以 \lstinline@int@ 类型输出。
\begin{lstlisting}
    cout << static_cast<int>(XL); //或者直接cout<<XL,会隐式类型转换
\end{lstlisting}\par
如果某个人的尺码发生了改变,我们可以直接赋值来改变,非常方便。
\begin{lstlisting}
    Erik = XXXL; //胖了
\end{lstlisting}
但是我们赋值也不能乱来;我们只能赋那些已经规定好了的项。
\begin{lstlisting}
    Eirk = XXXXXXXL; //错误!
\end{lstlisting}
既然 \lstinline@SuitsSize@ 也是一个类型,那么我们当然还可以定义这个类型的常量数据。
\begin{lstlisting}
    const SuitsSize Giovanni {XXXL}; //Giovanni的值一经定义,就不可改变
\end{lstlisting}\par
总而言之,我们可以把具名枚举的各项看作是整型数据,但是我们应该先把它当成一种特殊的自定义类型,它的值就是标识符本身,只不过可以转换为整型而已。作为类比,我们可以想一下 \lstinline@bool@ 类型,它的值就是标识符 \lstinline@true@ 或 \lstinline@false@ 本身,只是它可以隐式转换成整型而已。\par
每个枚举项的内存占用默认与 \lstinline@int@ 相同,这种做法有些时候有点浪费空间。比如说,\lstinline@SuitsSize@ 的所有枚举项的值都在 \lstinline@40@ 到 \lstinline@60@ 之间,用一个字节足够存储得下了。为了节约内存空间,我们可以改变它的枚举基,也就是它们基于哪种类型。
\begin{lstlisting}
enum SuitsSize : char {//...}; //省略
\end{lstlisting}
这样,\lstinline@SuitsSize@ 类型的内存占用就变成了一个字节,读者可以用 \lstinline@sizeof@ 检验之。\par
除了 \lstinline@char@ 之外,我们还可以用 \lstinline@bool@ 以及所有的整型类型。例如要表示一个人的性别,我们可以用一个 \lstinline@bool@ 型数据来实现。我们可以记 \lstinline@true@ 表示男性,\lstinline@false@ 表示女性。不过用 \lstinline@true@/\lstinline@false@ 的表示方法不够直接,我们何不自定义一个具名枚举呢?
\begin{lstlisting}
enum Sex : bool {male, female}; //以bool为枚举基
\end{lstlisting}
在这里,我们当然也可以指定 \lstinline@male@/\lstinline@female@ 的布尔值;但是这样做没太大的意义,因为我们真正要表示的信息其实是性别而不是那个布尔值。在应用的时候,我们可以直接这么用:
\begin{lstlisting}
    Sex group[5] {male, male, female, male, female}; //定义
    for(auto individual : group) { //范围for循环
        if(individual == male) { //用individual==male 作为判断条件
            //...
        }
        else {
            //...
        }
    }
\end{lstlisting}
