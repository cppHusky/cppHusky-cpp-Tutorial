\chapter{初识C++}
在本章,我们将会了解一些C++中最基本的知识,最常见的概念,和最简单的语法,并付诸代码来实现。\par
我们不要求读者搞懂每句代码的含义——那是不必要的。我们在本章中会给出一个基本的代码模板,读者只需修改其中的一小部分即可。毕竟,\textbf{学习都是从模仿开始的},C++也不例外。\par
有一点值得注意。我们需要有一个编译器,来把我们的代码(.cpp文件或者别的什么)变成可以执行和看到效果的程序(Windows中的.exe文件、Linux或MacOS上的可执行文件,或者别的什么),以便检查它能否实现我们想要的功能,并在日后使用这个程序。\par
根据自己计算机的系统,你可以选择\href{https://www.embarcadero.com/free-tools/dev-cpp}{Embarcadero Dev-C++ (Windows)}, \href{https://learn.microsoft.com/zh-cn/cpp/?view=msvc-170}{Microsoft Visual C++ (Windows)}, \href{https://developer.apple.com/xcode/}{Apple Xcode (MacOS)}, \href{https://github.com/open-watcom/open-watcom-v2}{Open Watcom C++ (Windows/Linux/DOS)}, \href{https://www.codeblocks.org/}{Code::Blocks (跨系统)}, \href{https://www.monodevelop.com/}{MonoDevelop (跨系统)}。这些都是集成开发环境\footnote{集成开发环境(Integrated Development Environment, IDE),是一种辅助进行软件开发的应用。在开发工具内部就可以编写源代码,并编译打包成为可用的程序。IDE需要的配置较少,对新手比较友好,上手容易。},你只需要在下载后做很简单的配置,就可以开始写代码了。\par
另外还有一些代码编辑器,比如\href{https://code.visualstudio.com/}{Visual Studio Code (Windows/Linux/MacOS)}, \href{https://www.sublimetext.com/}{Sublime Text (Windows/Linux/MacOS)}, \href{https://www.vim.org/}{Vim (跨系统)},你需要做额外的配置(比如安装插件),把它们变得如同集成开发环境一样。\par
另有一些在线的代码编译工具,让你可以无需安装软件,只要在浏览器上提供代码,就能远程编译并告诉你运行结果。比如\href{https://coliru.stacked-crooked.com/}{Coliru}和\href{https://wandbox.org/}{Wandbox}。不过相比于本地编译运行,这种方式的交互效果就相当差劲了,仅适合一些无输入或只有简单输入的程序。这些在线编译方式对于本章涉及的程序来说当然够用,但不要太过依赖。\par
\import{01_welcome_to_cpp/}{01_start_with_a_cpp_program.tex}
\import{01_welcome_to_cpp/}{02_data_and_information.tex}
\import{01_welcome_to_cpp/}{03_define_and_use_data.tex}
\import{01_welcome_to_cpp/}{04_operators_and_types.tex}
\import{01_welcome_to_cpp/}{05_sizeof_and_memory.tex}
\import{01_welcome_to_cpp/}{06_classes_and_objects.tex}
