\chapter{代码工程}
当读者已经读到这里的时候,以你的知识,其实已经足够实现很多功能了。将来我们的代码会写得越来越长,花样和技巧也会越来越多,功能的多样性和复杂程度也会逐渐攀升。\par
在开始下一步的内容之前,我深感有必要向读者传授一些基本的工程知识,这样我们才能够更加清晰、简洁、有效地管理我们的代码,提高生产力,并且不致在许多恼人问题上浪费时间。\par
在本章之后,我也会逐步调整示例代码的风格,从新手友好型的 \lstinline@using namespace std@ 逐渐转向通篇 \lstinline@std::@ 的严谨性风格上去。读者也应当学会适应不同风格的代码,并最好形成自己最习惯的一套风格来。\par
希望读者能够在学完本章之后掌握一些代码工程的相关知识,这对我们后续的编程大有裨益。\par
\import{07_projecting/}{01_separate_compilation.tex}
\import{07_projecting/}{02_scope_and_storage_duration.tex}
\import{07_projecting/}{03_namespace.tex}
\import{07_projecting/}{04_coding_style.tex}