\chapter{函数初步}
在前面的章节中我们讲过了数据、运算符、语法/语义、语序,还有流程控制。从细枝末节到整体结构,我们可以一窥编程大厦的样貌。笔者犹记得,自己当时只学了结构控制,就拿着C++来写命令行界面小游戏了——其实不难,无非就是人机战斗,用户给个输入,用选择结构判断该做什么,由程序来算几个四则运算,再给出输入。最后外面套个 \lstinline@while(true)@ 循环,一个自制小游戏就这么做完啦。\par
仅有我们此前所学的知识已经足够做出一个完整的、有交互功能的程序,只是工作量会大一点而已。但是别高兴太早,当你真正去实践编程的时候就会发现,要真正实现复杂功能,还是需要极高的工作量,并且其中的许多工作是无意义重复的。在我的游戏当中,我选择用Ctrl+C/V来解决这个问题,但这还是很麻烦。而且,如果我发现这部分的内容需要调整,我就要把所有相关代码全改一遍!\par
本章我们先来解决这些重复代码的问题,我们的选择是\textbf{函数(Function)}。\par
\import{04_introduction_to_functions/}{01_the_concept_of_functions.tex}
\import{04_introduction_to_functions/}{02_the_definition_and_use_of_functions.tex}
\import{04_introduction_to_functions/}{03_function_overloading.tex}
\import{04_introduction_to_functions/}{04_default_argument.tex}
\import{04_introduction_to_functions/}{05_recursion.tex}
\import{04_introduction_to_functions/}{06_introduction_to_function_templates.tex}