\section{公开继承与受保护成员}
\subsection*{基本语法}
继承的基本语法就是在定义\footnote{虽然在声明类时偶尔也可以这样做,但是我不推荐如此,因为在进行继承时编译器必须已经知道对应基类的定义。}一个类时,用冒号加``继承方式''和``基类名''的形式来表明它按何种方式继承自哪个类。
\begin{lstlisting}
class Base {
    //...
}; //Base类定义已毕
class Derived : public Base { //按public方式继承自Base类
    //...
}; //Derived公开继承自Base
\end{lstlisting}\par
其中的继承方式分为 \lstinline@public@, \lstinline@protected@ 和 \lstinline@private@ 三种,在本节中我们先讲 \lstinline@public@,即公有继承。\par
\subsection*{继承方式与访问权限}
我们已经很熟悉``访问权限''这个概念了。访问权限关乎一个成员是否对外界可见:
\begin{itemize}
    \item \lstinline@public@ 成员对外完全可见。
    \item \lstinline@private@ 成员对外完全不可见。
    \item \lstinline@protected@ 成员比较特殊,我们稍后讲到。这些成员对外界不可见,但对该类的派生类可见。
\end{itemize}
我们说,一个派生类对象具有基类的成员,比如 \lstinline@Husky@ 就有 \lstinline@age@ 和 \lstinline@weight@。那么在派生类中,这些基类的成员又是何种访问权限呢?这是由``基类成员访问权限''和``继承方式''二者共同决定的,见表9.1。\par
\begin{table}[htbp]
\centering
\begin{tabular}{cccccc}
\hline
\rule{0pt}{2.4ex}
\multirow{2}{*}{继承方式} & 基类成员权限 & \multirow{2}{*}{\lstinline@public@} & \multirow{2}{*}{\lstinline@protected@} & \multirow{2}{*}{\lstinline@private@} \\
\cline{2-2}
\rule{0pt}{2.4ex}
& 派生类成员权限 & & & \\
\hline
\hline
\rule{0pt}{2.4ex}
\lstinline@public@ & & \lstinline@public@ & \lstinline@protected@ & 不可见\\
\hline
\rule{0pt}{2.4ex}
\lstinline@protected@ & & \lstinline@protected@ & \lstinline@protected@ & 不可见\\
\hline
\rule{0pt}{2.4ex}
\lstinline@private@ & & \lstinline@private@ & \lstinline@private@ & 不可见\\
\hline
\end{tabular}
\caption{派生类的成员访问权限取决于基类成员访问权限和继承方式}
\end{table}
其中的 \lstinline@protected@ 成员(受保护成员)访问权限对于读者来说是个全新的概念,下面我就来解释一下它的作用和优缺点。\par
\subsection*{\texttt{protected}成员}
我们谈过,\lstinline@private@ 成员的优点在于它是封闭的,只对类内可见。但对于继承来说,它也可能是个麻烦。\par
举个例子,\lstinline@Husky@ 类中的 \lstinline@destory@ 成员函数需要用到 \lstinline@weight@ 这个成员变量。如果我们像上一节中那样在 \lstinline@Dog@ 类中把 \lstinline@weight@ 定义为 \lstinline@public@\footnote{我们此前提过,\lstinline@struct@ 成员的默认访问权限均为 \lstinline@public@;\lstinline@struct@ 类的默认继承方式也是 \lstinline@public@。所以 \lstinline@Dog@ 类中的成员都是公有成员,而 \lstinline@Husky@ 类以公开方式继承 \lstinline@Dog@ 类。},那么无论 \lstinline@Husky@ 类还是其它无关的外界类/函数都有了修改 \lstinline@weight@ 的权限,这是很危险的;然而,如果我们把 \lstinline@weight@ 定义成 \lstinline@private@,那么无论 \lstinline@Husky@ 还是其它外界类/函数都不能访问 \lstinline@weight@\footnote{一种观点认为,\lstinline@private@ 成员是``不可继承''的,换句话说,基类的私有成员并不是派生类的成员。这种看法是合理的,因为一个类的成员总该对这个类可见;如果对这个类都不可见,那么它也算不上是这个类的成员。不过笔者在这里顾及理解上方便,还是选择这样讲。}。\par
而 \lstinline@protected@ 则能很好地解决这个问题。基类的 \lstinline@protected@ 成员对于外界来说是不可见的,但它对于这个类的派生类来说则是可见的,如图9.2所示。\par
\begin{figure}[htbp]
    \centering
    \includegraphics[width=\textwidth]{../images/generalized_parts/09_protected_members.png}
    \caption{\lstinline@protected@ 成员对外的可见性}
\end{figure}
有了 \lstinline@protected@ 之后,我们可以这样写:
\begin{lstlisting}
class Dog {
protected: //受保护成员,对外不可见,但对Dog的派生类可见
    unsigned _age; //所有狗都有年龄属性
    double _weight; //所有狗都有体重属性
};
class Husky : public Dog { //公有方式继承自Dog
    //这个类拥有成员age和weight,成员权限为protected
public: //公有成员,对外界可见
    void destroy() { /*...*/ }; //哈士奇独特的拆家本领
};
\end{lstlisting}\par
这样一来,\lstinline@_age@ 和 \lstinline@_weight@ 就对 \lstinline@Husky@ 类可见,使 \lstinline@Husky@ 类的函数可以方便地访问这些成员;同时它们又对无关的其它类不可见,保证了成员不受篡改的安全性。\par
\subsection*{构造与初始化}
派生类拥有它基类的成员,这个关系很像是,一个派生类对象当中内嵌了一个基类对象。
\begin{figure}[htbp]
    \centering
    \includegraphics[width=.8\textwidth]{images/generalized_parts/09_built_in_base_class_object_in_derived_class.png}
    \caption{一个派生类对象当中内嵌了一个基类对象}
\end{figure}
所以当我们需要对基类对象进行初始化时,也不是按照``成员''进行初始化,而是把基类的对象当作一个整体,调用基类的构造函数进行初始化。\par
\begin{lstlisting}
class Dog {
protected: //受保护成员,对外不可见,但对Dog的派生类可见
    Dog(unsigned age, double weight) : _age {age}, _weight {weight} {}
    //如果把Dog的构造函数定义成protected权限,那么只有其派生类才能正常调用它
    unsigned _age; //所有狗都有年龄属性
    double _weight; //所有狗都有体重属性
};
class Husky : public Dog { //哈士奇类,公有方式继承自Dog
    //这个类拥有成员age和weight,成员权限为protected
private:
    int _destroy_ability; //拆家能力
public:
    Husky(unsigned age, double weight, int ability)
        : Dog {age, weight}, _destroy_ability {ability}
    {} //初始化内嵌Dog对象的成员,以及_destroy_ability成员
    void destroy() { /*...*/ }; //哈士奇独特的拆家本领
};
class Retriever : public Dog { //金毛寻回犬类,公有方式继承自Dog
    //这个类拥有成员age和weight,成员权限为protected
private:
    int _guide_ability; //导盲能力
public:
    Retriever(unsigned age, double weight, int ability)
        : Dog {age, weight}, _guide_ability {ability}
    {} //初始化内嵌Dog对象的成员,以及_guide_ability成员
    void guide() { /*...*/ } //金毛独特的导盲本领
};
\end{lstlisting}\par
在这里,我们把 \lstinline@Dog@ 类的构造函数定义成受保护成员。按前文所述,这些成员对派生类以外的外界都是不可见的,所以我们不能在外部使用这个函数来进行构造\footnote{其实这个类还有一个作为公有成员的默认拷贝构造函数和默认移动构造函数,所以确实还存在这样的方法使我们能够在外部定义对象。}。换言之,我们这样就保证了:在外界中不能直接定义 \lstinline@Dog@ 类对象,只能定义它的派生类的对象\footnote{不过,从另一个意义上讲,派生类的对象也是基类的对象。}。\par
两个派生类 \lstinline@Husky@ 和 \lstinline@Retriever@ 对象的构造函数都是公有的,我们就可以在类外调用了。它们都接收三个参数,分别是 \lstinline@age@, \lstinline@weight@, \lstinline@ability@。所以我们可以这样定义对象:
\begin{lstlisting}
int main() {
    Husky mine {5, 27, 1000000};
    Retriever zhang3 {4, 25.67, 99};
}
\end{lstlisting}
在创建 \lstinline@mine@ 对象时,程序将调用 \lstinline@Husky::Husky(unsigned,double,int)@ 函数,那么这个函数做了什么呢?
\begin{lstlisting}
public:
    Husky(unsigned age, double weight, int ability)
        : Dog {age, weight}, _destroy_ability {ability}
    {}
\end{lstlisting}\par
我们看到,这个函数在初始化时,其 \lstinline@Dog@ 成员部分是通过调用 \lstinline@Dog@ 类的构造函数来集中解决的;而独属于 \lstinline@Husky@ 的那部分,才是单独处理的。这一点对三种继承方式来说都是通用的——来自基类的成员,需要通过基类的构造函数来进行初始化。即便我们在代码中没有这么写,基类的构造函数也是会被调用的。我们可以用这段代码来验证之:
\begin{lstlisting}
struct Base { //struct成员默认为public成员,这里图方便就用struct了
    Base() { //Base的默认构造函数
        std::cout << "Base::Base() is called." << std::endl;
    }
};
struct Derived : Base { //struct类的默认继承方式也是公开继承
    Derived() { //在Derived构造函数当中并没有写明要调用Base的构造函数
        std::cout << "Derived::Derived() is called." << std::endl;
    }
};
int main() {
    Derived de; //定义Derived类的对象,预期会调用Derived类的构造函数
}
\end{lstlisting}
这段代码的运行结果如下:\\\noindent\rule{\linewidth}{.2pt}\texttt{
Base::Base() is called.\\
Derived::Derived() is called.
}\\\noindent\rule{\linewidth}{.2pt}
这段代码能体现出两条信息:其一,在调用派生类的构造函数时,基类的构造函数也会被自动调用;其二,基类构造函数体的执行要早于派生类的构造函数体\footnote{注意,这是函数体的执行顺序!这个实验并没有验证初值列的顺序。}。\par
\subsection*{析构}
不只是构造函数,派生类的析构函数在调用时,也会调用基类的析构函数。我们同样根据一个例子来看它的效果。
\begin{lstlisting}
struct Base {
    ~Base() { //Base的析构函数
        std::cout << "Base::~Base() is called." << std::endl;
    }
};
struct Derived : Base {
    ~Derived() { //Derived的析构函数
        std::cout << "Derived::~Derived() is called." << std::endl;
    }
};
int main() {
    Derived de;
}
\end{lstlisting}
这段代码的运行结果如下:\\\noindent\rule{\linewidth}{.2pt}\texttt{
Derived::~Derived() is called.\\
Base::~Base() is called.
}\\\noindent\rule{\linewidth}{.2pt}
我们发现,对象销毁时析构函数的调用顺序,与对象创建时构造函数的调用顺序,刚好是相反的。在析构的时候,派生类的析构函数先调用,然后才是基类的析构函数。\par
也正因为,派生类的析构函数总是要调用基类的析构函数,所以我们根本不需要在写派生类时还要为了``基类成员的内存泄漏''而提心吊胆(只要你写的基类没有这个问题)。
\begin{lstlisting}
class stack : std::vector<int> { //class的默认继承方式为private
public:
    //...
    ~stack() {} //无需为std::vector<int>的动态内存而担扰!自有它的析构函数来回收
};
\end{lstlisting}\par
