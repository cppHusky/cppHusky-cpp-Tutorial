\chapter{程序的流程控制}
在第二章我们讲解了数据类型、运算符,以及类型转换的相关知识。对于一些初学者容易犯的错误,我们也尽量加以解释。有了这些知识之后,读者就已经具备完成一个简单的程序的能力了。\par
但是仅有这些还不够。回想一下我们的\hyperref[lst:calc1]{计算程序},我们为了求出 \lstinline@(a+b*c)/d@ 的值,先定义了四个变量,再写了一个表达式,通过输出来求得算式的值。如果用户希望自定义各个数据的值呢?只能改代码。\par
``改代码''是很可怕的。我们不能苛求每个使用计算器的用户都有看懂并修改代码的本领。所以我们可能需要条件结构(或曰选择结构),根据用户按下的按键(加减乘除,或者归零,或者别的什么)来判断应当该使用何种运算符,然后把这些规则写到程序中。\par
一个计算器可以进行多次计算。它不会每算完一个数就自动关机,等你再次开(代码1.2)机才能进行下一次计算。但是我们的程序每次算完都会终止,如何让它能够多次计算呢?我们可能需要循环结构,让这个程序周而复始地运行同一段代码,直到有一个信号让它停下来。\par
对于初学者来说,这些听上去可能有点天方夜谭,完全无从下手。没关系,本章我们就来讲解如何设计和控制程序的流程,以便写出功能更加复杂的代码——一个简单的计算器程序当然也不在话下了。\par
\import{03_control_flow/}{01_introduction_to_structure_process_and_order.tex}
\import{03_control_flow/}{02_choice.tex}
\import{03_control_flow/}{03_loop.tex}
\import{03_control_flow/}{04_introduction_to_scope.tex}
\import{03_control_flow/}{05_exercise_simple_calculator.tex}
