\chapter{类与函数进阶}
到第七章结束,读者应当已经掌握了C++语言的必要知识,并具备了独立搭建简单C++工程的能力。因此从本章开始,我也会略微转换讲解风格,不再用长篇累牍的知识点,大水漫灌。我将带领读者,从一些简单的例子开始练习,进而完成一个类,一个类模版,或是一系列函数,从而实现一些功能。\par
我的目标是,在本章的结尾带领读者写出一个简化版的 \lstinline@string@ 类。C++的 \lstinline@string@ 库中已经有这个类,按理说不需要我们实现。但是,``会用''和``会写''还是不一样的!在用的时候,我们对很多陷阱一无所知,对很多问题浑然不觉——这恰恰是封装良好的优点,我们使用某个功能时无需在这些杂碎问题上浪费不必要的时间——一旦自己从头开始搭建,这些问题就会纷纷暴露出来。\par
只有用过了知识,我们才能掌握;只有暴露了问题,我们才能进步。任何一门编程语言都是如此,C++也不例外。\par
\import{08_a_step_forward_in_classes_and_functions/}{01_operator_overloading.tex}
\import{08_a_step_forward_in_classes_and_functions/}{02_member_functions_and_friend.tex}
\import{08_a_step_forward_in_classes_and_functions/}{03_constructor_and_destructor.tex}
\import{08_a_step_forward_in_classes_and_functions/}{04_property_of_member.tex}
\import{08_a_step_forward_in_classes_and_functions/}{05_compound_types_and_objects.tex}
\import{08_a_step_forward_in_classes_and_functions/}{06_type_cast_operator.tex}
\import{08_a_step_forward_in_classes_and_functions/}{07_exercise_string_class.tex}
