\section{流输入/输出\textbf{iostream}}
\subsection*{输入信息中的分隔符}
有的时候我们会遇到需要一次性输入多个数据的情况,这时我们的选择是用空格来分隔开。
\begin{lstlisting}
    std::istringstream sin {"25 1.3 Alan Turing"}; //需要包含sstream库
    std::cin.rdbuf(sin.rdbuf()); //重定向之后,cin的源就是sin中存的字符串
    int i;
    double d;
    char str[50];
    std::cin >> i >> d >> str;
    std::cout << i << '\n' << d << '\n' << str << '\n';
    std::cin >> str;
    std::cout << str << std::endl;
\end{lstlisting}
在这里我们使用了 \lstinline@std::istringstream@ 对象来存储一段字符串输入,然后让 \lstinline@std::cin@ 重定向到它。这样,我们就可以省去了运行程序时再输入内容的麻烦,而是直接把输入内容也写到代码里——无论输入流的源是什么,它们的效果都相同。\par
这段代码的运行结果是:\\\noindent\rule{\linewidth}{.2pt}\texttt{
25\\
1.3\\
Alan\\
Turing
}\\\noindent\rule{\linewidth}{.2pt}\par
在输入信息中,\texttt{25} 和 \texttt{1.3} 之间的空格就是一个\textbf{分隔符(Delimiter)}。 在处理输入的时候,程序一旦遇到这样的分隔符,就会暂停读取输入,并把当前已经获得的输入内容(字符 \texttt{2} 和 \texttt{5} )处理成一个数据,赋值给 \lstinline@i@;接下来,它会无视下一位的空格\footnote{这个空格符作为流的一部分,仍然要被程序所接收。但是当 \lstinline@std::cin@ 发现这是一个空格符的时候,它就会什么也不做——表现出来就是无视了空格。},把后面的 \texttt{1.3} 读入其中,然后再次在空格符之前止步,把这些字符处理成数据,赋值给 \lstinline@d@;接收,它又在另一个空格符之前停下,把 \texttt{Alan} 复制给 \lstinline@str@;最后,它在文件结尾字符 \texttt{EOF}\footnote{EOF(End of file)不是一个ASCII字符(虽然它有未指明的ASCII值,通常为\texttt{-1}),但它可以用来表示文件的结尾。当输入流对象读到文件结尾字符时,输入流对象的状态发生改变,输入也就停止了。}之前停下,把 \texttt{Turing} 复制给 \lstinline@str@。于是我们就能看到这样的输出结果了。\par
