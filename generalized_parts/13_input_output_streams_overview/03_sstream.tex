\section{字符串输入/输出\texttt{sstream}}
\lstinline@std::basic_stringstream@ 继承自 \lstinline@std::basic_iostream@,表示字符串输入/输出。而\newline\lstinline@std::basic_istringstream@ 和 \lstinline@std::basic_ostingstream@ 分别继承自 \lstinline@std::basic_istream@ 和 \lstinline@std::basic_ostream@。我们在上一节中已经看到了它的一点影子,那就是用 \lstinline@std::istringstream@ 对象来产生一个新的输入流,再把 \lstinline@std::cin@ 重定向到它。\par
这三个类都定义在 \lstinline@sstream@ 库中。它们最一般的用途是向字符串中输入/输出内容。程序把一段内容通过流输入到字符串中,这是``输出'';程序从一段内容中通过流读取信息,这是``输入'',不要搞混。我们只需始终记住:信息从程序中传到外界是输出,从外界传到程序中是输入。\par
对于 \lstinline@std::stringstream@ 系列的类来说,它们都有内置的底层字符串,用以存储内容。我们可以用 \lstinline@str@ 成员函数来访问或修改它。
\begin{lstlisting}
std::basic_string<CharT, Traits, Allocator> str()const; //访问
void str(const std::basic_string<CharT, Traits, Allocator> &s); //修改
\end{lstlisting}\par
注意这个访问函数重载,它的返回类型不是引用,所以我们不能靠它来修改数据,而是必须以提供参数的方式进行。\par
下面是一个简单的使用字符串输入/输出的示例。我们先把待输入的内容存进一个\newline\lstinline@std::istringstream@ 对象(的底层字符串)中,再陆陆续续把输出内容输出到\newline\lstinline@std::ostringstream@ 对象(的底层字符串)中。但是我们不能直接看到这些输出内容,所以为了检验,我们还会用 \lstinline@std::cout@ 把 \lstinline@std::ostringstream@ 对象中的内容输出到屏幕中。\par
\lstinputlisting[caption=\texttt{string\_stream\_calculator.cpp}]{code_in_book/13.2/string_stream_calculator.cpp}
这段代码的运行结果是:\\\noindent\rule{\linewidth}{.2pt}\texttt{
1.6 6.28 0.1 2
}\\\noindent\rule{\linewidth}{.2pt}\par
在这里我们对 \lstinline@sin@ 使用了 \lstinline@str@ 成员函数来修改内容,而对 \lstinline@sout@ 使用了 \lstinline@str@ 成员函数来访问内容。\par
读者需要注意,\lstinline@sin@ 的底层字符串和 \lstinline@input@ 是两个事物,我们只是拿 \lstinline@str@ 成员函数把后者的值赋给前者而已,然后它们就没有任何关系了。
\begin{lstlisting}
    std::string input {"12345"};
    std::istringstream sin {input};
    input = "abcde"; //改变input的值,不影响sin的底层字符串
    std::cout << sin.str(); //12345
\end{lstlisting}
这个与我们此前用的 \lstinline@rdbuf@ 重定向操作不同。
\begin{lstlisting}
    std::istringstream sin {"12345"};
    std::cin.rdbuf(sin.rdbuf()); //std::cin重定向到sin的缓冲区
    std::cout << std::cin.get() << ' '; //49
    std::cout << sin.get() << ' '; //50
    std::cout << std::cin.get() << ' '; //51
    std::cout << sin.get() << ' '; //52
\end{lstlisting}
在这里,当我们进行重定向之后,\lstinline@std::cin@ 和 \lstinline@sin@ 就指向了同一个缓冲区,所以它们就会共用输入内容。\par
字符串输入/输出的内容不多,本节只是作一个过渡,接下来就让我们看看文件输入/输出的相关知识。\par\pagebreak
