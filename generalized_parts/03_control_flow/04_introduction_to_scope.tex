\section{作用域初步}
现在我们希望写这样一个代码,它接收一个正整数$n$输入,然后计算$\log_2n$的下取整值。因为只需要求下取整的结果,不要求精确值,所以我们可以用相对简单的方式来实现它。\par
对于一个输入 \lstinline@n@ 来说,如果它等于 \lstinline@1@,那么对数下取整的值当然就是 \lstinline@0@ 了;如果它大于 \lstinline@1@,那我们只需要使用 \lstinline@for@ 循环,每次将 \lstinline@n@ 整除 \lstinline@2@,直到把 \lstinline@n@ 变为 \lstinline@1@ 为止。读者可以自行举两例并手动计算之,以此来理解算法的正确性。\par
我们考虑用这段代码来实现此功能:\pagebreak
\begin{lstlisting}
    int n; //定义一个int型变量n
    cin >> n ; //输入n的值
    for (int i = 0; n > 1; ++i) {
        n /= 2; //这也是复合赋值运算符的应用,等效于n = n / 2
    }
    cout << i; //i就是最终结果
//error: 'i' was not declared in this scope
\end{lstlisting}
我们的代码思路清晰,在 \lstinline@for@ 循环处定义了 \lstinline@i@,并且初始化和判断条件都设计得非常准确,看上去毫无瑕疵。但是如果你把这段代码拿去编译,就会发现有问题。\par
编译器的报错信息已经以注释的方式放在代码中,它的意思是:``\lstinline@i@ 在这个作用域中没有定义。''但是我们在前面明明定义了 \lstinline@i@ 啊!而且,我们使用 \lstinline@i@ 的位置都在定义之后,不应该出现上述问题才对。\par
其实这个问题的关键就在于\textbf{作用域(Scope)}。在C++中,并非所有变量一经定义就可以永久使用。在很多场合之下,我们需要使用一些临时变量,它们可以帮助我们更简单地实现一些功能。但是任何变量都会占用内存空间,如果临时变量太多,可能会导致内存空间被浪费(它们大都只在很短的时间里使用廖廖几次,却要长期占据内存空间而不起实际作用)。\par
我们希望这种临时变量只有很短的生存期,它们只在我们需要的时候``被定义出来'',而在我们不需要的时候``被销毁回收'',腾出内存空间。C++如何控制变量的生存期呢?就是通过作用域来实现的。\par
这里我们只对作用域作简单介绍。最简单的情况,一对花括号 \lstinline@{}@(Braces\textsuperscript{美}/Curly brackets\textsuperscript{英})套住的范围就是一个作用域\footnote{在变量定义的统一初始化时,我们也用到了花括号,但是那个不能称之为一个作用域。详细的内容我们暂且不讲。}。下面的代码展示了一个最基本的嵌套作用域问题。
\begin{lstlisting}
int main() { //这是一个外层作用域
    double x; // 在此作用域内定义的变量可以在更内层的作用域中使用
    { //这是一个内层的作用域
        double y; //在此作用域内定义的变量可以在本层作用域中使用
        x = y = .5; //外层作用域的x和本层作用域的y都可以使用
    } //作用域结束,在此作用域内定义的变量全部失效
    x = 3.14; //本层的x可以使用
    y = 2.; //错误!y的生存期已经结束,不能再使用
    {
        //这是另一个作用域了。在这里,变量x能使用,但y不能使用
        int y; //y经过定义就可以使用了,但这里的y不同于前面的y,它是另一个变量
    }
}
\end{lstlisting}\par
\lstinline@if@, \lstinline@else@, \lstinline@for@, \lstinline@while@ 等等结构控制语句也是自带一层作用域的。比如说,\lstinline@for@ 的初始化操作中定义的内容,只在 \lstinline@for@ 及其循环体内可以使用,然而一旦出了这个循环体,\lstinline@i@ 就失效了,无法使用。这个不受花括号使用的限制——有些情况下,我们可以省略花括号,但这些控制语句仍然自带一层作用域。
\begin{lstlisting}
int main() { //外层作用域
    if(<条件>) //自带一层作用域
        int x; //相当于在内层作用域定义了变量x
    x = 3; //错误!在外层作用域中,x已失效,不能使用
}
\end{lstlisting}\par
在 \lstinline@main@ 函数体之外,还有一个全局作用域。全局作用域仅供我们定义类、变量、函数这些东西;而其它的操作,如赋值,或者结构控制,必须在函数体内完成。全局作用域下的变量又叫全局变量,它们的生存期是从程序开始到程序结束。\lstinline@cin@ 和 \lstinline@cout@ 就是两个典型的全局变量,它们定义在 \lstinline@std@ 命名空间中。\par
我们会在第七章详细讲解作用域及命名空间的有关概念。\par
