\section{实操:简易计算器的设计}
在本章的开首,我们就在考虑设计一个计算器。但因为尚不掌握程序流程控制的方法,所以不能实现。\par
现在读者已经具备这种能力了,那就让我们写一个简单的计算器,以此温习我们本章的编程知识。\par
我们先设计一个只支持四则运算和取模的计算器,比如下面的例子,其中的黑体字是输入内容,白体字是输出内容:\\\noindent\rule{\linewidth}{0.2pt}\texttt{
\textbf{8/3}\\
=2.66667\\
\textbf{8\%3}\\
=2\\
\textbf{q}\\
程序结束
}\\\noindent\rule{\linewidth}{0.2pt}\par
显然,我们需要用 \lstinline@double@ 或者别的浮点类型来存储数据,这样计算除法时才不致有精度损失。那么遇到取模运算的时候要怎么办呢?我们可以用显式类型转换的方法来变成 \lstinline@int@ 型,然后再计算。\par
接下来考虑输入的问题,类似于 \lstinline@8/3@ 这样输入要怎么读取呢?实际上,这点我们不用太过关心,因为当输入数据为浮点数值类型时,程序会自动在输入流的非预期符号前阻断(我们会在第十三章中更详细地进行探诉讨,这里读者只需知道结论即可)。接下来我们用一个字符输入就可以读取这个运算符字符。最后再用另一个数值类型读取右边的数字即可。
\begin{lstlisting}
    double num1, num2; //定义两个操作数num
    char op; //定义一个字符用来存储运算符输入
    cin >> num1 >> op >> num2; //依次输入左操作数、运算符和右操作数
\end{lstlisting}
这样就可以读取 \lstinline@8/3@ 这种格式的输入了。\par
接下来的主要问题是读取到的 \lstinline@op@ 如何处理。这个计算器可以支持四则运算和取模这五种运算,所以 \lstinline@op@ 的可能取值有五种,分别对应着加减乘除模的ASCII码值。要根据具体情况来做不同的运算,那么用 \lstinline@switch@-\lstinline@case@ 结构当然是比较方便的选择了。
\begin{lstlisting}
    switch (op) { //判断op是什么
        case '+': //根据op的值,采取相应的操作
            double result {num1 + num2}; //定义成const更好,可以防止误修改
            break;
        case '-':
            double result {num1 - num2};
            break;
        case '*':
            double result {num1 * num2};
            break;
        case '/':
            double result {num1 / num2};
            break;
        case '%':
            int result {static_cast<int>(num1) % static_cast<int>(num2)};
            //显式类型转换,将double型的num1和num2转换为int,这样才能计算
            break; //末尾的break可省略
    }
    cout << result; //试图输出result的值
//'result' was not declared in this scope
\end{lstlisting}
这段代码如果真的运行的话,编译器可能会报大量的错误。我以注释的方式将其中一条放在这个代码中,它的含义是``\lstinline@result@ 在这个作用域中没有定义''。\par
这是因为我们遇到了老问题,\lstinline@switch@ 内部是一个内层作用域,我们在内层作用域中定义的\linebreak\lstinline@result@ 只在 \lstinline@switch@ 的作用域内有效;而一旦离开这个作用域,\lstinline@result@ 的生存期就结束了,这个变量失效。于是编译器就会报告这样的错误。\par
解决这个问题的方法有两种,一是直接不定义 \lstinline@result@ 了,在每个 \lstinline@case@ 之下直接输出;二是在外层作用域中定义变量 \lstinline@result@(这时就不能再定义成 \lstinline@const@ 了,因为我们要使用赋值的方式来存储计算结果)。这里我们采用第二种方法。\par
最后,我们希望这个计算器可以多次运行。我们约定,当用户提供一个非预期的值(即,每行的开首不是 \lstinline@num1@ 预期得到的数字,而是字母或者别的什么,比如 \lstinline@q@)时程序结束,否则就可以反复循环运行。根据这个约定,我们就可以设计相应的处理方案了。\par
我们可以用 \lstinline@cin@ 的状态作为 \lstinline@while@ 循环的条件来判断循环是否要终止,而在 \lstinline@cin@ 之后必须检测一次 \lstinline@cin@ 的状态,如果不正常,就要 \lstinline@break@。
\begin{lstlisting}
    while (cin) { //如果cin状态正常,就进行循环(非必需,改成true也可)
        double num1, num2, result; //定义两个操作数num及结果值result
        char op; //定义一个字符用来存储运算符输入
        cin >> num1 >> op >> num2; //依次输入左操作数、运算符和右操作数
        if (!cin) { //如果cin状态不正常,说明输入了非预期内容,退出循环
            break;
        }
        //switch-case结构部分省略
        cout << "=" << result << endl; //输出result的值
    }
    cout << "Done" << endl; //提示用户程序结束
\end{lstlisting}\par
那么我们的思路就比较清晰了。现在让我们把头文件包含等等杂七杂八的代码加上,就可以形成一个完整的程序了。\par
\lstinputlisting[caption=\texttt{simple\_calculator.cpp},label=lst:SimpleCalculator]{../code_in_book/3.5/simple_calculator.cpp}\par
读者可以自行测试此计算器的效果。\par
