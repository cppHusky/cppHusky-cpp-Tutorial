\section{数据的定义和使用}
前一节我们介绍了三种数据类型:整型、浮点型和字符型。这一节让我们回到代码中来,看看如何定义一个数据,又如何使用它。\par
C++中有一些关键字,用特定的关键字就可以定义特定类型的数据,比如
\begin{lstlisting}
#include <iostream>
using namespace std;
int main(){
    int i = -1; //定义一个int整型数据,并初始化为-1
    double d = 3.14159; //定义一个double浮点型数据,并初始化为3.14159
    char c = '0'; //定义一个char字符型数据,并初始化为'0',其ASCII码为48
    return 0;
}
\end{lstlisting}
这三个语句的结构非常相似,我们可以把它们归纳成一个统一的形式:
\begin{lstlisting}
    <类型关键字> <名字> = <值>;
\end{lstlisting}
接下来我们分别介绍这三个部分。\par
\subsection*{类型关键字(Keyword)}
C++为不同类型提供了丰富的关键字。\par
对于整型来说,\lstinline@int@ 是最基本的关键字。我们可以通过添加前缀\footnote{前缀(Prefix)和后缀(Postfix)的概念在编程中十分普遍。在编程语言中,前缀就是在一个词前面添加若干信息;而后缀就是在一个词后面添加若干信息。}的方式来将 \lstinline@int@ 型改为其它整型。\par
\begin{itemize}
    \item \textbf{改变符号性}:我们可以添加 \lstinline@signed@(有符号)前缀,允许它表示负数;或者增加 \lstinline@unsigned@(无符号)前缀,禁止它表示负数。如果不加前缀,将默认允许表示负数。因此,如果没有禁止负数的需要,无需加前缀 \lstinline@unsigned@。
    \item \textbf{改变``容量''}:我们可以添加 \lstinline@short@, \lstinline@long@ 或者 \lstinline@long long@ 前缀来改变它的容量。简便起见,如果有这三个前缀之一,可以直接省略 \lstinline@int@。因此,我们可以直接用 \lstinline@long long@ 这样的关键字而不必用 \lstinline@long long int@。
\end{itemize}\par
无论改变符号性还是改变容量,其目的都在于让我们的数据合乎需求。如果我们要统计某物的数量,那么这个数字肯定是非负的,于是我们可以用 \lstinline@unsigned@ 来限制它的取值范围;如果某物的数量可能很多,\lstinline@int@ 类型的容量存不下,那么我们就要考虑使用 \lstinline@long long@了。\par
对于浮点型来说,\lstinline@double@ 是最常用的关键字。另有 \lstinline@float@ 和 \lstinline@long double@,它们的有效位数和容量也不同。简单来说,\lstinline@float@ 的有效位数和容量最少,\lstinline@double@ 居中,\lstinline@long double@ 最多\footnote{但是不尽然,比如有些开发环境中 \lstinline@double@ 和 \lstinline@long double@ 一样多。}。浮点型数据无法改变符号性,它们必须允许负数的表示。\par
对于字符型来说,\lstinline@char@ 是最常用的关键字\footnote{C++还规定了两个类型:\lstinline@signed char@ 和 \lstinline@unsigned char@。请注意,它们和 \lstinline@char@ 是三种互不相同的类型。这一点与 \lstinline@int@ 等于 \lstinline@signed int@ 的情况截然不同。读者尚无必要纠结其中细节,精讲篇会予以介绍。},它能表示ASCII的所有字符,但对非拉丁字母(比如中日韩文字及变体)无能为力。为了表示非拉丁字母,我们可以使用 \lstinline@wchar_t@ 类型\footnote{不过,随着Unicode的普及,\lstinline@wchar_t@ 这种宽字符表示法的必要性已经越来越低了。泛讲篇中我们也不讲它。}。\par
其实我们在实际编程时最常使用的关键字是 \lstinline@int@, \lstinline@double@, \lstinline@char@。而其它的关键字,如 \lstinline@unsigned@ 或 \lstinline@long double@ 往往是我们有特殊需求时才会去考虑使用的。简便起见,我们在这一章只使用这三种基本类型。它们足够应付我们的需要了。\par
\subsection*{标识符(名字)(Identifier)}
在定义数据时,我们需要给它一个名字,以便日后使用。这个名字就是一种标识符。\par
在C++中,标识符指代的概念甚广,包含一切关键字,以及我们定义的所有名字。在遵守下述要求的前提下,我们可以起各种千奇百怪的名字:
\begin{enumerate}
    \item 一个标识符只能由英文字母A-Z和a-z、数字0-9以及下划线\_构成。\footnote{这是不完全的,实际上还可以用其它具有XID\_Start及XID\_Continue属性的字符。本书无意在这个问题上走得太远,也不打算使用这种字符来作为标识符。感兴趣的读者可以参阅 \href{https://www.unicode.org/reports/tr31/\#Table\_Lexical\_Classes\_for\_Identifiers}{Properties for Lexical Classes for Identifiers - Unicode Technical Reports} 获取更多信息。}
    \item 标识符的首字母不能是数字,但可以是字母和下划线。\footnote{有一类特殊的反例,参见 \href{https://zh.cppreference.com/w/cpp/language/user_literal}{用户定义字面量-cppreference}。这种情况不在我们的讨论范围之内。}
    \item 标识符对大小写敏感。这意味着,即便两个标识符只有大小写上的区别,它们也是不相同的。
    \item C++中的关键字均被保护\footnote{许多代码编辑器会对关键字做特殊着色处理。本书将关键字一律以蓝色加粗来标明(偶有不宜着色者除外)。},你不能定义一个与关键字相同的名字。
    \item 标识符的长度可能受到系统或开发环境的限制,不能太长。\footnote{C++标准并未对标识符的长度作明确规定,实际编程中也几乎不会有长度超限的情况,所以本条不那么重要。}
\end{enumerate}
比如我在上述代码中起了三个名字:\lstinline@i@, \lstinline@d@, \lstinline@c@,这些名字都是允许的。\par
不只是数据名。以后我们可能需要起函数名、结构名和类名,这些命名方式全部应该按照这样的标准来起名。举几个例子:``\lstinline@lhs@'', ``\lstinline@it_1@''和``\lstinline@__max@'' 都是可以使用的名字,而``\lstinline@12a@'', ``\lstinline@张三@''和``\lstinline@new@''都是不能使用的名字(\lstinline@new@ 是C++中的关键字)。\par
\subsection*{初始化(Initialization)}
一个数据存储一个值。整型数据存储整数值,浮点型数据存储小数值,字符型数据存储字符值(对于 \lstinline@char@ 型来说,它存储的值可以用对应的ASCII码值来表示。)如果你不为它指定一个值的话,它就会存储一个不确定的值\footnote{严格说来,只有局部变量才会如此。全局变量如果不加指定,将会自动初始化为零。}。使用``不确定的值''是危险的,我们要尽量避免。\par
初始化是一种``在定义的同时就给定值''的方法。在本例中,我们使用\lstinline@=@加一个值的形式来实现初始化。
\begin{lstlisting}
    unsigned long long ull = 4294967296;
    //定义一个unsigned long long整型数据,并初始化为4294967296
    long double ld = 3.141592653590;
    //定义一个long double浮点型数据,并初始化为3.141592653590
\end{lstlisting}
以上初始化语法也叫作\textbf{直接初始化(Direct Initialization)}。\par
初始化的语法不只这一种。在C++11以后的语言标准中,我更推荐使用\textbf{统一初始化(Uniform Initialization)}\footnote{有关``统一初始化''这个词,实在让人费解。C++标准中没有查到过这个词,cppreference中也没有查到这个词。但这个词却在网络中普遍存在,十分常用。笔者无法在概念问题上耗费太多时间,因此请读者留意,本书很有可能会在这里犯错。}的方法:
\begin{lstlisting}
    <类型关键字> <名字> = {值}; //统一初始化语法
    <类型关键字> <名字> {值}; //另一种统一初始化语法
\end{lstlisting}
统一初始化的语法更强大,在进行初始化时会进行检测,避免范围缩限等问题。\par
举个例子,某开发环境中的 \lstinline@short@ 型变量只能容纳-32768到32767(包含两端)之间的整数\footnote{这个范围可能因电脑和开发环境而异。如果读者要自己尝试,应以自己的实际情况为准。}。那么这样的代码是可以通过编译并且运行的:
\begin{lstlisting}
    short s1 = 40000; //定义s1并直接初始化为40000
    cout << s1; //输出s1的值
\end{lstlisting}
这个输出的结果可能是 \lstinline@-25536@。同时,有些编译器(如GCC)还可能给出如下警告信息:
\begin{lstlisting}
//warning: overflow in conversion from ‘int’ to ‘short int’ changes value
//from ‘40000’ to ‘-25536’ [-Woverflow]
\end{lstlisting}\par
在这里,给一个不能容纳40000的数据存40000的数值本来就是违规操作。警告信息(Warning)虽然是一个提醒,但仍然不会阻止你运行这个程序并得到错误结果,更遑论一些老旧编译器甚至不会给你警告信息。统一初始化则不然,它会直接以报告错误(Error)的方式阻止你运行程序,你必须改正这个违规操作才行。
\begin{lstlisting}
    short s2 = {40000}; //定义s2并统一初始化为40000,这种方法是禁止的
\end{lstlisting}\par
统一初始化还有许多其它的优点,这也是我推荐使用统一初始化的理由。\par
\subsection*{实操:让编译器充当计算器}
有了数据之后,我们可以拿它们来做一些简单的计算了。我们希望定义一些数据,让它们进行简单的四则运算。我们在开始动工之前,需要粗略规划一下我们要做的事情。我们的大致步骤是:
\begin{enumerate}
    \item 定义几个数据,并初始化。考虑到我们可能会计算小数,我们应该用浮点型的 \lstinline@double@ 类型。
    \item 把我们想算的式子写到代码的输出语句中。比如说,我想计算 \lstinline@(a+b*c)/d@,那么我应该把 \lstinline@cout<<@ 后面改成我想输出的这个式子。    \item 编译、运行,看下输出的结果。
\end{enumerate}\par
现在我们完成第一步,定义数据。从我们的式子中可以看出,这里需要4个浮点型变量,那么我就定义4个浮点型变量。同时我还需要给出4个初始值。
\begin{lstlisting}
    double a = {1.5};
    double b = {0.3};
    double c {0.5}; //这也是一种统一初始化语法,省略了=
    double d {4.5};
\end{lstlisting}
其实C++还支持另一种简便的语法。如果要定义的量是同类型的,那么我们可以用单个类型关键字,将变量名隔开来定义。
\begin{lstlisting}
    double a {1.5}, b {0.3}, c {0.5}, d {4.5}; //定义并初始化
\end{lstlisting}
注意,句末用的是分号,而句中用的是逗号。这样就不用写4次 \lstinline@double@ 了,省时省力。\par
接下来是第二步,我们改一下输出代码,让它输出我们需要的内容。
\begin{lstlisting}
    cout << (a + b * c) / d; //输出(a+b*c)/d的值
\end{lstlisting}\par
这段代码不能单独存在,现在让我们``借鉴''一下\texttt{Hello\_World.cpp}的代码,把它改成这样:
\lstinputlisting[caption=\texttt{calc.cpp}\label{lst:calc1}]{../code_in_book/1.3/calc.cpp}
最后的输出结果是一个近似值 \lstinline@0.366667@,与实际计算器算出的结果相近。\par
你还可以修改代码中的初始值,或者是把输出的式子改成别的式子,观察一下结果。\par
