\section{类与对象}
C++语言脱胎于C语言。斯特劳斯特鲁普\footnote{比雅尼·斯特劳斯特鲁普(Bjarne Stroustrup),丹麦计算机科学家。他发明并发展了C++语言,以其对C++语言的卓著贡献而享誉世界。如果读者有意了解关于他和C++的更多信息,可以访问 \href{https://www.stroustrup.com/}{Stroustrup的个人主页}。}最初选择了C语言,首先在C的基础上发展出了C with Classes。后来C++不断扩容并增加新功能,直至C++98标准发布,C++语言的基本功能已经成形。之后的C++11又是一个划时代的更新,不过这些就离我现在想要谈的事情比较远了。\par
迈耶斯\footnote{斯科特·道格拉斯·迈耶斯(Scott Douglas Meyers),美国作家、软件顾问和C++语言程序设计专家,以其所著\textit{Effective C++}系列书而闻名。}在他的著作《Effective C++》中,将C++语言视为四种``子语言''的统合体:
\begin{itemize}
    \item \textbf{C}:这是脱胎于C语言的部分,它继承和发展了C语言中的部分功能。所以当有人问起``如果没学过C语言,是不是要先学C再学C++''的时候,我们的回答都是``不必如此''!在C++的起步阶段,你学习的内容已经基本涵盖C语言的基本语法了。
    \item \textbf{面向对象的C}:这里有了比C语言中 \lstinline@struct@ 丰富得多的功能,它也是C with Classes最初区别于C的特征。
    \item \textbf{泛型C}:这里提供了泛型编程和元编程的支持,使C++成为一门更为强大的语言。它们是C++与C with Classes的重要区别之一。
    \item \textbf{标准模板库(STL)}:STL由斯捷潘诺夫\footnote{亚历山大·亚历山德罗维奇·斯捷潘诺夫(Alexander Alexandrovich Stepanov),俄罗斯裔美国计算机科学家,以其提倡泛型编程与设计实现了C++语言中的标准模板库而闻名。}设计实现。相比于单纯的泛型编程,它有一套相对独立的逻辑。STL更多地支持对``现成''数据结构和算法的使用,其优点是方便、高效、简洁。
\end{itemize}
这些内容都会在本书的泛讲篇或精讲篇有所涉及。但本书并非进阶教程,读者若有意深入挖掘,还应选择合适的进阶教材阅读。\par
从C++的发展史来看,\textbf{面向对象编程(Object-oriented programming, OOP)}是C++相对于C跨出的第一步。面向对象的概念广泛用于C++的语法中,以至于我们往往日用而不知。\par
我本来不打算在第一章就介绍面向对象的概念,但我很快就发现那很难。如果不了解什么是面向对象,也不清楚相关的基本概念,后面几章的知识依然都能学(别忘了,C语言可没有面向对象,那些C有关的语法也都不依托这方面的知识);然而,这会在``C''部分与``面向对象的C''部分的过渡处造成很大的障碍(尤其对于初学者来说)。正如迈耶斯所说,它们几乎是两个子语言,为此你需要一定的适应期,并且回顾很多以往的知识,才能真正理解它。\par
出于上述考虑,我决定在第一章就介绍面向对象,并且在后面的章节中也会时常提起它(甚至把它和一部分C内容串讲)。我希望通过这种方式,读者能够真正把``一门语言''学成``一门语言'',而不是``四门语言''——如果可以的话,这也是我对本书的一点期许。\par
\subsection*{什么是类?什么是对象?}
类与对象是一对概念,不得不并举。\par
\textbf{类(Class)}是一个抽象的概念,它规定它的所有对象都有一些属性和功能。而\textbf{对象(Object)}\footnote{``对象''这个译名容易引起困感;相比之下,英文原名object和繁体中文译名``物件''就显得易懂些。}则是一个个具体的实现。举例来说,``哈士奇''是一个抽象的概念(类),它规定了其对象的一些属性,比如``身高''、``体重'',还有其对象的一些功能,比如``拆家''。而``我家的哈士奇''和``张三家的哈士奇''是两个实体(对象),它们衍生自``哈士奇''这个概念。\par
同一个类的不同对象都有这些属性,但属性的具体值是因对象而异的。我家的哈士奇的体重是27公斤,而张三家哈士奇的体重是20公斤。属性不同,会导致它们在执行一些功能时有不同的表现。比如,有个沙发的承重是25公斤,那么我家哈士奇拆沙发的效率当然就要好于张三家的哈士奇。\par
我们经常容易把类与对象的关系和另外两组关系搞混。所以不妨让我们来做一下辨析:\par
\begin{itemize}
    \item ``狗''与``哈士奇''的关系不是类与对象关系。它们两个都是抽象的概念,只不过``哈士奇''在``狗''的基础上进一步细化了属性和功能。比如说,``拆家''不是所有狗的特征,它不是``狗''的功能;而``哈士奇''在``狗''这个概念的基础上增加了``拆家''的功能。然而说到底,``哈士奇''依然是一个概念,而非实体。\footnote{在第九章中我们会讲到,这是两个类的继承关系。}
    \item ``猫''与``我家哈士奇''的关系不是类与对象关系。``猫''的确是一个类,``我家哈士奇''也的确是一个对象,但它们完全八竿子打不着。我家哈士奇根本就不是猫。
    \item ``张三''与``张三的心脏''的关系不是类与对象关系。``张三''确实是由各个器官构成的,然而张三是一个实体,它们之间的关系应当是整体与部分,而不是类与对象。
    \item ``人''与``张三的心脏''的关系也不是类与对象关系。我们不能说``张三的心脏''就是一个人。
    \item ``人''与``张三''的关系是类与对象关系。张三是一个人。
\end{itemize}
可以看出,类与对象的关系可以用``a是一个A''的语句来检验和阐述,比如:秦始皇是一个皇帝;上海是一座城市;银河系是一个``有生命存在的星系''。\par
\subsection*{C++中的类与对象}
面向对象的概念并非C++独有;不过本书只介绍C++中的面向对象。\par
在C++中,类型(Type)和类(Class)在很多时候是等价概念。\footnote{在泛型编程中,我们一般将其视为等价概念,使用 \lstinline@typename@ 和 \lstinline@class@ 关键字的效果是相同的。不过,在另一些语境下,class是比type更为广泛的概念。}假设我定义了一个 \lstinline@int@ 型数据 \lstinline@i@,那么它们两者之间的关系是不是类与对象呢?\par
我们可以用``a是一个A''的语句来检验这个关系。就像我们反复强调的那样,\lstinline@i@ 是一个 \lstinline@int@ 型的数据,于是``\lstinline@i@ 是一个 \lstinline@int@''的判断成立,它们是类与对象的关系。\par
我们可以在同一个类之下定义很多对象。
\begin{lstlisting}
    double a {.1}, b {2.2}, c {3.}; //在double类之下定义三个对象a,b,c
\end{lstlisting}
这里的 \lstinline@double@ 就是一个抽象的概念,而 \lstinline@a@, \lstinline@b@, \lstinline@c@ 就是这一概念下创造出来的实体。\par
C++中还有一个更好的例子,就是我们一直在用但是还没有讲的 \lstinline@cout@。这是一个 \lstinline@ostream@ 类的外部对象\footnote{关于外部对象等概念,我们将在第七章中介绍。}。\lstinline@ostream@ 类重载了 \lstinline@<<@ 运算符\footnote{关于运算符重载,我们将在第八章中介绍。},使得我们可以方便地输出很多基本类型对象的值。\par
\lstinline@ostream@ 只是一个类,它提供了诸多输出方面的功能支持,但是倘若没有一个实体,这些功能都是空中楼阁,无从实现。为此,C++在 \lstinline@std@ 命名空间中定义了 \lstinline@cout@ 作为标准输出对象,所以我们才可以使用这个对象来实现基本类型值的输出。
\begin{lstlisting}
namespace std{
    extern std::ostream cout;
};
\end{lstlisting}
这个对象定义在头文件 \lstinline@iostream@ 中,所以我们必须要在每个完整的程序代码之前加上
\begin{lstlisting}
#include <iostream> //包含头文件iostream
\end{lstlisting}
方可使用它。\par
C++还定义了标准输入 \lstinline@cin@,也在头文件 \lstinline@iostream@ 当中。我们会在下一节中使用它。\par
