\section{运算符与类型}
在上一节的末尾,我们实现了一个简单的基于浮点数的计算器。这一节我想走得更远,增加一些数据类型和新的运算,看看会不会有什么新变化。\par
\subsection*{整数的模运算}
\textbf{模运算(Modulo)},俗称取余,简记为$\mathrm{mod}$,就是求两个数相除得到的余数。例如,$5\div2=2\ldots1$,这里的余数就是$1$,所以$5\:\mathrm{mod}\:2=1$;再例如,$9\div3=3\ldots0$,这里的余数就是$0$,所以$9\:\mathrm{mod}\:3=0$。\par
一般意义上讲,只有整数才有带余除法,所以我们在计算时应该使用整型,比如 \lstinline@int@。
如果要计算某两个数 \lstinline@a@ 与 \lstinline@b@ 的模值,我们可以用如下的代码来实现:
\begin{lstlisting}
    int a {1584}, b {7}; //定义a, b并初始化
    cout << a % b; //求a mod b的值
\end{lstlisting}
这里使用的百分记号(\lstinline@%@)可能会让人费解。其实在C++中,\lstinline@%@ 并不是百分记号,而是取模运算符,含义是左边整数除以右边整数得到的余数(模值)。在各类编程语言中,这种``记号含义与传统含义不同''的现象比比皆是,以下是C++当中的部分例子(读者无需现在就掌握,但日后要留心):
\begin{itemize}
    \item \lstinline@^@ 不是冪指数的含义,而是位异或运算。\footnote{关于位异或和移位运算符,参见附录C.2 布尔代数基础。}%tbt
    \item \lstinline@<<@ 和 \lstinline@>>@ 既不是书名号,也不是数学意义上的``远小于''和``远大于'',而是移位运算符。
    \item \lstinline@[]@ 和 \lstinline@{}@ 不是数学上的``中括号''和``大括号'',它们分别有另外的含义。数学意义上的括号统一使用 \lstinline@()@ 嵌套来实现。
    \item \lstinline@=@ 是赋值运算符,而传统意义上相等性的判断应当用 \lstinline@==@。
    \item \dots\dots
\end{itemize}\par
要计算 \lstinline@1584%7@ 的值,也可以跳过定义数据这一步,直接在输出语句中使用字面量\footnote{字面量(Literal),是一个值在代码中不具名的文本表示。比如 \lstinline@1584@,它没有``名字'',在代码中被写出来就是1584,它的值也是1584。与之相对的是具名数据,它们经过定义,就有了名字。}完成。
\begin{lstlisting}
    cout << 1584 % 7; //直接求 1584 mod 7,不必定义a和b
\end{lstlisting}
这种写法更加简便、直观,不需要定义数据。但也请注意,它无法支持程序层面的交互(如,用户输入两个数并计算。我们总不能要求用户来亲自修改代码。关于程序输入,我们会在第二章介绍)。\par
\subsection*{字面量的类型}
C++会自动识别字面量的类型,比如前面介绍过的这个例子(编译器的报错信息以注释形式附于其后):
\begin{lstlisting}
    short s2 = {40000};
//error: narrowing conversion of '40000' from 'int'
//to 'short int' [-Wnarrowing]
\end{lstlisting}
编译器的报错信息是很重要的,它能把出错的关键信息告诉我们,帮助我们有针对性地查找和解决问题。这个报错信息是``\lstinline@40000@ 在从 \lstinline@int@ 到 \lstinline@short int@ 的过程中发生了收缩转换'',意思就是``原本 \lstinline@40000@ 是 \lstinline@int@ 类型的,完全装得下;但是你要往 \lstinline@short@ 里面塞,那就装不下了''。\par
于是从报错信息中我们能看出,\lstinline@40000@这个值被编译器识别成 \lstinline@int@ 型。\par
再来个类似的例子:
\begin{lstlisting}
    int i {2147483648};
//error: narrowing conversion of '2147483648' from 'long long int'
//to 'int' [-Wnarrowing]
\end{lstlisting}
这个报错信息是``\lstinline@2147483648@ 在从 \lstinline@long long int@\footnote{这点可能因电脑的环境而异,比如有些电脑的 \lstinline@long@ 类型就可以容纳这个数,那么报错信息应当是 \lstinline@long int@;而另一些则不能容纳这个数,报错信息应当是 \lstinline@long long int@。} 到 \lstinline@int@ 的过程中发生了收缩转换'',意思就是``原本的 \lstinline@long long@ 类型完全装得下,但是 \lstinline@int@ 类型装不下''。换言之,字面量 \lstinline@2147483648@ 被编译器识别成了 \lstinline@long long@ 型。\par
总得说来,如无前缀或后缀\footnote{字面量可加前缀或后缀来指定优先使用的类型,第二章会对此进行介绍。},代码中的浮点字面量统一识别成 \lstinline@double@ 型。代码中的整数字面量会按照 \lstinline@int@, \lstinline@long@, \lstinline@long long@ 的顺序,找到能容纳它的类型,作为这个字面量的类型;而字符字面量会把ASCII字符统一识别成 \lstinline@char@ 型\footnote{关于宽字符,我们会在精讲篇中再谈。}。\par
字面量的实际知识远比我们介绍的要多。如果读者有意进一步了解相关知识,可以查阅\href{https://zh.cppreference.com/w/cpp/language/expressions#.E5.AD.97.E9.9D.A2.E9.87.8F}{字面量-表达式-cppreference}。\par
\subsubsection*{\texttt{1}和\texttt{1.0}的区别是什么?}
我们在写代码时常常会遇到 \lstinline@1.0@ 这样的写法。你可能会好奇,为什么不写成 \lstinline@1@ 呢?究其原因,在于 \lstinline@1.0@ 是一个浮点字面量,而 \lstinline@1@ 则是一个整数字面量。它们的类型并不相同,一个是 \lstinline@double@,另一个则是 \lstinline@int@。\par
C++中还有更简洁的语法,就是用 \lstinline@1.@ 和 \lstinline@.1@ 来取代 \lstinline@1.0@ 和 \lstinline@0.1@ 这样的浮点字面量。比如说
\begin{lstlisting}
    cout << 1. / .1; //等价于cout << 1.0 / 0.1;
\end{lstlisting}\par
\subsection*{整数除法问题}
让我们来看一下这段代码:
\begin{lstlisting}
    cout << 8 / 3; //计算8/3的值
\end{lstlisting}
猜一下它的运行结果。如果你手头有编译条件,可以跑一下代码看看(记得加上那一堆必需代码)。\par
如果你猜的结果是 \lstinline@2.66667@ 或者是类似的近似结果,那么恭喜你,这是初学者常犯的错误!\par
整数除法与浮点数除法有所不同。如果两数不能整除,那么结果就会被``截尾''\footnote{截尾(Truncation),广义上指对小数点后数字个数的限制。在这里指的是直接截掉小数点后的部分,比如$1.618\rightarrow1$和$-2.718\rightarrow-2$。这种方法与四舍五入和单纯上/下取整都不同,可以把它理解成``对正数向下取整,对负数向上取整''。}。如果要避免截尾,获得较精确的数据,应该使用符点型来操作。
\begin{lstlisting}
    cout << 8. / 3.; //计算8.0/3.0的值
\end{lstlisting}\par
那如果我要计算两个具名整型数据的除法呢?比如在 \lstinline@a@ 和 \lstinline@b@ 都是整型的情况下计算 \lstinline@a/b@?这时我们可以使用第二章中将会介绍的类型转换的方法。\par
这里有一种比较简单的类型转换方法:
\begin{lstlisting}
    cout << 1. * a / b; //计算a/b的浮点数结果
\end{lstlisting}
但是请注意,下面的写法是不行的:
\begin{lstlisting}
    cout << a / b * 1.; //计算a/b的浮点数结果,但事与愿违
\end{lstlisting}
关于第一种方法为什么可行,第二种为什么不可行,我们会在第二章中探讨。\par
总而言之,类型不同,可能会让结果出现一些意想不到的差异。因此类型是很重要的数据特征,日后编程时我们需要多加留意。\par
