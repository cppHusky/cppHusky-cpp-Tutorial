\section{开始一个C++程序}
各种编程语言的教程都喜欢用``输出Hello World!''作为第一个编程示例,本书也不例外。\par
在C++中,我们可以用如下的代码来输出``Hello World!'':
\lstinputlisting[caption=\texttt{Hello\_World.cpp},label={lst:HelloWorld}]{../code_in_book/1.1/Hello_World.cpp}\par
如果你使用的是集成开发环境,那么只需要用一两个选项或快捷键就可以完成编译、运行步骤,并看到程序的运行结果了。这个快捷键因软件而异,不过总得说来,可以概括成三个过程:
\begin{enumerate}
    \item \textbf{编译(Compile)}:编译器将你写在文件中的源代码\footnote{源代码(Source Code),是一系列人类可读的计算机语言指令,在C++中一般指.cpp文件中的代码。}编译成目标代码\footnote{目标代码(Object Code),是源代码经编译器或汇编器处理后产生的代码,对人类来说可读性很差。机器不能直接执行源代码,只能执行目标代码。}。
    \item \textbf{链接(Link)}:将目标代码与其它代码链接。C++标准库提供了非常丰富的功能,而链接器负责将我们的源代码与这些功能对应的库,以及其它必需文件结合起来,形成可执行文件。
    \item \textbf{运行(Run)}:可执行文件在运行时,可以进行计算、存取和用户交互等各种行为。程序可以长期处于运行状态,也可以在某个指令后结束运行。
\end{enumerate}
\begin{figure}[htbp]
    \centering
    \includegraphics[width=0.8\textwidth]{../images/generalized_parts/01_From_source_code_to_executable.drawio.png}
    \caption{从源代码到可执行文件}
\end{figure}
实际上的过程会更复杂,比如在编译之前还会进行预处理(Preprocess)。但是我们现在还不需要操心这么多,因为在按下``编译''键之后,预处理和编译都会进行,所以不妨把它统称为一个过程。\par
\subsection*{编译器的选择}
C++的\textbf{编译器(Compiler)}非常丰富。\href{https://en.wikipedia.org/wiki/List_of_compilers\#C++_compilers}{List of C++ Compilers - Wikipedia} 条目为我们提供了一个不完全的C++编译器列表。\par
一般我们选择使用Clang、GCC或Visual C++(MSVC),它们都是跨系统的编译器,对语言标准的支持比较好\footnote{Clang和GCC支持C++17以前标准和C++20的大部分标准,而MSVC支持C++20的全部标准和C++23的部分标准。},许多IDE(如Dev\_C++, Code::Blocks,Visual C++等)也都默认提供这些编译器之一。另外,Coliru, Wandbox和C++ Shell等在线编译器也都支持使用Clang或GCC。\par
同种的编译器也会有版本上的差异。很多古老的版本(如GCC 4.9.4)并不支持新近的语法(如C++17\footnote{C++17是C++委员会制定的一个语言标准,发布于2017年,代号为 \lstinline@201703L@。})。新版本的编译器往往有很多好用的特性,比如对代码有更好的优化。\footnote{另请注意,最好选择稳定版本的编译器,而非试验性的预览版,以免编译器自身存在问题带来的麻烦。}\par
\subsection*{语言标准的选择}
不同的C++\textbf{标准(Standard)}对语法的支持不同。比如说,在C++11以后的标准中下,可以使用 \lstinline@auto@ 来自动确定我们要定义的变量类型。这是一种很方便的语法,但C++11以前的版本就不支持使用。\par
C++语言标准的更新总会带来很多新的功能,也会对以往标准中的缺陷和漏洞加以弥补,所以越新的标准往往就越强大、易用。另一方面,新标准通常都能兼容旧标准\footnote{只是通常如此。有很多反例,比如C++17标准移除了古已有之的 \lstinline@std::auto_ptr@。},所以你也无需担心为了适应新标准而放弃很多旧知识,或者换用新标准后很多代码编译失败了。\par
可惜并非所有编译器都能支持C++新标准的特性。因此,在编写本书时,考虑到许多编译器对语言标准支持的情况,本书选择使用C++17标准。\textbf{今后若无特殊说明,默认本书所有知识和代码遵循C++17标准。}\par
\subsubsection*{Windows下运行的可能问题}
在Windows系统中,很多IDE允许你在一个辅助窗口中运行程序。有些IDE在程序执行完毕后,会保留辅助窗口,等待用户``按任意键继续'',于是我们有足够的时间看到程序运行的结果;还有一些IDE在程序结束时会立刻关闭辅助窗口,我们就来不及看程序运行的结果。\par
为了解决这个问题,我们可能需要对代码做一些修改:
\begin{lstlisting}
    //...前面的代码不变
    cin.get();//可能需要添加的语句
    cin.get();//甚至可能需要再加这句
    return 0;
}
\end{lstlisting}
\lstinline@cin.get()@ 的作用是读取下一个键盘输入,所以如果你不按下键盘上的Enter键,程序就会一直等待下去。(只有按下Enter,输入内容才能发送给程序,我们后面会慢慢道来)\par
至于用两个 \lstinline@cin.get()@,这是为了应对可能的 \lstinline@'\n'@ 遗留被第一个 \lstinline@cin.get()@ 吞没。我们现在尚无必要在乎这些细枝末节,只需要知道:如果窗口没有保持打开,就加一个 \lstinline@cin.get()@;如果还不行,就加两个 \lstinline@cin.get()@。\footnote{值得留意的是,用 \lstinline@cin.get()@ 来阻断程序进度只是一种临时的解决方案,目的在于方便我们进行学习和演示。而在实际编程中,人们通常不会在程序中添加这种代码。}\par
\subsection*{回到Hello World}
现在让我们回来看Hello World代码。这个代码可以在任何操作系统上,用任何编译器和任何语言标准来编译。程序的输出就是一个单词 \texttt{Hello World!}。\par
\begin{lstlisting}
// 这是一个简单的C++程序,用于输出Hello World
\end{lstlisting}
程序的第一行是一个\textbf{注释(Comment)}。注释以两个双斜线开头,其后直到行末的内容均会被编译器忽略,所以你写什么都可以。注释是为了给人类阅读的,以便我们理解程序和代码的含义。\par
\begin{lstlisting}
#include <iostream> //标准输入输出头文件
\end{lstlisting}
程序的第二行是一个\textbf{文件包含(File Inclusion)}。通过这个语句,预处理器会对代码作预处理\footnote{如前所述,源代码在编译之前需要经过预处理,例如文件包含和宏的展开。预处理结束后,编译才会开始。},把 \lstinline@iostream@ 库中的代码导入该文件中\footnote{所以这个源代码的实际的内容量会比你看上去的大很多,绝大多数来自于文件包含命令。}。\par
\begin{lstlisting}
using namespace std; //使用std命名空间
\end{lstlisting}
程序的第三行是一个\textbf{命名空间(Namespace)}使用。C++中,\lstinline@cout@ 被定义\footnote{定义(Definition),是为程序中的一个实体(如变量、函数、类等)分配内存并提供其具体的实现或值的过程。在以C++为代表的静态语言中,一个实体必须定义才能使用。}在 \lstinline@std@ 命名空间中,我们要通过 \lstinline@std::cout@ 这样的语法才可以使用它。而有了 \lstinline@using namespace std@ 之后,我们就可以直接使用 \lstinline{cout}。\par
关于文件包含和命名空间的概念,我们会在第七章进行讲解。目前我们无需过分纠结。\par
\begin{lstlisting}
int main(){//主函数,程序执行始于此
    //函数体代码
}
\end{lstlisting}
程序的第四行是\textbf{主函数(Main Function)}的定义。主函数是程序的入口,它必须唯一。程序从主函数开始按顺序执行各句代码,除非遇到了条件或循环结构。关于函数,我们会在第四章讲解;关于结构,我们会在第三章讲解。\par
\begin{lstlisting}
    cout << "Hello World!";//输出Hello World
\end{lstlisting}
程序的第五行是\textbf{输出(Output)}语句。这一句对我们来说更重要,因为我们接下来要模仿的代码主要是基于修改输出内容来实现的。\par
\lstinline@<<@ 左侧的 \lstinline@cout@ 是在标准库 \lstinline@iostream@ 中定义的 \lstinline@std@ 空间的对象,控制标准输出。我们无需对它作修改;\lstinline@<<@ 右侧的 \lstinline@"hello World!"@ 则是我们要输出的内容。外套双引号表示它是一个字符串,我们会在第五章讲解字符串。\par
如果我们想要输出一个数字,我们只需要修改 \lstinline@<<@ 右侧的内容即可,比如
\begin{lstlisting}
    cout << 3.14159;//输出3.14159这个数
\end{lstlisting}
第三节我们就会开始介绍如何使用 \lstinline@cout@ 来输出数据。\par
\subsection*{输出你的编译器/语言标准信息}
读者可能搞完了自己的开发环境,但是对编译器和语言标准的问题依旧感到茫然。下面这段程序可以输出当前所用的编译器和语言标准信息:
\lstinputlisting[caption=\texttt{standard\_and\_compiler.cpp}\label{lst:standard_and_compiler}]{code_in_book/1.2/standard_and_compiler.cpp}
这段代码,我们就不讲了,只是拿来运行就好。它的输出结果因你的开发环境而异\footnote{根据已知信息,Visual Studio 2022对于 \lstinline@__cplusplus@ 的输出结果总是 \lstinline@199711@,这可能是一个Bug。如果要让它输出正确结果,需要在编译命令行中添加 \texttt{/Zc:\_\_cplusplus} 指令才行。这对于初学者来说可能有点难,但是也不要担心,这个问题不会影响我们对于语言标准的正常使用,所以不这么做也没问题。}。对于我的开发环境来说,它的输出结果是\\\noindent\rule{\linewidth}{.2pt}\texttt{
Current C++ Standard: 201703\\
Current C++ Compiler: GCC
}\\\noindent\rule{\linewidth}{.2pt}\\
这说明我当前使用的C++标准是C++17。对于C++11以前的标准来说,它会显示 \lstinline@199711@;对于C++11, C++14, C++20和C++23来说,它们会分别显示 \lstinline@201103@, \lstinline@201402@, \lstinline@202002@ 和 \lstinline@202302@。读者可以对号入座,了解自己当前使用的C++语言标准。\par
如果读者使用的标准不是C++17,希望读者根据软件的使用说明,修改自己的语言标准;如果编译器本身不支持C++17,希望读者换成较新版本的编译器。如果读者使用的不是本书推荐的三种编译器之一,也希望读者能改用这三种编译器,以获得最佳的使用效果。\par
