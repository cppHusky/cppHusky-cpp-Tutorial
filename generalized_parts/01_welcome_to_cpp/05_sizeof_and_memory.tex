\section{\texttt{sizeof}与内存空间}
我们在之前总是反复提及,整型与整型之间是不同的,浮点型与浮点型之间也是不同的。对于整型来说,有符号与无符号是一个区别,而``容量''是另一个区别。如何理解这里的``容量''呢?这就要我们深入到``内存''中去,一探究竟。\par
\subsection*{什么是内存?}
张三正在用电脑打单机游戏,这时突然停电了。好不容易等到来电,张三心急如焚地打开电脑,刚准备继续玩,突然发现自己忘了存档,当前游戏进度还是昨天的进度。这让他非常不爽。\par
那么问题来了,明明程序中都有我们当前游戏的进度,为什么我们还要通过``存档''来把这些内容保存下来呢?这就涉及到内存和外存的区别了。\par
\textbf{非易失型存储器(Non-Volatile Memory, NVM)},俗称\textbf{外存},这种存储器内的信息不会随着电脑关机、供电中断等因素而丢失。但是,外存信息的访问速度慢,可能导致程序运行卡顿。张三的游戏存档就储存在外存当中。\par
\textbf{易失型存储器(Volatile Memory, VM)},俗称\textbf{内存},这种存储器内的信息会随着电脑关机、供电中断等因素消失。不过,内存信息的访问速度更快,程序运行更加流畅,所以计算机程序主要都是在内存当中运行的,偶尔才会与外存进行数据交换。在张三打游戏的过程中,游戏数据就存储于内存当中,只有通过存档才能把信息保存至外存。\par
正因如此,张三今天的游戏数据(储存于内存)丢失了,但昨天的存档(储存于外存)依然健在。\par
同样,我们写出来的程序在运行时也会在内存当中运转。我们定义的数据,也都保存在内存当中。\par
\subsection*{内存如何存储数据?}
存储器的内部结构很像是一大群排列紧密、有序的房屋,彼此都是相同的模样。每个房屋有8个同款的房间,每个房间都存储着0或1,看上去十分单调。但就是这种看似无趣的0和1的排列组合,却构成了我们眼里丰富多彩的信息。一部时长一百分钟的高清电影,大小7.07GB,换算过来是76亿个这样的房屋,由607亿个这样的0和1的排列组合而成。\footnote{严格说来,这部电影存储于外存。但内存与外存都有这样的特征,所以借来阐述并不为过。}\par
前文提到,我们可以用编码的方式来将多样的信息转换成10种阿拉伯数字的排列组合,又可以通过解码的方式将这些数据转换回去。而在存储器中,多样的信息都会被编码成0和1的排列组合。我们把每一位这样的0或1称为一个\textbf{比特(Binary digit, bit)}。\par
比特,如同生命中的细胞,化学中的原子,建成大厦的砖块和社会中的每个人,是构成信息最基本、最微小的单元。无数细胞构成了绚丽多彩的生命,无数原子构成了气象万千的世界,而无数比特则构成了我们眼中目不睱接的信息。\par
不过,存储器中是不允许``单细胞生物''或者``单原子分子''存在的。在存储器中,每8个比特构成一个\textbf{字节(Byte)}\footnote{历史上,每个字节包含多少个比特没有特定标准,因而在一定时期是混乱的。如今的标准规定每个字节包含八个比特,但可能有极少数例外。}。字节是度量信息的单位,也是可寻址\footnote{关于内存地址,我们会在第五章讲解。}的最小单元。\par
回到我们刚才的比喻。存储器的内部结构是排列有序的房屋。每个房屋代表一个字节,它是相对独立的最小可寻址单元。而每个房屋有8个房间,各自存储一比特的信息。图1.3可以帮助我们更直观地理解这种关系。\par
\begin{figure}[htbp]
    \centering
    \includegraphics[width=0.75\textwidth]{../images/generalized_parts/01_Memory_byte_and_bit.drawio.png}
    \caption{存储器、字节与比特}
\end{figure}
\subsection*{信息与容量}
香农\footnote{克劳德·艾尔伍德·香农(Claude Elwood Shannon),美国数学家、电子工程师、计算机科学家和密码学家,被誉为``信息论之父''。}是最早把``比特''的概念引入到信息技术中的人。在论文\footnote{\textit{A Mathematical Theory of Communication},于1948年发表于《贝尔实验室技术期刊》。这篇论文奠定了现代信息论的基础,被《科学美国人》誉为``信息时代的《大宪章》''。}的第一页,他就提出:$N$个比特可以表示$2^N$种状态。\footnote{原文:A device with two stable positions, such as a relay or a flip-flop circuit, can store one bit of information. $N$ such devices can store $N$ bits, since the total number of possible states is $2^N$ and $\log_2 2^N=N$.}\par
我们可以把一个数据的值看作一个\textbf{状态(State)}。当这个数据是 \lstinline@20@ 的时候,它的状态就是20;当这个数据的值是 \lstinline@13@ 的时候,它的状态就发生了改变,变为13。可以试想,如果我们有1个字节的数据,那么这8个比特一共可以表示的状态数量就是$2^8=256$个。换言之,它只能表示最多256种状态,也就只能编码256种信息。\par
C/C++中的 \lstinline@char@ 数据占用的内存空间是一个字节,所以它能表示256种信息。而ASCII码只有128种信息,所以用一个 \lstinline@char@ 足够表示一个ASCII字符。假如我们定义了一个 \lstinline@char@ 数据并初始化为 \lstinline@'A'@,那么它在内存中的8个比特就有了某个特定的排布方式(假设排布方式是 \lstinline@0100 0001@)。下一次,我给它另一个值,比如 \lstinline@'Z'@。改变了它的值,也就改变了它的状态,于是这8个比特的排布方式也就改变了(假设排布方式改为了 \lstinline@0101 1010@)。但是无论怎么修改它的值,这8个比特的信息永远都在256种可能性中打转,绝对不会出现第257种排布方式。\textbf{于是我们可以说,它的``容量''就是256。}\par
在大多数电脑上,\lstinline@int@ 类型的数据占用的内存空间是4字节\footnote{C++标准规定,\lstinline@int@ 类型的最低内存占用为2字节(16比特),但现在的大多数电脑都使用4字节。},也即32比特,所以它能表示$2^{32}=4'294'967'296$种状态。所有这些状态被用来给数字编码。对于 \lstinline@int@ 类型来说,它可以表示从$−2'147'483'648$到$2'147'483'647$的全部整数(含两端);而 \lstinline@unsigned@ 类型可以表示从$0$到$4'294'967'295$的全部整数(含两端)。\textbf{于是我们可以说,它的``容量''就是$4'294'967'296$。}\par
值得注意的是,虽然 \lstinline@signed int@(亦即 \lstinline@int@)和 \lstinline@unsigned int@(亦即 \lstinline@unsigned@)占用的内存空间大小相等,但它们能表示的数据范围是不相同的!这是一个``我们能够编码多少数据''和``我们编码了什么数据''的问题。对于 \lstinline@int@ 和 \lstinline@unsigned@ 来说,它们都只能编码$4'294'967'296$个数据,但因为它们编码的数据不尽相同,于是它们能表示的数据范围也就有所差异(但数据宽围的``宽度''是相等的)。我们会在精讲篇中更细致地谈论这个问题。\par
那么如果你想要表示比$4'294'967'295$还要大的数字呢?我们就要选择能容纳它的类型,比如 \lstinline@long@\footnote{在很多电脑上,\lstinline@long@ 类型和 \lstinline@int@ 类型一样占据4字节空间,这时就不能用 \lstinline@long@。} 或者 \lstinline@long long@,甚至是 \lstinline@unsigned long long@。总得说来,占据内存空间越多的类型,它能表示的数据范围也就越宽。\par
\subsection*{\texttt{sizeof}运算符}
在不同的计算机和开发环境中,同一数据类型占据的字节数可能并不相同,这就给我们造成了困扰。如何知道在自己电脑上每个类型占据多少内存空间呢?我们可以用 \lstinline@sizeof@ 运算符来搞定这个问题。\par
\lstinline@sizeof@ 是一个单目运算符\footnote{有关运算符的问题,我们将在第二章讲解。},可以接收一个类型或数据信息,并求出其内存占用。它的语法是
\begin{lstlisting}
    sizeof (<类型或数据>); //对类型求size必须套括号
    sizeof <数据>; //如果只对单个数据求size,可以不套括号
\end{lstlisting}
这只是单纯的求值。如果要输出我们求出来的值,我们就需要在前面加上 \lstinline@cout<<@。
\begin{lstlisting}
    cout << sizeof (char); //求char类型的内存占用
\end{lstlisting}
这样它就会输出 \lstinline@char@类型的内存空间大小。这个值在任何电脑上都是 \lstinline@1@。\par
我们还可以让它求出某个数据的内存空间占用,比如
\begin{lstlisting}
    cout << sizeof 2.71828; //求双精度浮点数2.71828的内存占用
\end{lstlisting}
在Coliru上,它的输出结果是 \lstinline@8@。\par
还可以输出我们定义的数据的内存空间大小,比如
\begin{lstlisting}
    unsigned long long ull = {10}; //定义一个unsigned long long数据并初始化为10
    cout << sizeof (ull); //求ull的内存占用,可以不加括号
\end{lstlisting}
在Coliru上,它的输出结果是 \lstinline@8@。\par
\subsection*{编译时行为与运行时行为}
\lstinline@sizeof@也是一个运算符。不过它与我们之前见到的加减乘除和模运算不同,它不是在程序运行时进行计算的,而是早在编译时就已经计算好了的。\par
我们在第1节中讲过,从写完代码到运行程序可以粗略分为三个过程:编译、链接和运行。有些值在编译的时候就能确定下来,比如 \lstinline@sizeof (a)@,无论它在内存中处于什么位置,也无论它的值是多少,它占据的内存空间的大小永远相同。(还记得吗,无论数据怎么变化,本质上只是若干个比特的排布方式发生了改变,但这个数据仍然占据同样尺寸的内存空间)\par
所以 \lstinline@sizeof@ 在编译时就已经计算出来了,运行时就无需再浪费时间去计算 \lstinline@sizeof@;而四则运算和取模,它们必须要在运行时才能确定结果,所以编译时不会进行计算。\par
C++11标准引入了 \lstinline@constexpr@ 语法,它扩展了编译时行为的范围。比如说
\begin{lstlisting}
    int a = {3}, b = {5};
    int c = {a+b}; //a+b在运行时计算
\end{lstlisting}
假如使用 \lstinline@constexpr@ 呢?
\begin{lstlisting}
    constexpr int a {3}, b {5};
    constexpr int c {a+b}; //a和b都是constexpr,所以a+b将在编译时计算
\end{lstlisting}
这说明,\lstinline@constexpr@ 改变了编译器处理数据的方式,使得一些数据可以在编译时就被计算出来。\par
本章只对编译时行为、运行时行为和 \lstinline@constexpr@ 作简要介绍,后面我们还会深入讲解。\par
