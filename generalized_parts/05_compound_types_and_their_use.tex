\chapter{复合类型及其使用}
在前面的章节中,我们讲了基本数据类型。不同类型数据的使用可以帮助我们解决一些小规模的问题;再搭配函数的使用,可以让我们的代码更成体系,更有逻辑。\par
\textbf{复合类型(Compound type)}的出现让函数的功能大放异彩。借助复合类型,我们可以实现更加复杂的功能,比如数据的批量处理(数组),函数的实参修改(指针和引用),以及动态内存分配。很多技术含量较高的代码必须借助复合类型才能实现,所以本章所介绍的内容有着承前启后的作用。\par
本章主要介绍指针、引用和数组相关的知识;至于结构体、联合体和类,我们留到下一章再讲解。\par
在本章,我们也将使用 \lstinline@typeid@ 和 \lstinline@std::is_same@ 等运算符或函数,来帮助我们梳理复合类型与基本类型之间的关系,它们是我们处理类型问题时相当有用的工具。\par
\import{05_composite_types_and_their_use/}{01_pointer.tex}
\import{05_composite_types_and_their_use/}{02_constant_pointer_and_pointer_to_constant.tex}
\import{05_composite_types_and_their_use/}{03_lvalue_reference_and_passing_arguments.tex}
\import{05_composite_types_and_their_use/}{04_one_dimensional_arrays.tex}
\import{05_composite_types_and_their_use/}{05_string_with_arrays.tex}
\import{05_composite_types_and_their_use/}{06_array_of_pointer_and_pointer_to_array.tex}
\import{05_composite_types_and_their_use/}{07_multidimensional_arrays_and_higher_order_pointers.tex}
\import{05_composite_types_and_their_use/}{08_dynamic_memory_allocation.tex}