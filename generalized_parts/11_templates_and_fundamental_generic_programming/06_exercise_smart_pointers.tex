\section{实操:智能指针}
初学者在进行动态内存分配及管理之后,常常会在不用的时候忘记回收——其实老手也蛮容易犯这种错误的。好在,C++的析构函数为我们回收动态内存提供了方便。对象中分配的任何动态内存,都可以靠析构函数来回收,不会有漏网之鱼。换句话说,通过析构函数,我们可以把动态生存期的东西变成自动生存期的(只要这个对象是自动生存期的),同时又能保留``动态改变大小''的可能性。\par
C++中的智能指针也是为了实现这个功能而设计的。智能指针可以指向动态内存空间,并改变指向(虽然它也可以指向静态的内存,但是我们最好别这样做)。当这个智能指针的生存期结束时,它的析构函数会把其指向的动态内存空间一并回收。\par
C++17有三种智能指针:\lstinline@std::unique_ptr@ 禁止不同智能指针指向相同地址;\lstinline@std::shared_ptr@ 允许不同智能指针指向相同地址;\lstinline@std::weak_ptr@ 与 \lstinline@std::shared_ptr@ 相似,但是它主要用于跟踪记录对象,而不能像真的指针那样访问对象。它们都定义在 \lstinline@memory@ 库中。\par
本节,我们就来用类模板及其特化,尝试实现一个简化版的 \lstinline@std::unique_ptr@ 智能指针。对于 \lstinline@unique_ptr@ 来说,虽然我们可以通过一个指针值来进行构造(比如传入一个 \lstinline@new@ 的返回值),但是它的严谨用法应该是在构造时调用 \lstinline@std::make_unique@ 函数。
\begin{lstlisting}
    std::unique_ptr<int> p1 {new int{3}}; //不够严谨的用法
    std::unique_ptr<int> p2 {std::make_unique<int>{3}} //严谨用法
\end{lstlisting}
但是限于读者的知识储备和接受能力,我们还是把这个功能实现得简单点吧。\par
而对于 \lstinline@std::shared_ptr@ 来说,不同智能指针对象可以指向同一地址,这就会在析构时带来麻烦——C++绝对不允许我们对未经分配(包括已经回收的)的动态内存和任何静态内存进行 \lstinline@delete@/\lstinline@delete[]@ 操作。如果不加检测地回收动态内存,很容易导致程序直接崩溃。\par
为了防止重复回收等问题,\lstinline@std::unique_ptr@ 和 \lstinline@std::shared_ptr@ 都有一个检测机制。它们通过 \lstinline@std::unordered_map@ 来记录``指针-保存次数''这个键-值对。这样,如果保存次数超过 \lstinline@1@,就说明除这个指针以外还有其它智能指针指向该地址,析构时不应回收动态内存;如果保存次数为 \lstinline@1@,就说明没有其它智能指针指向该地址了,析构该指针时应当顺便回收它。\par
\lstinline@std::unique_ptr@ 也是用 \lstinline@std::unordered_map@ 来实现的。不过若是按照标准库的写法来,我们还需要设计一个 \lstinline@hash@,这对于读者来说还是太麻烦了。所以我们到时候会用简单一点的方法来设计。\par
接下来我们就来看一下智能指针支持的主要功能,并选择一部分来实现。智能指针的成员函数不多,但是比前面那些 \lstinline@std::array@ 等内容更加抽象,注意事项也很杂,所以这个实操项目绝非易事。\par
\subsection*{功能简介}
\lstinline@std::unique_ptr@ 是一个类模板。它有一个针对 \lstinline@T[]@ 类型的特化。这个原因也很简单,毕竟由 \lstinline@new[]@ 分配的动态内存空间必须经由 \lstinline@delete[]@ 来回收,因此单对象类型与数组类型的操作方式天然地不同。
\begin{lstlisting}
template<
    class T,
    class Deleter = std::default_delete<T>
> class unique_ptr; //主模板
template <
    class T,
    class Deleter
> class unique_ptr<T[], Deleter>; //特化
\end{lstlisting}
至于这里的 \lstinline@Deleter@,我们不去深究,也不去使用。只管设计一个模板参数 \lstinline@T@ 就行了。\par
\lstinline@std::unique_ptr@ 的构造函数可以接收 \lstinline@T*@ 类型的指针作为参数,或者是进行移动构造。鉴于移动构造要放在精讲篇中介绍,我们在这里就不讲解了。至于拷贝构造函数,对于 \lstinline@unique_ptr@ 来说,拷贝构造会违反``一个地址只能由一个指针所指''的规则,所以我们不仅不应该设计拷贝构造函数,还应该把默认拷贝构造函数删除,详见后文。\par
析构函数负责回收这个智能指针申请到的动态内存。这个指针可能指向 \lstinline@nullptr@;不过不用担心,\lstinline@delete nullptr@ 这样的语句不会引发任何问题(也没有任何意义)。\par
\lstinline@std::unique_ptr@ 同样不支持用赋值运算符进行内容拷贝。我们应该把拷贝赋值运算符删除。至于移动赋值,这里不讲。\par
\lstinline@release@ 成员函数可以把一个智能指针指向的地址值(以返回值形式)交给外界。此后,这个智能指针的指向变为 \lstinline@nullptr@,而原来的动态内存空间仍保留,且必须由外界来回收,该对象不再对此负责。
\begin{lstlisting}
    std::unique_ptr<int> p {new int{2}};
    int *q {p.release()}; //此后p指向nullptr
    delete q; //需要由外界负责回收
\end{lstlisting}\par
\lstinline@get@ 成员函数和 \lstinline@release@ 很像,它也可以返回指针值。区别在于,调用 \lstinline@get@ 函数不会改变指针的指向。比如说很多标准库函数只能接收指针参数,不能接收智能指针参数。这时我们就必须通过 \lstinline@get@ 取指针值再传入函数中。
\begin{lstlisting}
    std::unique_ptr<char[]> str {new char[10]{'A', 'B', 'C', '\0'}};
    std::cout << std::strlen(str.get()); //需要通过str.get()取char*才能传参
\end{lstlisting}\par
\lstinline@reset@ 成员函数的作用是重置一个新的动态内存。对于主模板来说没有什么特别的,但对于特化 \lstinline@T[]@ 来说就要复杂一些,实现它的 \lstinline@reset@ 成员需要一些元编程的知识——所以我们换一个简化版本来实现,它只支持改变数组长度。
\begin{lstlisting}
    std::unique_ptr<int[]> p {new int[3]{1, 2, 3}}; //原来的动态数组长度为3
    p.reset(new int[1000]); //重置成长度为1000的更大数组
\end{lstlisting}
顺便提醒,这个重置操作是不会复制内容的,所以使用之前请做好备份。\par
\lstinline@operator bool@,或者说是到 \lstinline@bool@ 类型的自定义转换函数,可以表示一个智能指针是否指向实际内容(不是的话,就是指向 \lstinline@nullptr@)。这个很容易理解。\par
\lstinline@swap@ 是我们的老朋友了。对于 \lstinline@std::unique_ptr@ 来说,\lstinline@swap@ 函数既有成员函数版本,又有非成员函数版本。它们的效果相同。无论哪个版本,它们都只能严格地与同类型的智能指针进行交换。\par
比较运算符没什么特别的。简化起见,我们就不实现了。\par
至于内容访问,单对象版本的指针支持使用取内容运算符 \lstinline@*@ 和指针的成员访问运算符 \lstinline@->@;而数组版本的只支持使用下标运算符 \lstinline@[]@,不支持使用 \lstinline@*@ 和 \lstinline@->@。\par
部分编译器为智能指针提供了输出功能,也就是 \lstinline@operator<<(std::ostream&,/*...*/)@,它不属于C++17标准的一部分,不过实现起来很简单,我们也把它写一下吧。\par
\subsection*{规划}
类模板及其特化的声明如下:
\begin{lstlisting}
namespace user {
template<typename T>
class unique_ptr; //定义类模板unique_ptr
template<typename T>
class unique_ptr<T[]>; //定义特化unique_ptr<T[]>
\end{lstlisting}\par
我们会为每一个 \lstinline@user::unique_ptr<T>@ 类和 \lstinline@user::unique_ptr<T[]>@ 类设计一个静态的 \lstinline@user::unordered_set<T*>@ 成员,用来记录 \lstinline@T*@ 的使用情况。在新赋值时,如果发现集合中已经存有相同值的指针,那就说明有别的智能指针已经指向它了,这个赋值操作应该被打断。我们可以用抛出异常的方式来提醒用户程序出问题了;不过我们还没学异常处理,所以就用 \lstinline@nullptr@ 之类的替代方案——我们到后面再说。\par
对于类模板设计来说,我们往往还会命名一些``成员类型'',比如这样:
\begin{lstlisting}
template<typename T>
class unique_ptr {
public:
    using pointer = T*; //pointer就是T*的别名,也是unique_ptr的一个成员类型
    //...
\end{lstlisting}
这样一来,我们就可以在类内用 \lstinline@pointer@ 指代 \lstinline@T*@ 类型;而在类外,我们也可以用 \lstinline@unique_ptr<T>::pointer@ 指代 \lstinline@T*@ 类型。这样的好处当然是统一、直观、易于修改。在之前设计 \lstinline@user::array@ 类的例子中,我也建议读者改用这样的语法。\par
这里我们只需要两个成员类型,一个是 \lstinline@pointer=T*@,一个是 \lstinline@element_type=T@。\par
接下来我们就把类模板及其特化的基本代码写出来:
\begin{lstlisting}
//namespace user
template<typename T>
class unique_ptr { //分配动态单对象,使用unique_ptr<T>语法
public:
    using pointer = T*; //通过using定义成员类型pointer,公有
    using element_type = T; //定义成员类型element_type
private:
    pointer _ptr {nullptr}; //pointer类型的私有成员,成员类型pointer必须定义在前
    static std::unordered_set<pointer> record;
public:
    //...待补充
};
template<typename T>
class unique_ptr<T[]> { //特化,分配动态数组,使用unique_ptr<T[]>语法
public:
    using pointer = T*; //T[]只是意味着分配T元素的动态数组,其指针类型还是T*
    using element_type = T;
private:
    pointer _ptr {nullptr};
    static std::unordered_set<pointer> record;
public:
    //...待补充
};
\end{lstlisting}
看上去可能有点复杂,但是跟着注释,想必还是不难看懂的。\par
接下来我们先声明主模板的成员函数,再看特化类模板。\par
\subsubsection*{主模板的成员函数}
主模板用来动态分配单个对象,所以它应该接收 \lstinline@new@ 分配的指针而不是 \lstinline@new[]@。它的构造函数只能接收指针或移动构造。我们不设计移动构造,只需要把拷贝构造函数删除就行了。
\begin{lstlisting}
public:
    explicit unique_ptr(pointer = {nullptr}); //接收pointer类参数,或nullptr
    unique_ptr(const unique_ptr&) = delete; //删除拷贝构造函数
\end{lstlisting}
函数删除的语法就像这样,在声明的末尾写 \lstinline@=delete@ 就好。被删除的函数在任何场合都不能调用。\par
析构函数没什么可说的,声明就行了。
\begin{lstlisting}
public:
    ~unique_ptr(); //析构函数
\end{lstlisting}\par
拷贝赋值应当删除;移动赋值就不写了。如果我们有更改指向的需求,应该用 \lstinline@reset@ 等更安全的函数。
\begin{lstlisting}
public:
    unique_ptr& operator=(const unique_ptr&) = delete; //删除拷贝赋值函数
\end{lstlisting}\par
接下来是 \lstinline@release@, \lstinline@get@ 成员函数。它们的操作逻辑不同,但返回值都是 \lstinline@pointer@ 类型。
\begin{lstlisting}
public:   
    pointer release(); //release函数会把指向改为nullptr
    pointer get()const; //get函数不会改变成员;至于透过get()改变内容,那是允许的
\end{lstlisting}\par
\lstinline@reset@ 成员函数不需要返回值,设为 \lstinline@void@ 就好。它接收一个参数,我们也可以为它设置默认值 \lstinline@nullptr@。这样一来,如果我们调用 \lstinline@reset()@ 的话,就可以既回收内存又把指向置为空(\lstinline@release@ 可不会回收内存)。
\begin{lstlisting}
public:
    void reset(pointer = {nullptr}); //重置指向
\end{lstlisting}\par
\lstinline@operator bool@ 的实现很简单,我就连定义一起写了。
\begin{lstlisting}
public:
    operator bool()const { return _ptr != nullptr; } //表明指向是否非空
\end{lstlisting}\par
\lstinline@swap@ 的成员函数版本不需要是函数模板——它只要求能和同类进行交换就可以;否则就很容易出问题。
\begin{lstlisting}
public:
    void swap(unique_ptr&);
\end{lstlisting}
至于非成员函数,它需要是函数模板,但作用仅限于对 \lstinline@user::unique_ptr@ 的类型进行抽象。我们先在类外声明它。
\begin{lstlisting}
template<typename T>
void swap(unique_ptr<T>&, unique_ptr<T>&);
\end{lstlisting}
然后在类内声明一个友元就行。
\begin{lstlisting}
    template<typename U>
    friend void swap(unique_ptr<U>&, unique_ptr<U>&);
\end{lstlisting}
注意,这里不要不小心写成 \lstinline@T@ 了,否则会引发模板参数名称冲突。\par
比较运算符就不写了。取内容运算符很简单,我们直接定义出来就行。
\begin{lstlisting}
public:
    element_type& operator*()const { return *get(); } //注意返回值得是引用
\end{lstlisting}
至于指针的成员访问运算符,它的重载比较特别。当重载这个运算符时,它的返回值只能是裸指针或者对象。如果是裸指针,那么会通过它的``指针的成员访问运算符''来取成员。我们先写代码,然后再讲它。
\begin{lstlisting}
public:
    pointer operator->()const { return get(); } //注意返回类型得是裸指针或对象
\end{lstlisting}
假如说我们有一个 \lstinline@user::unique_ptr<std::vector>@ 对象,名为 \lstinline@ptr@,那么当我写 \lstinline@ptr->resize(0)@ 时,这段代码会被解释成 \lstinline@(ptr.operator->())->resize(0)@,也就是 \lstinline@(ptr.get())->resize(0)@。这是C++运算符重载中一个极特殊的例子。\par
最后还有一个输出操作。它是一个友元函数模板,我们需要在类外声明它:
\begin{lstlisting}
template<typename T>
std::ostream& operator<<(std::ostream&, const unique_ptr<T>);
\end{lstlisting}
然后在类内声明友元。
\begin{lstlisting}
    template<typename U>
    friend std::ostream& operator<<(std::ostream&, const unique_ptr<U>&);
\end{lstlisting}\par
\subsubsection*{模板特化的成员函数}
看完了主模板的成员函数,我们来看看模板特化的成员函数。\par
对于大部分成员函数来说,它们和主模板没有声明上的区别。\lstinline@reset@ 函数本来是有区别的,但是当我们简化之后,它们也是一样的。\par
非成员函数 \lstinline@swap@ 和 \lstinline@operator<<(std::ostream&,/*...*/)@ 看似需要重载,但其实也不用。不信的话你试试,当你写了一份函数模板重载之后会发现它们的函数体可以完全一样:
\begin{lstlisting}
template<typename T>
void user::swap(unique_ptr<T> &lhs, const unique_ptr<T> &rhs) {
    std::swap(lhs._ptr, rhs._ptr); //
}
template<typename T>
void user::swap(unique_ptr<T[]> &lhs, const unique_ptr<T[]> &rhs) {
    std::swap(lhs._ptr, rhs._ptr); //完全一样
}
\end{lstlisting}
既然函数体都完全一样,那么我们没必要对 \lstinline@T[]@ 类型加以特殊对待;前一个函数模板完全可以涵盖后一个函数模板的操作,所以我们就把它省了吧。\par
其实需要专门介绍的就一个,那就是主模板没有的下标运算符重载。
\begin{lstlisting}
//专属于特化模板的成员函数
public:
    element_type& operator[](std::size_t i)const { return _ptr[i]; }
    //下标运算符,访问第i个元素
\end{lstlisting}\par
其余的定义基本和主模板相同,除了没有重载 \lstinline@*@ 和 \lstinline@->@ 运算符。这里我们就不再赘述了,读者可以到后面阅读本例的完整代码。\par
\subsection*{实现}
首先不要忘了在头文件中用 \lstinline@inline@ 定义静态成员。
\begin{lstlisting}
template<typename T>
inline std::unordered_set<
    typename user::unique_ptr<T>::pointer //typename关键字说明它是一个类型
> user::unique_ptr<T>::record;
template<typename T>
inline std::unordered_set<
    typename user::unique_ptr<T[]>::pointer
> user::unique_ptr<T[]>::record;
\end{lstlisting}
无论主模板还是特化,它们的静态成员都需要分别提供定义。这可不像非成员函数那样,有共享同一套函数模板的余地。\par
这里还有一个需要提及的点:如果我们要传入模板参数 \lstinline@user::unique_ptr<T>::pointer@ 的话,编译器是不能判断这个语句代表一个``类型''的。为了让编译器明确知道它是一个类型名,我们需要加之以 \lstinline@typename@ 关键字。这样编译器就不会误会。\par
接下来我们来实现一下尚未定义的函数。先看非成员函数。这些都不难,读者跟着注释就可以看明白。
\begin{lstlisting}
template<typename T>
void user::swap(unique_ptr<T> &lhs, unique_ptr<T> &rhs) {
    std::swap(lhs._ptr, rhs._ptr); //只要交换它们底层的指针就行了
}
template<typename T>
std::ostream& user::operator<<(
    std::ostream &out,
    const user::unique_ptr<T> &ptr
) {
    return out << ptr._ptr; //直接输出ptr内部的指针值,这也正是我们想要的
    //特别地,如果是char/char[]型的话,将输出字符串哦
}
\end{lstlisting}
对于构造函数来说,如果它接收的参数是一个未被记录的量,那么我们允许它保存该地址值。但是如果 \lstinline@record@ 中已有记录,我们就不能保存它,直接置为 \lstinline@nullptr@便好——这正是我们在定义中给出的默认成员初始值。实际上,标准库中提供的 \lstinline@std::unique_ptr@ 是没有这项检测的,所以我们不要随意地卡它的bug,否则后果难料\footnote{笔者注:事实上,没有几个标准库功能可以真正做到事无巨细地把所有异常情况都考虑到,并加以防范——即便可以,这样做也可能并不经济。作为使用者,很多雷区应当由我们在使用过程中加以留意,而不是完全依赖于内置的功能。}。
\begin{lstlisting}
template<typename T>
user::unique_ptr<T>::unique_ptr(pointer ptr) {
    if (record.find(ptr) == record.end()) { //先在记录当中寻找是否有ptr
        _ptr = ptr;
        record.insert(ptr); //如果没找到,_ptr可以安全赋值,并记录这个值
    } //否则什么也不做
}
\end{lstlisting}\par
析构函数则相反,如果我们回收了某个动态内存空间,我们应当从 \lstinline@record@ 中移除它,表示``当前没有指针指向此处''。
\begin{lstlisting}
template<typename T>
user::unique_ptr<T>::~unique_ptr() {
    if (record.find(_ptr) != record.end()) { //如果在记录中找到了_ptr
        delete _ptr; //回收动态内存
        record.erase(_ptr); //清除记录
    }
}
\end{lstlisting}
接下来是 \lstinline@release@ 和 \lstinline@get@ 函数,请读者注意这里 \lstinline@auto@ 关键字的使用。
\begin{lstlisting}
template<typename T>
auto user::unique_ptr<T>::release() -> pointer { //用auto来方便我们写返回类型
    pointer tmp {_ptr}; //临时变量存值,以便返回
    _ptr = nullptr; //_ptr指向空
    return tmp;
}
template<typename T>
auto user::unique_ptr<T>::get()const -> pointer {
    return _ptr;
}
\end{lstlisting}
我们曾讲过,在返回值处,编译器无法判断函数名之前出现的一个单独的 \lstinline@pointer@ 代表什么,所以这样写是不行的:
\begin{lstlisting}
pointer user::unique_ptr<T>::release() -> pointer { /*...*/ } //错误
\end{lstlisting}
因此我们只能写全名:
\begin{lstlisting}
user::unique_ptr<T>::pointer user::unique_ptr<T>::release() { /*...*/ }
\end{lstlisting}
这样虽然正确,但是十足地麻烦。所以我们就用这种 \lstinline@auto@ 配合尾随返回类型的写法,这样编译器看到 \lstinline@pointer@ 之后就会默认去找 \lstinline@user::unique_ptr<T>@ 的成员类型,于是我们写起来也就方便很多了。\par
\begin{lstlisting}
template<typename T>
void user::unique_ptr<T>::reset(pointer ptr) {
    if (record.find(_ptr) != record.end()) { //先清理旧的指针
        delete _ptr;
        record.erase(_ptr);
        _ptr = nullptr; //注意把_ptr置空,否则可能会出现野指针
    }
    if (record.find(ptr) == record.end()) { //如果新的地址值不在记录当中
        _ptr = ptr;
        record.erase(_ptr);
    } //否则什么也不做
}
\end{lstlisting}\par
最后的 \lstinline@swap@ 成员函数很简单,我也不啰嗦了。
\begin{lstlisting}
template<typename T>
void user::unique_ptr<T>::swap(unique_ptr &ptr) {
    std::swap(_ptr, ptr._ptr);
}
\end{lstlisting}\par
至于 \lstinline@T[]@ 特化的版本,它们没什么特别的,大部分代码都可以原封不动地抄过去。
\begin{lstlisting}
template<typename T>
user::unique_ptr<T[]>::unique_ptr(pointer ptr) {
    if (record.find(ptr) == record.end()) { //同上,没什么特别的
        _ptr = ptr;
        record.insert(ptr);
    }
}
\end{lstlisting}\par
唯独有一点要注意!回收动态数组应当用 \lstinline@delete[]@ 而不是 \lstinline@delete@。
\begin{lstlisting}
template<typename T>
user::unique_ptr<T[]>::~unique_ptr() {
    if (record.find(_ptr) != record.end()) {
        delete[] _ptr; //一定不要搞错了!
        record.erase(_ptr);
    }
}
\end{lstlisting}
除了这个小问题以外,其余的都没什么需要特别注意,我把代码放在最后了。\par
\subsection*{完整代码}
以下是本工程的完整代码。我对内容做了重排和整合。其中类模板定义在\texttt{.hpp}文件中,而成员函数主要定义在\texttt{.tpp}文件中。\par
\lstinputlisting[caption=\texttt{unique\_ptr.hpp}]{code_in_book/11.8-11.9/unique_ptr.hpp}
\lstinputlisting[caption=\texttt{Definition.tpp}]{code_in_book/11.8-11.9/Definition.tpp}\par
在代码的末尾还是需要提醒读者,这里的内容都是对 \lstinline@std::unique_ptr@ 功能的部分实现,它不能替代 \lstinline@std::unique_ptr@ 来用。