\section{函数模板的实例化与特化}
我们讲过``模板与实例''的关系。一个类就是一个广义的模板,这个类的对象则是这一模板所产生的实例,它们拥有相同的成员,表示同类事物。同样地,函数模板也是一个抽象的概念,它不代表任何实体。只有在\textbf{实例化(Instantiation)}之后产生的实例——也就是一个个具体的函数,它们才是真正意义上的实体。\par
\subsection*{实例化}
以 \lstinline@swap@ 为例,我们在上一节中写了两个 \lstinline@swap@ 函数模板,它们的声明分别是
\begin{lstlisting}
namespace user{
template<typename T>
void swap(T&, T&);
template<typename T, std::size_t N>
void swap(T(&)[N], T(&)[N]);
}; //end namespace user
\end{lstlisting}\par
当我们调用函数时,编译器会根据我们提供的实参的类型,推导出模板参数。
\begin{lstlisting}
    double a {1}, b{2};
    user::swap(a, b); //编译器推导出T=double
\end{lstlisting}
这说明程序需要一个 \lstinline@T=double@ 的函数,那么编译器就会生成一个这样的函数,它的全名叫作 \lstinline@user::swap<double>@,尖括号里的内容就是它的模板实参;参数列表是 \lstinline@(double&,double&)@。同理,如果我们传入的是两个 \lstinline@int@ 数据,那么编译器就会推导出 \lstinline@T=int@,然后生成一个这样的函数,它的全名叫作 \lstinline@user::swap<int>@,参数列表是 \lstinline@(int&,int&)@。\par
我们可以用全名来调用它,这样就不用麻烦编译器做模板参数推导了(毕竟,它的推导结果可能并不是我们想要的样子)。
\begin{lstlisting}
    int a {1}, b {2};
    user::swap<int>(a, b); //指名道姓地调用user::swap<int>
\end{lstlisting}
这样,编译器就不需要做模板参数推导,就可以直接生成一份 \lstinline@std::swap<int>@ 函数实例,并在之后的操作中直接调用它。\par
你可以在全名的模板参数中省略一部分\footnote{但是必须从右到左,比如你可以把 \lstinline@fun<int,int>@ 省略成 \lstinline@fun<int>@,但不能省略成 \lstinline@fun<,int>@。},乃至全部模板参数,乃至把尖括号都省了。总之,只要你提供的函数参数信息足够编译器用来做参数推导,那就足够了。
\begin{lstlisting}
    char s1[10]{ "Husky" }, s2[10]{ "Samoyed" };
    user::swap<char, 10>(s1, s2); //使用全名调用
    user::swap<char>(s1, s2); //指定char,另一个模板参数交由编译器推导
    user::swap<>(s1, s2); //两个模板参数都交由编译器推导
    user::swap(s1, s2); //编译器没有找到同名的非模板函数,所以开始考虑模板函数
\end{lstlisting}
在没有其它同名函数/模板定义的情况下,这四个语句的执行效果完全一致。\par
以上过程都属于\textbf{隐式实例化(Implicit Instantiation)}。隐式实例化可以概括为:编译器根据我们在代码中的实际需要,生成对应的函数实例。\par
除了隐式实例化以外,我们还可以进行\textbf{显式实例化(Explicit Instantiation)}。显式实例化的语法很简单,一个 \lstinline@template@ 关键字加上正常的函数格式就行了,其中的函数名必须用全名。
\begin{lstlisting}
template void user::swap<int>(int&,int&);
//显式实例化user::swap<int>(int&,int&)
template void user::swap<double>(double&,double&);
//显式实例化user::swap<double>(double&,double&)
\end{lstlisting}\par
一般来说,显式实例化的工作都可以用隐式实例化来替代;而且只要我们用的是全名,那么隐式实例化不会出现任何问题。所以当我们写一些小规模代码时,完全没有进行显式实例化的必要。不过,显式实例化并非毫无用处,读者可以参阅\href{https://stackoverflow.com/q/2351148/22112284}{Explicit template instantiation - when is it used? - Stack Overflow},这里的内容相当详细,本书就不讲了。\par
\subsection*{函数模板的重载}
先不急看特化,我来插一小节函数模板重载的知识。函数模板重载方面的知识其实相当杂,但是我们只拣常见问题来说。\par
函数模板重载是比函数重载更上位的概念,对于函数模板来说,无论是在函数形参列表方面,还是在模板形参列表方面,都可以进行重载。\par
\begin{lstlisting}
namespace user {
template<typename T> //模板形参为<typename>
T max(T a, T b) { //函数形参为(T,T)
    return a > b ? a : b;
}
template<typename T> //模板形参为<typename>
T max(T a, T b, T c) { //函数形参为(T,T,T)
    return a > b ? (a > c ? a : c) : (b > c ? b : c);
}
template<typename T, std::size_t N1, std::size_t N2>
//模板形参为<typename,std::size_t,std::size_t>
T max(const T (&a)[N1], const T (&b)[N2]) {
//函数形参为(const T(&)[N1],const T(&)[N2])
    T suma {std::accumulate(a, a+N1, 0)}; //调用了algorithm库中的求和函数
    T sumb {std::accumulate(b, b+N2, 0)};
    return suma > sumb ? suma : sumb;
}
};
\end{lstlisting}\par
代码11.1中的两个 \lstinline@user::swap@ 函数之间也有重载关系。当我们不加模板实参地调用 \lstinline@user::swap@ 时,编译器会根据一套复杂的重载决议规则,选择最匹配的函数模板,并调用相应的函数实例。这套规则的简化版(足够对付日常编程中的常见情况了)是这样的:
\begin{enumerate}
    \item 如果同一个名字下有一个参数类型相匹配\footnote{这个匹配过程可能涉及隐式类型转换,比如数组到指针的类型转换等。}的非模板函数,那么将优先调用非模板函数。
    \item 如果这个名字之下都是函数模板,那么将优先调用函数形参更特殊\footnote{实际上,重载函数模板之间存在一种偏序关系,编译器将根据它们之间的偏序关系来决定各个重载的优先级。}的模板重载。举例来说,\lstinline@fun(T*)@ 就比 \lstinline@fun(T)@ 更特殊,因为前者的限制比后者多,虽然两者都可以接收各种指针\footnote{包括一阶指针,高阶指针,数组指针等各种指针。},但是前者不能接收指针以外的类型(比如非指针的数据)——所以 \lstinline@fun(T*)@ 更特殊,而 \lstinline@fun(T)@ 更一般。
    \item 在找到了最适合的函数模板之后,如果有特化,调用其特化;如果无特化,按照模板生成对应的实例,并调用之。
\end{enumerate}\par
至于更多有关函数模板重载的问题,我们到了精讲篇再谈。\par
\subsection*{特化}
函数模板为编译器生成函数实例提供了比较一般的方法,但是我们难免遇到一些特殊情况,需要特殊处理。举个例子吧,我们可以写一个 \lstinline@less@ 函数,它可以比较两个参数的大小,并返回``参数1是否小于参数2''的 \lstinline@bool@ 结果。实现这个功能的最佳方案是用类模创建一个函数对象,不过我们退而求其次使用函数模板也未为不可。
\begin{lstlisting}
namespace user {
template<typename T>
bool less(T lhs, T rhs) {
    return lhs < rhs;
}
};
\end{lstlisting}
通过这个,我们就可以比较任何两个数、两个对象(只要这个对象有重载小于号)、两个指针的小于号关系了。\par
不过有一种情况比较特殊,那就是 \lstinline@bool less<const char*>(const char*,const char*)@,即两个 \lstinline@const char*@ 比较大小。事实上,我们一般都是把 \lstinline@char*@ 当作字符串来用的,而不是单纯指向单个字符的指针——这也就是为什么 \lstinline@std::ostream::operator<<@ 在遇到 \lstinline@char*@ 类型操作数时会输出一个字符串而不是地址值。\par
当我们比较字符串的大小时,我们也不希望它仍然在进行指针大小比较,而是应该按照字典序来进行比较。所以我们应当对于 \lstinline@T=const char*@ 的情形进行特殊处理,这就是\textbf{模板特化(Template specialization)}。\par
\begin{lstlisting}
namespace user {
template<> //一定不要遗漏!
bool less<const char*>(const char *lhs, const char *rhs) {
    return std::lexicographical_compare( //需要algorithm库
        lhs, lhs + std::strlen(lhs), //需要cstring库
        rhs, rhs + std::strlen(rhs)
    );
}
};
\end{lstlisting}\par
读者一定要分清实例化和特化的语法。模板特化的 \lstinline@template@ 之后是根着一对尖括号的\footnote{对于类模板来说,部分特化时还可以在尖括号中添加其它的模板参数,详见后文。函数模板没有部分特化,故只能是一对尖括号。},而实例化没有这样的尖括号。
\begin{lstlisting}
template<> bool less<const char*>(const char*, const char*); //声明一个特化
template<>
bool less<const char*>(const char *lhs, const char *rhs) { //定义一个特化
    //...
}
template bool less<const char*>(const char*, const char*); //实例化
\end{lstlisting}\par
特化的函数模板仍然属于函数模板的一个实例,这种关系是不会变的。\par
其实除了定义模板特化之外我们还有一种方法,就是定义非模板函数。非模板函数不是函数模板的实例,它们是重载关系。
\begin{lstlisting}
bool less(const char *lhs, const char *rhs) { //定义非模板函数
    return std::lexicographical_compare(
        lhs, lhs + std::strlen(lhs),
        rhs, rhs + std::strlen(rhs)
    );
}
\end{lstlisting}\par
既然函数重载和函数模板特化都能实现我们的目的,那么我们选择哪个呢?一般我们选择重载而不是特化,因为函数模板特化有太多的局限性,而且在重载决议中的优先级也不如函数重载高,总之容易引发许多我们意想不到的问题。所以编程界的共识是倾向于使用重载而非特化。如果读者想要了解这方面的更多内容,可以参考 \href{http://www.gotw.ca/publications/mill17.htm}{Why Not Specialize Function Templates?} 这篇文章。\par
\subsection*{实操:自制 \lstinline@is_same@ 函数模板}
让我们再来看看 \lstinline@std::is_same@ 吧。我们曾用 \lstinline@std::is_same@ 来判断两个类是否相同——这也是它作为一个元编程类的本来用途。不过我们偶尔也拿它来判断两个对象是否同类,或者某个对象是否是某个类。这时我们就必须对那个对象取 \lstinline@decltype@,然后才能放在模板参数中进行判断,像这样:
\begin{lstlisting}
    int (*p)[3];
    std::cout << std::is_same<decltype(*p), int(&)[3]>::value;
\end{lstlisting}\par
看上去这也挺麻烦的,我们干脆自己写一个 \lstinline@user::is_same@ 函数模板,并加以优化,让我们可以方便地进行类与类、类与对象、对象与对象的类型比较。
\lstinputlisting[caption=\texttt{is\_same.hpp}]{../code_in_book/11.2-11.3/is_same.hpp}
这里有三个函数模板重载,对于第一个重载来说,我们无法通过函数参数提供任何信息,编译器不能进行模板参数推导,所以我们必须把两个模板参数都写出来才行;对于第二个重载,编译器可以根据我们提供的实参来推导第二个模板参数,所以我们可以省去第二个模板参数;对于第三个重载,编译器可以根据我们提供的两个实参来推导所有模板参数,所以我们可以把它们全都省去了。\par
下面就展示这三个模板的调用过程:当我们需要比较两个类时,就把这两个类都放到模板参数中;当我们需要比较两个对象时,就把这两个对象都放到函数参数中;当我们需要比较一个类和一个对象时,就把类放到模板参数中,对象放到函数参数中。\par
\lstinputlisting[caption=\texttt{main.cpp}]{../code_in_book/11.2-11.3/main.cpp}
读者可能注意到了一些新的问题:这里我们判断 \lstinline@*p1@ 和 \lstinline@int(&)[3]@ 的时候,结果并没有如同预期那样输出 \lstinline@1@。这是因为,\lstinline@user::is_same@ 接收参数 \lstinline@p1@ 时会进行数组到指针的隐式类型转换,所以到了函数内部,\lstinline@obj@ 形参的类型已经是 \lstinline@int*@ 了,我们只有提供 \lstinline@int*@ 参数才能得到 \lstinline@true@ 的返回值。\par
我们可以添加一个专为数组而设计的函数模板,从而解决这个类型转换的问题。
\begin{lstlisting}
namespace user { //namespace是可以拼接的
template<typename T1, typename T2, std::size_t N>
constexpr bool is_same(const T2(&arr)[N]) { //为防止数组-指针转换而设计的模板
    return std::is_same<T1, T2[N]>::value;
}
};
\end{lstlisting}
不过,这种传递函数参数的方式还是会导致部分信息的损失,比如让我们无法判断引用类型。\par
总之,如果需要最严格的类型判断,模板参数无疑是比函数参数更为严格和可靠的。对于模板参数来说,一是一,二是二,类型参数不会进行转换。\par
