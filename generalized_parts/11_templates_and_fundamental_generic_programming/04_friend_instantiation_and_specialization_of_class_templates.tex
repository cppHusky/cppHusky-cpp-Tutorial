\section{类模板的友元、实例化与特化}
\subsection*{类模板中的友元}
让我们以 \lstinline@get@ 和 \lstinline@swap@ 为例了解一下友元。\par
首先读者不要忘记,\lstinline@user::array@ 不是一个类,而是一个类模板;\lstinline@user::get@ 也不是一个函数,而是一个函数模板。\textbf{把一个函数模板作为友元,不等于把一个函数实例作为友元。}对于 \lstinline@user::get@ 来说,我们只希望它的 \lstinline@user::get<T,N>@ 实例能够一对一地访问 \lstinline@user::array<T,N>@ 的私有成员,而不要求它对于另一个类 \lstinline@user::array<U,M>@ 的私有成员也有访问权限。所以我们写成这样就好:
\begin{lstlisting}
namespace user{

}; //end namespace user
\end{lstlisting}