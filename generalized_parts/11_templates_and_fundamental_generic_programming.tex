\chapter{模版与泛型编程基础}
本章将介绍C++的两个重要部分:\textbf{泛型编程(Generic programming)}与\textbf{标准模版库(Standard template library, STL)}。其中STL部分不会细讲,留到精讲篇再详细阐述。\par
我们在第四章中已简要了解过函数模版,又在之后的章节中反复用到 \lstinline@std::vector@ 等类模版。在本章,我将带领读者从函数模版到类模版,系统性地学习泛型编程的基本知识,并在这之后完成一个``指能指针''的实操练习。\par
而在本章的末尾,我会带读者了解一些STL的基本知识——尤其是迭代器。它是指针的延伸,但其作用远比指针更加丰富。\par
\import{11_templates_and_fundamental_generic_programming/}{01_function_templates.tex}
\import{11_templates_and_fundamental_generic_programming/}{02_instantiation_and_specialization_of_function_templates.tex}
\import{11_templates_and_fundamental_generic_programming/}{03_class_templates.tex}
\import{11_templates_and_fundamental_generic_programming/}{04_instantiation_and_specialization_of_class_templates.tex}
\import{11_templates_and_fundamental_generic_programming/}{05_exercise_smart_pointers.tex}
\import{11_templates_and_fundamental_generic_programming/}{06_introduction_to_STL.tex}