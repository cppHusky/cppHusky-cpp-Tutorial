\chapter{模板与泛型编程基础}
{\kaishu \large ``泛型编程是关于算法和数据结构的抽象和分类。它的灵感来自于高德纳,而不是类型理论。它的目标是系统地、增量地构建有用、高效和抽象的算法与数据结构。''\footnote{原文:Generic programming is about abstracting and classifying algorithms and data structures. It gets its inspiration from Knuth and not from type theory. Its goal is the incremental construction of systematic catalogs of useful, efficient and abstract algorithms and data structures.}}
\begin{flushright}——亚历山大·斯捷潘若夫\end{flushright}\par
本章将介绍C++的两个重要部分:\textbf{泛型编程(Generic programming)}与\textbf{标准模板库(Standard template library, STL)}。其中STL部分不会细讲,留到精讲篇再详细阐述。\par
我们在第四章中已简要了解过函数模板,又在之后的章节中反复用到 \lstinline@std::vector@ 等类模板。在本章,我将带领读者从函数模板到类模板,系统性地学习泛型编程的基本知识,并在这之后完成一个``指能指针''的实操练习。\par
而在本章的末尾,我会带读者了解一些STL的基本知识——尤其是迭代器。它是指针的延伸,但其作用远比指针更加丰富。\par
\import{11_templates_and_fundamental_generic_programming/}{01_function_templates_and_constexpr.tex}
\import{11_templates_and_fundamental_generic_programming/}{02_instantiation_and_specialization_of_function_templates.tex}
\import{11_templates_and_fundamental_generic_programming/}{03_class_templates.tex}
\import{11_templates_and_fundamental_generic_programming/}{04_friend_instantiation_and_specialization_of_class_templates.tex}
\import{11_templates_and_fundamental_generic_programming/}{05_exercise_smart_pointers.tex}
\import{11_templates_and_fundamental_generic_programming/}{06_introduction_to_STL.tex}